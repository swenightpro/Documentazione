% ===================================================================
%  TEMPLATE DOCUMENTO PROGETTO (File Unico)
%  Gruppo: NightPRO
%  Progetto: Ingegneria del Software 2025/2026
%
%  Basato sulla struttura:
%  - configurazione.tex
%  - intestazione_titolo.tex
%  - indice.tex
%  - informazioni.tex
%  - odg.tex
%  - diario.tex
%  - decisioni.tex
%  - todo.tex
% ===================================================================

% -------------------------------------------------------------------
%  SEZIONE: configurazione.tex
% -------------------------------------------------------------------
\documentclass[a4paper, 11pt, oneside]{scrartcl} % Classe KOMA-Script

% --- Pacchetti Fondamentali ---
\usepackage[utf8]{inputenc}     % Codifica UTF-8
\usepackage[T1]{fontenc}        % Font encoding moderno
\usepackage[italian]{babel}     % Lingua italiana
\usepackage{lmodern}            % Font "Latin Modern"

% --- Grafica e Layout ---
\usepackage{graphicx}           % Per includere immagini
\usepackage[a4paper, top=2.5cm, bottom=3cm, left=2.5cm, right=2.5cm]{geometry} % Margini
\usepackage{fancyhdr}           % Per header e footer personalizzati
\usepackage{microtype}          % Migliora la tipografia
\usepackage[svgnames]{xcolor}   % Colori

% --- Utility ---
\usepackage{booktabs}           % Tabelle più professionali
\usepackage{enumitem}           % Per personalizzare liste
\usepackage{hyperref}           % Rende i link cliccabili
\hypersetup{
    colorlinks=true,
    linkcolor=DarkBlue,
    filecolor=DarkBlue,      
    urlcolor=DarkBlue,
    citecolor=DarkBlue,
    pdftitle={Documento Progetto - NightPRO},
    pdfauthor={Gruppo NightPRO},
}

% --- Path per le Immagini ---
% (Assicurati che ci sia una cartella 'immagini' accanto a questo file)
\graphicspath{ {immagini/} }

% ===================================================================
%  IMPOSTAZIONE HEADER E FOOTER
% ===================================================================
\pagestyle{fancy}
\fancyhf{} % Pulisce tutti i campi
\fancyhead[L]{NightPRO - Progetto Ingegneria del Software}
\fancyhead[R]{Anno Accademico 2025/2026}
\fancyfoot[C]{\thepage} % Numero di pagina al centro in basso
\renewcommand{\headrulewidth}{0.4pt} % Linea sottile sotto l'header
\renewcommand{\footrulewidth}{0pt}

% ===================================================================
%  INIZIO DEL DOCUMENTO
% ===================================================================
\begin{document}

% -------------------------------------------------------------------
%  SEZIONE: intestazione_titolo.tex
% -------------------------------------------------------------------
\thispagestyle{empty}
\begin{titlepage}
    \centering
    
\begin{figure}
    \centering
    \includegraphics[width=0.4\textwidth]{src/immagini/logo.jpg}
\end{figure}

    \vfill
    
    {\small UNIVERSITÀ DEGLI STUDI DI PADOVA \par}
    {\small CORSO DI LAUREA IN INFORMATICA (L-31) \par}
    \vspace{0.5cm}
    {\large Corso di Ingegneria del Software \par}
    {\small Anno Accademico 2025/2026 \par}


    
    \vfill
    
    {\Huge \bfseries Verbale di Riunione \par}
    
    \vspace{1cm}
    
    % Inserisci qui il titolo specifico del documento
    {\Large \itshape Verbale Esterno del 22 ottobre 2025 \par} 
    
    \vfill
    
    {\Large \bfseries Gruppo: NightPRO \par}
    \vspace{0.5cm}
    {\large \href{mailto:swe.nightpro@gmail.com}{swe.nightpro@gmail.com} \par}
    
    \vfill
    
    % Inserisci qui la data di redazione del documento
    {\large Data: 20 ottobre 2025 \par}

\end{titlepage}

% -------------------------------------------------------------------
%  SEZIONE: indice.tex
% -------------------------------------------------------------------
\newpage
\tableofcontents % Genera l'indice
\pagestyle{fancy} % Riattiva lo stile di pagina da qui in poi

% -------------------------------------------------------------------
%  SEZIONE: informazioni.tex
% -------------------------------------------------------------------
\newpage
\section{Informazioni Generali}

\subsection{Componenti del Gruppo}
Elenco dei membri del gruppo di lavoro NightPRO.

\begin{table}[h!]
\centering
\begin{tabular}{@{}llc@{}}
\toprule
\textbf{Cognome} & \textbf{Nome} & \textbf{Matricola} \\
\midrule
Biasuzzi & Davide & 2111000 \\
Bilato & Leonardo & 2071084 \\
Zanella & Francesco & 2116442 \\
Romascu & Mihaela-Mariana & 2079726 \\
Ogniben & Michele & 2042325 \\
Perozzo & Samuele & 2110989 \\
Ponso & Giovanni & 2000558 \\
\bottomrule
\end{tabular}
\caption{Componenti del Gruppo NightPRO.}
\end{table}


\subsection{Dettagli Riunione}
\begin{itemize}
    \item \textbf{Data:} 22 ottobre 2025
    \item \textbf{Ora:} 14:30 - 15:00
    \item \textbf{Luogo:} Meeting Zoom
    \item \textbf{Partecipanti:} Zanella Francesco, Perozzo Samuele, Ponso Giovanni, Bilone Leonardo, Gianluca Carlesso (Ergon)
    \item  \textbf{Redatto da: } Francesco Zanella
    \item  \textbf{Verificato da:} Davide Biasuzzi
    \item \textbf{Versione: } 1.0
\end{itemize}


% -------------------------------------------------------------------
%  SEZIONE: odg.tex (Ordine del Giorno)
% -------------------------------------------------------------------
\newpage
\section{Ordine del Giorno (Agenda)}
\begin{itemize}
    \item Richiedere al proponente di Ergon di fornire risposta alle nostre domande e registrare le risposte ottenute per permetterne una successiva valutazione.
\end{itemize}

% -------------------------------------------------------------------
%  SEZIONE: diario.tex (Diario della riunione)
% -------------------------------------------------------------------

\newpage
\section{Diario della Riunione}
\begin{itemize}
    \item Discussione con il referente di Ergon sul prodotto richiesto, guidato dalle nostre domande
\end{itemize}

\footnotesize  % Font più piccolo
\setlength{\tabcolsep}{8pt}  % Riduce lo spazio tra le colonne
\begin{tabular}{p{0.45\textwidth}p{0.45\textwidth}}
\textbf{Domande} & \textbf{Risposte} \\
\\
\textbf{1. Requisiti Minimi sulle Modalità di Input}

Il capitolo 3 ("Tecnologie, suggerimenti e risorse utili") consiglia di avviare lo sviluppo con una o due modalità di input per il PoC. In ottica di prodotto finale, vorremmo chiarire se esista un vincolo sul numero minimo di modalità che il sistema completo dovrà obbligatoriamente gestire. 
& 
Per il Proof of Concept è sufficiente concentrarsi su una o due modalità di input. Per quanto riguarda il prodotto finale, non esiste un numero minimo obbligatorio di modalità da implementare. La parte testuale è da considerarsi fondamentale, mentre l'integrazione audio rappresenta un'aggiunta auspicabile ma non vincolante. La gestione delle immagini è consigliato farla per ultima data la difficoltà e l'approccio diverso. \\
\\

\textbf{2. Architettura proposta del sistema - Gestione Input da Canali Eterogenei}

In riferimento al paragrafo "Architettura proposta del sistema", il primo layer del sistema gestisce dati da canali eterogenei. Per definire correttamente i confini del nostro sviluppo, ci sarebbe utile capire se dobbiamo ipotizzare l'esistenza di un'infrastruttura a monte che ci fornisca i dati, o se la progettazione di connettori specifici per i vari canali rientri nell'ambito del progetto.
&
Il recupero dei dati per la generazione degli ordini può essere considerato una funzionalità opzionale ma non obbligatoria per il progetto. È consigliato l'utilizzo di una webapp per fare in modo che il cliente verifichi la correttezza dell'ordine \\
\\

\textbf{3. Validazione e Arricchimento Dati}

Sempre nel paragrafo "Architettura proposta del sistema" vorremmo approfondire il livello di automazione atteso per quanto riguarda il layer 6 - "Validazione e Arricchimento Dati". È da prevedere un intervento umano supervisionato (human-in-the-loop) o l'obiettivo è tendere a una validazione completamente automatica?
&
Il sistema dovrebbe "parcheggiare" gli ordini quando si presentano ambiguità nei prodotti o quando si presentano richieste poco chiare da parte dell'utente. In questi casi, il sistema deve fornire il modo di poter contattare direttamente un operatore. \\
\\

\textbf{4. Monitoraggio e Feedback}

Il progetto menziona un retraining periodico. Vorremmo comprendere se tale processo sia da implementare come un ciclo automatizzato basato su feedback o come un'operazione da supervisionare manualmente.
&
Il miglioramento delle prestazioni dovrebbe avvenire in modo automatico, basandosi sui feedback degli utenti o sulla percentuale di correttezza degli ordini generati dal sistema. \\
\\

\textbf{5. Preferenze Tecnologiche}

Il capitolo 3 elenca diversi framework per l'interfaccia utente (es. .NET Blazor, React). Considerando che Ergon sviluppa soluzioni ERP proprietarie, esiste uno stack tecnologico di riferimento interno o una preferenza con cui sarebbe opportuno allineare lo sviluppo?
&
Per la gestione dell'intelligenza artificiale è consigliato l'utilizzo di Python. Per il resto dello stack tecnologico non ci sono vincoli particolari, le tecnologie consigliate sono comunque indicate nel documento di riferimento. \\
\end{tabular}
\normalsize


% -------------------------------------------------------------------
%  SEZIONE: todo.tex (Attività da svolgere)
% -------------------------------------------------------------------
\newpage
\section{Attività da Svolgere (To-Do)}

\begin{table}[h!]
\centering
\begin{tabular}{@{}lll@{}}
\toprule
\textbf{Attività} & \textbf{Assegnatario/i} \\
\midrule
Analizzare le risposte fornite dal referente per capire se \\il progetto \textit{Smart Order} è adatto a noi & \textit{Tutto il gruppo singolarmente}\\\\
Stesura verbale riunione 22/10/25 & Zanella Francesco \\
\bottomrule
\end{tabular}
\caption{Riepilogo task assegnati.}
\end{table}

\vspace{2cm}
\begin{flushright}
\makebox[6cm]{\hrulefill}\\
{\small Firma}
\end{flushright}
\end{document}