\documentclass[a4paper, 11pt, oneside]{scrartcl} % Classe KOMA-Script

% --- Pacchetti Fondamentali ---
\usepackage[utf8]{inputenc}     % Codifica UTF-8
\usepackage[T1]{fontenc}        % Font encoding moderno
\usepackage[italian]{babel}     % Lingua italiana 
\usepackage{lmodern}            % Font "Latin Modern"

% --- Grafica e Layout ---
\usepackage{graphicx}           % Per includere immagini
\usepackage{currfile}
\graphicspath{{src/immagini/}{\currfiledir contenuti/}{\currfiledir contenuti/immagini/}}

\usepackage[a4paper, top=2.5cm, bottom=3cm, left=2.5cm, right=2.5cm]{geometry} % Margini
\usepackage{fancyhdr}           % Per header e footer personalizzati
\usepackage{microtype}          % Migliora la tipografia
\usepackage[svgnames]{xcolor}   % Colori

% --- Utility ---
\usepackage{booktabs}           % Tabelle più professionali
\usepackage{enumitem}           % Per personalizzare liste
\usepackage{hyperref}           % Rende i link cliccabili
\hypersetup{
    colorlinks=true,
    linkcolor=DarkBlue,
    filecolor=DarkBlue,      
    urlcolor=DarkBlue,
    citecolor=DarkBlue,
    pdftitle={Documento Progetto - NightPRO},
    pdfauthor={Gruppo NightPRO},
}

% ===================================================================
%  IMPOSTAZIONE HEADER E FOOTER
% ===================================================================
\pagestyle{fancy}
\fancyhf{} % Pulisce tutti i campi
\fancyhead[L]{NightPRO - Progetto Ingegneria del Software}
\fancyhead[R]{Anno Accademico 2025/2026}
\fancyfoot[C]{\thepage} % Numero di pagina al centro in basso
\renewcommand{\headrulewidth}{0.4pt} % Linea sottile sotto l'header
\renewcommand{\footrulewidth}{0pt}

% ===================================================================
%  INIZIO DEL DOCUMENTO
% ===================================================================
\begin{document}

% -------------------------------------------------------------------
%  SEZIONE: intestazione_titolo.tex
% -------------------------------------------------------------------
\thispagestyle{empty}
\begin{titlepage}
    \centering
    
\begin{figure}
    \centering
    \includegraphics[width=0.4\textwidth]{logo.jpg}
\end{figure}

    \vfill
    
    {\small UNIVERSITÀ DEGLI STUDI DI PADOVA \par}
    {\small CORSO DI LAUREA IN INFORMATICA (L-31) \par}
    \vspace{0.5cm}
    {\large Corso di Ingegneria del Software \par}
    {\small Anno Accademico 2025/2026 \par}


    
    \vfill
    
    {\Huge \bfseries Verbale di Riunione \par}
    
    \vspace{1cm}
    
    {\Large \itshape Verbale Interno del 27 ottobre 2025 \par} 
    
    \vfill
    
    {\Large \bfseries Gruppo: NightPRO \par}
    \vspace{0.5cm}
    {\large \href{mailto:swe.nightpro@gmail.com}{swe.nightpro@gmail.com} \par}
    
    \vfill
  
    {\large Data: 27 ottobre 2025 \par}

\end{titlepage}

% -------------------------------------------------------------------
%  SEZIONE: indice.tex
% -------------------------------------------------------------------
\newpage
\tableofcontents % Genera l'indice
\pagestyle{fancy} % Riattiva lo stile di pagina da qui in poi

% -------------------------------------------------------------------
%  SEZIONE: informazioni.tex
% -------------------------------------------------------------------
\newpage
\section{Informazioni Generali}

\subsection{Componenti del Gruppo}
Elenco dei membri del gruppo di lavoro NightPRO.
\begin{table}[h!]
\centering
\begin{tabular}{@{}llc@{}}
\toprule
\textbf{Cognome} & \textbf{Nome} & \textbf{Matricola} \\
\midrule
Biasuzzi & Davide & 2111000 \\
Bilato & Leonardo & 2071084 \\
Zanella & Francesco & 2116442 \\
Romascu & Mihaela-Mariana & 2079726 \\
Ogniben & Michele & 2042325 \\
Perozzo & Samuele & 2110989 \\
Ponso & Giovanni & 2000558 \\
\bottomrule
\end{tabular}
\caption{Componenti del Gruppo NightPRO.}
\end{table}

\subsection{Dettagli Riunione}
\begin{itemize}
    \item \textbf{Data:} 27 ottobre 2025
    \item \textbf{Ora:} 16:30 - 18:15
    \item \textbf{Luogo:} Google Meet
    \item \textbf{Partecipanti:} Tutti i membri del gruppo tranne Michele Ogniben
    \item \textbf{Redatto da: } Davide Biasuzzi
    \item \textbf{Verificato da:} Francesco Zanella
    \item \textbf{Versione: } 1.0
\end{itemize}


% -------------------------------------------------------------------
%  SEZIONE: odg.tex (Ordine del Giorno)
% -------------------------------------------------------------------
\newpage
\section{Ordine del Giorno (Agenda)}
\begin{itemize}
    \item[1.] Analisi strumenti di project management (JIRA vs GitHub Projects)
    \item[2.] Definizione e setup repository e organizzazione GitHub
    \item[3.] Aggiornamento scelta capitolato (Zucchetti vs VIMAR)
    \item[4.] Pianificazione e assegnazione task per documenti di presentazione (Lettera, Preventivo, Valutazione)
    \item[5.] Discussione struttura file LaTeX
    \item[6.] Pianificazione prossimi incontri e scadenze
\end{itemize}

% -------------------------------------------------------------------
%  SEZIONE: diario.tex (Diario della riunione)
% -------------------------------------------------------------------
\newpage
\section{Diario della Riunione}

\subsection{Analisi strumenti di project management (JIRA vs GitHub Projects)}
Si è discusso sull'utilizzo di JIRA o GitHub Projects per la gestione delle attività. 
Giovanni Ponso ha mostrato un esempio di un progetto passato (MTSS) gestito su GitHub, illustrando la gestione delle \textit{issue} (task), l'assegnazione ai membri del team, l'uso di tag multipli, la definizione di \textit{milestones} e la tracciabilità delle scadenze.
Viste le funzionalità, si è deciso di procedere con GitHub Projects.

\subsection{Definizione e setup repository e organizzazione GitHub}
È stata creata l'organizzazione GitHub ufficiale del gruppo. 
Samuele Perozzo ha sottolineato l'importanza di inserire una descrizione chiara e dettagliata per ogni commit.
Francesco Zanella e Giovanni Ponso si occuperanno di creare la repository definitiva per la documentazione, migrando la struttura di prova già funzionante (incluso il convertitore LaTeX). 
È stato specificato che i verbali firmati (PDF) dovranno essere inseriti manualmente nella posizione corretta nel repository, affiancando quelli compilati automaticamente.

\subsection{Aggiornamento scelta capitolato (Zucchetti vs VIMAR)}
Si è preso atto che un incontro con Zucchetti (proponente C6) è al momento fuori discussione, data l'elevata richiesta da parte di altri gruppi e la nostra propensione per il capitolato C8. 
Si è quindi deciso di scartare questa opzione e di concentrarsi sul capitolato VIMAR. 
Si è deciso di procedere fissando un colloquio con VIMAR per approfondire la proposta.

\subsection{Pianificazione e assegnazione task per documenti di presentazione}
Stabilita la scelta di VIMAR, si è deciso di procedere immediatamente con la stesura dei documenti necessari per la presentazione del gruppo all'azienda:
\begin{itemize}
    \item Lettera di Presentazione
    \item Preventivo dei Costi ed Impegno Orario
    \item Valutazione dei Capitolati
\end{itemize}
I compiti per la redazione della prima bozza sono stati così suddivisi:
\begin{itemize}
    \item \textbf{Lettera di Presentazione:} Samuele Perozzo, Francesco Zanella, Michele Ogniben.
    \item \textbf{Valutazione dei Capitolati:} Davide Biasuzzi, Giovanni Ponso, Mariana Romascu.
    \item \textbf{Preventivo dei Costi ed Impegno Orario:} Leonardo Bilato e Mariana Romascu.
\end{itemize}

\subsection{Discussione struttura file LaTeX}
Giovanni Ponso ha illustrato la possibilità di suddividere i documenti LaTeX in file separati (per sezione) per una migliore organizzazione, o in alternativa di mantenerli in un unico file per semplicità operativa. Si è deciso di mantenere la struttura unica per evitare eccessiva complessità.

\subsection{Pianificazione prossimi incontri e scadenze}
È stata fissata una nuova riunione interna per il 29 ottobre 2025, con l'obiettivo di revisionare i documenti prodotti. 
L'obiettivo è inviare la mail con la documentazione a VIMAR entro il 30 ottobre.
Davide Biasuzzi è stato incaricato di scrivere l'email a VIMAR e di redigere il presente verbale.

% -------------------------------------------------------------------
%  SEZIONE: decisioni.tex (Decisioni prese)
% -------------------------------------------------------------------
\newpage
\section{Decisioni Prese}

\begin{enumerate}
    \item Adottare GitHub Projects come strumento di task management.
    \item Creare la repository GitHub definitiva per la documentazione.
    \item Richiedere una descrizione chiara e dettagliata per ogni commit.
    \item Scartare il capitolato Zucchetti e procedere con il capitolato VIMAR.
    \item Fissare un colloquio con VIMAR.
    \item Suddividere la preparazione dei documenti di presentazione (Lettera, Valutazione, Preventivo) tra i membri.
    \item Fissare la prossima riunione interna per il 29 ottobre 2025.
\end{enumerate}

% -------------------------------------------------------------------
%  SEZIONE: todo.tex (Attività da svolgere)
% -------------------------------------------------------------------
\newpage
\section{Attività da Svolgere (To-Do)}

\begin{table}[h!]
\centering
\begin{tabular}{@{}lll@{}}
\toprule
\textbf{Attività} & \textbf{Assegnatario/i} & \textbf{Scadenza} \\
\midrule
Redazione Verbale Riunione 27/10 & Davide Biasuzzi & 28/10/25 \\
Inviare mail a VIMAR per colloquio & Davide Biasuzzi & 28/10/25 \\
Creazione Progetto GitHub (Kanban, issues) & Samuele Perozzo, Giovanni Ponso & 29/10/25 \\
Creazione repository documentazione definitiva & Francesco Zanella, Giovanni Ponso & 29/10/25 \\
Redazione "Lettera di Presentazione" (Bozza) & S. Perozzo, F. Zanella, M. Ogniben & 29/10/25 \\
Redazione "Valutazione dei Capitolati" (Bozza) & D. Biasuzzi, G. Ponso, M. Romascu & 29/10/25 \\
Redazione "Preventivo dei Costi" (Bozza) & Leonardo Bilato, M. Romascu & 29/10/25 \\
\bottomrule
\end{tabular}
\caption{Riepilogo task assegnati.}
\end{table}

\end{document}
