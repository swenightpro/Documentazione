\documentclass[a4paper, 11pt, oneside]{scrartcl} % Classe KOMA-Script

% --- Pacchetti Fondamentali ---
\usepackage[utf8]{inputenc}     % Codifica UTF-8
\usepackage[T1]{fontenc}        % Font encoding moderno
\usepackage[italian]{babel}     % Lingua italiana 
\usepackage{lmodern}            % Font "Latin Modern"

% --- Grafica e Layout ---
\usepackage{graphicx}           % Per le immagini
\graphicspath{{../../../assets/}}
\usepackage[a4paper, top=2.5cm, bottom=3cm, left=2.5cm, right=2.5cm]{geometry} % Margini
\usepackage{fancyhdr}           % Per header e footer personalizzati
\usepackage{microtype}          % Migliora la tipografia
\usepackage[svgnames]{xcolor}   % Colori

% --- Utility ---
\usepackage{booktabs}           % Tabelle più professionali
\usepackage{enumitem}           % Per personalizzare liste
\usepackage{hyperref}           % Rende i link cliccabili
\hypersetup{
    colorlinks=true,
    linkcolor=DarkBlue,
    filecolor=DarkBlue,      
    urlcolor=DarkBlue,
    citecolor=DarkBlue,
    pdftitle={Documento Progetto - NightPRO},
    pdfauthor={Gruppo NightPRO},
}

% ===================================================================
%  IMPOSTAZIONE HEADER E FOOTER
% ===================================================================
\pagestyle{fancy}
\fancyhf{} % Pulisce tutti i campi
\fancyhead[L]{NightPRO - Progetto Ingegneria del Software}
\fancyhead[R]{Anno Accademico 2025/2026}
\fancyfoot[C]{\thepage} % Numero di pagina al centro in basso
\renewcommand{\headrulewidth}{0.4pt} % Linea sottile sotto l'header
\renewcommand{\footrulewidth}{0pt}

% ===================================================================
%  INIZIO DEL DOCUMENTO
% ===================================================================
\begin{document}

% -------------------------------------------------------------------
%  SEZIONE: intestazione_titolo.tex
% -------------------------------------------------------------------
\thispagestyle{empty}
\begin{titlepage}
    \centering
    
\begin{figure}
    \centering
    \includegraphics[width=0.4\textwidth]{logo.png}
\end{figure}

    \vfill
    
    {\small UNIVERSITÀ DEGLI STUDI DI PADOVA \par}
    {\small CORSO DI LAUREA IN INFORMATICA (L-31) \par}
    \vspace{0.5cm}
    {\large Corso di Ingegneria del Software \par}
    {\small Anno Accademico 2025/2026 \par}
    
    \vfill
    
    {\Huge \bfseries Verbale di Riunione \par}
    
    \vspace{1cm}
    
    {\Large \itshape Verbale Interno del 29 ottobre 2025 \par} 
    
    \vfill
    
    {\Large \bfseries Gruppo: NightPRO \par}
    \vspace{0.5cm}
    {\large \href{mailto:swe.nightpro@gmail.com}{swe.nightpro@gmail.com} \par}
    
    \vfill
  
    {\large Data: 29 ottobre 2025 \par}

\end{titlepage}

% -------------------------------------------------------------------
%  SEZIONE: indice.tex
% -------------------------------------------------------------------
\newpage
\tableofcontents % Genera l'indice
\pagestyle{fancy} % Riattiva lo stile di pagina da qui in poi

% -------------------------------------------------------------------
%  SEZIONE: informazioni.tex
% -------------------------------------------------------------------
\newpage
\section{Informazioni Generali}

\subsection{Componenti del Gruppo}
Elenco dei membri del gruppo di lavoro NightPRO.
\begin{table}[h!]
\centering
\begin{tabular}{@{}llc@{}}
\toprule
\textbf{Cognome} & \textbf{Nome} & \textbf{Matricola} \\
\midrule
Biasuzzi & Davide & 2111000 \\
Bilato & Leonardo & 2071084 \\
Zanella & Francesco & 2116442 \\
Romascu & Mihaela-Mariana & 2079726 \\
Ogniben & Michele & 2042325 \\
Perozzo & Samuele & 2110989 \\
Ponso & Giovanni & 2000558 \\
\bottomrule
\end{tabular}
\caption{Componenti del Gruppo NightPRO.}
\end{table}

\subsection{Dettagli Riunione}
\begin{itemize}
    \item \textbf{Data:} 29 ottobre 2025
    \item \textbf{Ora:} 16:30 - 18:30
    \item \textbf{Luogo:} Google Meet
    \item \textbf{Partecipanti:} Tutti i membri del gruppo tranne Michele Ogniben
    \item \textbf{Redatto da: } Davide Biasuzzi
    \item \textbf{Verificato da:} Francesco Zanella
    \item \textbf{Versione: } 1.0
\end{itemize}


% -------------------------------------------------------------------
%  SEZIONE: odg.tex (Ordine del Giorno)
% -------------------------------------------------------------------
\newpage
\section{Ordine del Giorno (Agenda)}
\begin{itemize}
    \item[1.] Presentazione e discussione sito GitHub Pages
    \item[2.] Analisi tempistiche ruoli Analista/Verificatore e definizione criteri di rotazione
    \item[3.] Gestione documenti di presentazione (Registro Modifiche, allineamento layout, logo)
    \item[4.] Definizione procedure di verifica e approvazione documenti
    \item[5.] Aggiornamento e chiarimenti capitolato VIMAR (visione FAQ)
    \item[6.] Aggiornamento task su GitHub
\end{itemize}

% -------------------------------------------------------------------
%  SEZIONE: diario.tex (Diario della riunione)
% -------------------------------------------------------------------
\newpage
\section{Diario della Riunione}

\subsection{Presentazione e discussione sito GitHub Pages}
Giovanni Ponso e Leonardo Bilato presentano lo stato di avanzamento del sito GitHub Pages del gruppo, attualmente su repository privata.
È stato mostrato lo script Python per l'ordinamento e la gestione dei documenti, che include il riconoscimento di formati di data diversi (es. \texttt{YYYY-MM-DD} o \texttt{YYYY/MM/DD}).

Sono state discusse le seguenti modifiche e migliorie da implementare prima del rilascio ufficiale:
\begin{itemize}
    \item Aggiunta informazioni sul gruppo e sul progetto.
    \item Predisposizione menu di navigazione per le fasi future (PB, RTB).
    \item Centratura del logo.
    \item Implementazione funzione dark mode.
    \item Aggiunta sfondo di fallback.
    \item Miglioramento della logica di orientamento utente nel sito.
\end{itemize}
Durante la riunione sono state effettuate modifiche live alla grafica. Il gruppo approva il lavoro svolto e si decide di pubblicare il sito entro il 30 ottobre 2025.

\subsection{Gestione documenti di presentazione}
\subsubsection{Allineamento layout e Logo}
Davide Biasuzzi segnala di aver allineato (prima della riunione) i layout dei file LaTeX dei tre documenti di presentazione (Lettera, Preventivo, Valutazione) per renderli coerenti.
Giovanni Ponso segnala che attualmente si sta utilizzando una versione obsoleta del logo. Si occuperà di aggiornare il file del logo e ricompilare tutti i documenti che lo utilizzano.

\subsubsection{Registro delle Modifiche}
Si decide di implementare un "Registro delle Modifiche" per i tre documenti di presentazione. Tale registro dovrà essere posizionato all'inizio di ogni documento, prima dell'indice.
L'incarico per la creazione del registro è così suddiviso:
\begin{itemize}
    \item \textbf{Valutazione dei Capitolati:} Davide Biasuzzi.
    \item \textbf{Preventivo dei Costi:} Mihaela-Mariana Romascu.
    \item \textbf{Lettera di Presentazione:} Francesco Zanella.
\end{itemize}

\subsubsection{Verifica e Approvazione}
Si stabilisce che la verifica e l'approvazione finale dei tre documenti (Lettera, Preventivo, Valutazione) deve avvenire entro il 30 ottobre 2025.
Per garantire oggettività, si decide che i ruoli di redattore, verificatore e approvatore non possono essere ricoperti dalle stesse persone per lo stesso documento.

\subsection{Analisi tempistiche ruoli (Analista/Verificatore)}
Davide Biasuzzi solleva la questione delle tempistiche assegnate ai ruoli di Analista e Verificatore nel documento "Preventivo", ritenute leggermente sottostimate rispetto al ruolo di Programmatore.
Si discute la necessità di definire meglio i criteri per la rotazione dei ruoli. Si concorda su una rotazione periodica e flessibile, da organizzare tramite incontri dedicati.
Leonardo Bilato si occuperà di rivedere le stime e aggiungere i criteri di rotazione nel documento.
Di conseguenza, sarà necessario aggiornare anche il costo totale nel "Preventivo dei Costi".

\subsection{Aggiornamento capitolato VIMAR}
Si prende atto della risposta ricevuta da VIMAR, la quale, data l'impossibilità di fissare un incontro a breve, ha inviato un documento FAQ contenente le domande poste dagli altri gruppi.
Il gruppo ha preso visione del documento. La preferenza del gruppo rimane confermata sul capitolato C8 (Ergon).

\subsection{Aggiornamento Task Management}
Si conferma che le decisioni prese e i task assegnati durante la riunione sono stati inseriti nel backlog (Agile task workflow) dell'organizzazione GitHub e assegnati ai rispettivi responsabili.

% -------------------------------------------------------------------
%  SEZIONE: decisioni.tex (Decisioni prese)
% -------------------------------------------------------------------
\newpage
\section{Decisioni Prese}

\begin{enumerate}
    \item Approvare la versione attuale del sito GitHub Pages e procedere con la pubblicazione entro il 30/10/25.
    \item Aggiornare il file del logo (versione corretta) e ricompilare tutti i documenti.
    \item Rivedere le stime orarie per i ruoli di Analista e Verificatore nel "Preventivo dei Costi".
    \item Definire e inserire nel "Preventivo" i criteri per la rotazione flessibile dei ruoli.
    \item Aggiornare il costo totale nel "Preventivo" in base alle nuove stime orarie.
    \item Introdurre un "Registro delle Modifiche" all'inizio dei tre documenti di presentazione (Lettera, Preventivo, Valutazione).
    \item Stabilire che le verifiche e approvazioni (entro 30/10) siano fatte da membri che non hanno redatto il documento specifico.
    \item Confermata la preferenza per il capitolato C8, anche dopo aver visionato le FAQ fornite dall'azienda VIMAR.
    \item Assegnare la redazione del presente verbale a Davide Biasuzzi.
\end{enumerate}

% -------------------------------------------------------------------
%  SEZIONE: todo.tex (Attività da svolgere)
% -------------------------------------------------------------------
\newpage
\section{Attività da Svolgere (To-Do)}

\begin{table}[h!]
\centering
\begin{tabular}{@{}lll@{}}
\toprule
\textbf{Attività} & \textbf{Assegnatario/i} & \textbf{Scadenza} \\
\midrule
Redazione Verbale Riunione 29/10 & Davide Biasuzzi & 29/10/25 \\
Revisione tempistiche ruoli e criteri rotazione & Leonardo Bilato & 29/10/25 \\
Pubblicazione finale sito GitHub Pages & G. Ponso, L. Bilato & 30/10/25 \\
Aggiornamento file Logo e ricompilazione docs & Giovanni Ponso & 30/10/25 \\
Redazione "Registro Modifiche" (Valutazione) & Davide Biasuzzi & 30/10/25 \\
Redazione "Registro Modifiche" (Preventivo) & Mihaela M. Romascu & 30/10/25 \\
Redazione "Registro Modifiche" (Lettera) & Francesco Zanella & 30/10/25 \\
Aggiornamento costi in "Preventivo Costi" & L. Bilato, M. M. Romascu & 30/10/25 \\
Verifica e Approvazione Documenti (Lettera, & Membri non redattori & 30/10/25 \\
Preventivo, Valutazione) & & \\
\bottomrule
\end{tabular}
\caption{Riepilogo task assegnati.}
\end{table}

\end{document}