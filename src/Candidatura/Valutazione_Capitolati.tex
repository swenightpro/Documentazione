\documentclass[a4paper, 11pt, oneside]{scrartcl} % Classe KOMA-Script

% --- Pacchetti Fondamentali ---
\usepackage[utf8]{inputenc}     % Codifica UTF-8
\usepackage[T1]{fontenc}        % Font encoding moderno
\usepackage[italian]{babel}     % Lingua italiana
\usepackage{lmodern}            % Font "Latin Modern"

% --- Grafica e Layout ---
\usepackage{graphicx}           % Per includere immagini
\usepackage{currfile}
\graphicspath{{src/immagini/}{\currfiledir contenuti/}{\currfiledir contenuti/immagini/}}

\usepackage[a4paper, top=2.5cm, bottom=3cm, left=2.5cm, right=2.5cm]{geometry} % Margini
\usepackage{fancyhdr}           % Per header e footer personalizzati
\usepackage{microtype}          % Migliora la tipografia
\usepackage[svgnames]{xcolor}   % Colori

% --- Utility ---
\usepackage{booktabs}           % Tabelle più professionali
\usepackage{enumitem}           % Per personalizzare liste
\usepackage{hyperref}           % Rende i link cliccabili
\hypersetup{
    colorlinks=true,
    linkcolor=DarkBlue,
    filecolor=DarkBlue,      
    urlcolor=DarkBlue,
    citecolor=DarkBlue,
    pdftitle={Documento Progetto - NightPRO},
    pdfauthor={Gruppo NightPRO},
}

% ===================================================================
%  IMPOSTAZIONE HEADER E FOOTER
% ===================================================================
\pagestyle{fancy}
\fancyhf{} % Pulisce tutti i campi
\fancyhead[L]{NightPRO - Progetto Ingegneria del Software}
\fancyhead[R]{Anno Accademico 2025/2026}
\fancyfoot[C]{\thepage} % Numero di pagina al centro in basso
\renewcommand{\headrulewidth}{0.4pt} % Linea sottile sotto l'header
\renewcommand{\footrulewidth}{0pt}

% ===================================================================
%  INIZIO DEL DOCUMENTO
% ===================================================================
\begin{document}

% -------------------------------------------------------------------
%  SEZIONE: intestazione_titolo.tex
% -------------------------------------------------------------------
\thispagestyle{empty}
\begin{titlepage}
    \centering
    
\begin{figure}
    \centering
    \includegraphics[width=0.4\textwidth]{logo.jpg}
\end{figure}

    \vfill
    
    {\small UNIVERSITÀ DEGLI STUDI DI PADOVA \par}
    {\small CORSO DI LAUREA IN INFORMATICA (L-31) \par}
    \vspace{0.5cm}
    {\large Corso di Ingegneria del Software \par}
    {\small Anno Accademico 2025/2026 \par}


    
    \vfill
    
    {\Huge \bfseries Preventivo Costi e Assunzione Impegni \par}
    
    \vspace{1cm}
    
    % Inserisci qui il titolo specifico del documento
    {\Large \itshape Redattori: Romascu Mihaela Mariana; Ponso Giovanni; Biasuzzi Davide  \par} 
 
    
    \vfill
    
    {\Large \bfseries Gruppo: NightPRO \par}
    \vspace{0.5cm}
    {\large \href{mailto:swe.nightpro@gmail.com}{swe.nightpro@gmail.com} \par}
    
    \vfill
    
    % Inserisci qui la data di redazione del documento
    {\large Data: 28 ottobre 2025 \par}
    {\Large \itshape Versione: 0.1 \par} 

\end{titlepage}

% -------------------------------------------------------------------
%  SEZIONE: indice.tex
% -------------------------------------------------------------------
\newpage
\tableofcontents % Genera l'indice
\pagestyle{fancy} % Riattiva lo stile di pagina da qui in poi

% -------------------------------------------------------------------
%  SEZIONE: informazioni.tex
% -------------------------------------------------------------------
\newpage
\section{C1. Automated EN18031 Compliance Verification}
\subsection*{Valutazione Critica del Capitolato}
Il progetto di \textbf{Bluewind} mira a realizzare un’applicazione software (desktop o web) per automatizzare la verifica di conformità alla norma EN18031, relativa alla sicurezza informatica dei dispositivi radio secondo la Direttiva RED.\\
L’idea risponde a un’esigenza reale: ridurre la complessità e gli errori del processo manuale di verifica normativa.

La soluzione digitalizza i \emph{Decision Tree} della norma, consentendo all’utente di navigarli, ottenere risultati automatici (“Pass”, “Fail”, “Not applicable”) e visualizzarli in una dashboard interattiva.\\
Il caso di studio scelto — una macchina del caffè connessa via Wi-Fi — è realistico e utile per testare autenticazione, scambio sicuro di dati e logiche di verifica.

Il capitolato offre flessibilità tecnologica (Python consigliato, ma non vincolante) e un approccio Agile, favorendo autonomia e adattabilità. Tuttavia, la complessità normativa e la scarsa specificità tecnica del documento possono rendere impegnativa la realizzazione per studenti senza conoscenze in ambito cybersecurity o standard europei.

\subsubsection*{Pro}
\begin{itemize}
    \item Tema reale e utile nel contesto della sicurezza IoT.
    \item Automazione di un processo complesso e ripetitivo.
    \item Libertà di scelta tecnologica.
    \item Supporto aziendale costante.
    \item Caso di test concreto e chiaro.
    \item Approccio Agile e flessibile.
\end{itemize}

\subsubsection*{Contro}
\begin{itemize}
    \item Comprensione complessa della norma EN18031.
    \item Mancanza di dettagli su architettura e parsing dei Decision Tree.
    \item Carico cognitivo elevato nella modellazione dei flussi decisionali.
    \item Scarsa chiarezza sui criteri di valutazione automatica.
    \item UI/UX poco approfondita nel documento.
\end{itemize}

% ======================================================================
\section{C2. Code Guardian}
\subsection*{Valutazione Critica del Capitolato}
Il progetto propone una piattaforma intelligente per analizzare repository GitHub, valutando qualità del codice, sicurezza e manutenzione, e fornendo report e suggerimenti di miglioramento.\\
L’architettura prevede un orchestratore centrale che coordina agenti specializzati (per test, sicurezza, documentazione, report), con una dashboard web in React.

\subsubsection*{Pro}
\begin{itemize}
    \item Tema innovativo e rilevante per il mondo DevOps.
    \item Architettura moderna e modulare con AI e multi-agenti.
    \item Stack tecnologico attuale (Node.js, Python, React, AWS).
    \item Supporto aziendale professionale e continuo.
    \item Forte componente formativa in sicurezza e qualità del software.
\end{itemize}

\subsubsection*{Contro}
\begin{itemize}
    \item Elevata complessità tecnica e di integrazione.
    \item Ambito vasto, rischio di dispersione.
    \item Definizione non chiara delle metriche di valutazione.
    \item Tempi lunghi per ottenere una demo completa.
\end{itemize}

% ======================================================================
\section{C3. DIPReader}
\subsection*{Valutazione Critica del Capitolato}
Il progetto punta a realizzare un software multipiattaforma per la consultazione e ricerca di archivi digitali provenienti da sistemi di conservazione documentale.\\
L’uso di SQLite e FAISS rende il sistema leggero ma moderno, con possibilità di integrare ricerca semantica.

\subsubsection*{Pro}
\begin{itemize}
    \item Tema concreto e utile per aziende e PA.
    \item MVP chiaro e realistico.
    \item Tecnologie leggere e diffuse.
    \item Funzionamento offline e multipiattaforma.
    \item Supporto aziendale solido e materiale reale.
\end{itemize}

\subsubsection*{Contro}
\begin{itemize}
    \item Necessità di ottimizzare prestazioni su grandi volumi.
    \item Ricerca semantica complessa da implementare.
    \item Comprensione minima del contesto normativo richiesta.
    \item UI/UX da progettare con cura per evitare eccessiva tecnicità.
\end{itemize}

% ======================================================================
\section{C4. L’app che Protegge e Trasforma}
\subsection*{Valutazione Critica del Capitolato}
Il progetto propone un’app mobile per prevenire e gestire situazioni di violenza di genere, con funzionalità di analisi AI, alert silenziosi e geolocalizzazione sicura.

\subsubsection*{Pro}
\begin{itemize}
    \item Forte valore sociale e impatto reale.
    \item Tecnologie moderne (AI, AWS, Flutter).
    \item Supporto tecnico e formativo continuo.
    \item Focus su sicurezza, privacy e inclusività.
\end{itemize}

\subsubsection*{Contro}
\begin{itemize}
    \item Alta complessità tecnica e gestionale.
    \item Rischio di dispersione se non si limita l’MVP.
    \item Implementazione delicata della privacy.
    \item Richiede sensibilità etica oltre che tecnica.
\end{itemize}

% ======================================================================
\section{C5. NEXUM}
\subsection*{Valutazione Critica del Capitolato}
Il progetto NEXUM punta a digitalizzare la gestione HR nelle PMI, migliorando la comunicazione tra aziende, consulenti e dipendenti, con moduli basati su AI generativa e OCR.

\subsubsection*{Pro}
\begin{itemize}
    \item Progetto concreto e destinato al mercato reale.
    \item Innovazione con AI generativa e OCR.
    \item Stack moderno e formazione Agile.
    \item Supporto costante e ambiente professionale.
\end{itemize}

\subsubsection*{Contro}
\begin{itemize}
    \item Alta complessità tecnica e integrazione AI.
    \item Requisiti di precisione elevati.
    \item Dipendenza da servizi esterni (OCR, LLM).
\end{itemize}

% ======================================================================
\section{C6. Second Brain}
\subsection*{Valutazione Critica del Capitolato}
Il progetto Zucchetti propone un editor di testo intelligente che utilizza LLM per assistere nella scrittura, revisione e generazione di testi, basato su Markdown.

\subsubsection*{Pro}
\begin{itemize}
    \item Tema innovativo e attuale.
    \item Progetto scalabile e sperimentale.
    \item Requisiti chiari e raggiungibili.
    \item Supporto tecnico disponibile.
\end{itemize}

\subsubsection*{Contro}
\begin{itemize}
    \item Limitato impatto industriale immediato.
    \item Prompt engineering complesso.
    \item Possibili problemi di privacy con API esterne.
\end{itemize}

% ======================================================================
\section{C7. Sistema di acquisizione dati da sensori}
\subsection*{Valutazione Critica}
Il progetto prevede la creazione di un’infrastruttura cloud per la gestione di dati da sensori BLE, simulando un ecosistema IoT.

\subsubsection*{Pro}
\begin{itemize}
    \item Architettura coerente e professionale.
    \item Focus su sicurezza e scalabilità.
    \item Tecnologie moderne e richieste dal mercato.
    \item Ottima esperienza formativa.
\end{itemize}

\subsubsection*{Contro}
\begin{itemize}
    \item Elevata complessità tecnica e di setup.
    \item Limitata componente frontend.
    \item Rischio di tempi lunghi di sviluppo.
\end{itemize}

% ======================================================================
\section{C8. Smart Order}
\subsection*{Valutazione Critica Generale}
Il progetto propone una piattaforma multimodale intelligente capace di interpretare ordini d’acquisto provenienti da testo, audio e immagini, trasformandoli in ordini strutturati per ERP.

\subsubsection*{Pro}
\begin{itemize}
    \item Innovativo e concreto.
    \item Architettura modulare e scalabile.
    \item Libertà tecnologica.
    \item Supporto aziendale solido.
\end{itemize}

\subsubsection*{Contro}
\begin{itemize}
    \item Alta complessità tecnica.
    \item Necessità di dati di qualità per training.
    \item Integrazione ERP complessa.
\end{itemize}

% ======================================================================
\section{C9. View4Life}
\subsection*{Valutazione Critica}
Il progetto Vimar prevede una piattaforma cloud e web per la gestione di impianti domotici nelle residenze per anziani, basata su KNX IoT API.

\subsubsection*{Pro}
\begin{itemize}
    \item Progetto reale con finalità sociali.
    \item Tecnologie moderne (IoT, Cloud, API).
    \item Esperienza con dispositivi fisici.
    \item Supporto aziendale concreto.
\end{itemize}

\subsubsection*{Contro}
\begin{itemize}
    \item Alta complessità tecnica e di integrazione.
    \item Rischio logistico legato ai dispositivi fisici.
    \item Necessario coordinamento costante del team.
\end{itemize}


\section{Conclusione}

L’analisi complessiva dei capitolati mostra un’elevata qualità e varietà delle proposte, che spaziano da progetti tecnici e industriali a soluzioni con forte impatto sociale.\\[0.5em]
In generale, i capitolati più formativi e bilanciati risultano quelli che combinano \textbf{chiarezza degli obiettivi}, \textbf{supporto aziendale solido} e \textbf{complessità gestibile}, come \textit{DIPReader} e \textit{Second Brain}.\\[0.5em]
Progetti come \textit{Code Guardian} o \textit{Smart Order} offrono invece sfide più avanzate, ideali per team esperti e ben organizzati.\\[0.5em]
In sintesi, ogni proposta rappresenta un’opportunità concreta per sviluppare \textbf{competenze tecniche}, \textbf{collaborative} e \textbf{progettuali} di alto livello.

% ======================================================================
\end{document}