\documentclass[a4paper, 11pt, oneside]{scrartcl} % Classe KOMA-Script

% --- Pacchetti Fondamentali ---
\usepackage[utf8]{inputenc}     % Codifica UTF-8
\usepackage[T1]{fontenc}        % Font encoding moderno
\usepackage[italian]{babel}     % Lingua italiana
\usepackage{lmodern}            % Font "Latin Modern"

% --- Grafica e Layout ---
\usepackage{graphicx}           % Per le immagini
\graphicspath{{../../assets/}}
\usepackage[a4paper, top=2.5cm, bottom=3cm, left=2.5cm, right=2.5cm]{geometry} % Margini
\usepackage{fancyhdr}           % Per header e footer personalizzati
\usepackage{microtype}          % Migliora la tipografia
\usepackage[svgnames]{xcolor}   % Colori

% --- Utility ---
\usepackage{booktabs}           % Tabelle più professionali
\usepackage{enumitem}           % Per personalizzare liste
\usepackage{hyperref}           % Rende i link cliccabili
\hypersetup{
    colorlinks=true,
    linkcolor=DarkBlue,
    filecolor=DarkBlue,      
    urlcolor=DarkBlue,
    citecolor=DarkBlue,
    pdftitle={Documento Progetto - NightPRO},
    pdfauthor={Gruppo NightPRO},
}


% ===================================================================
%  HEADER E FOOTER
% ===================================================================
\pagestyle{fancy}
\fancyhf{} % Pulisce i campi
\fancyhead[L]{\textbf{NightPRO – Progetto Ingegneria del Software}}
\fancyhead[R]{Anno Accademico 2025/2026}
\fancyfoot[C]{\thepage} % Numero di pagina al centro
\renewcommand{\headrulewidth}{0.4pt}
\renewcommand{\footrulewidth}{0.4pt} % Modificato per coerenza, sebbene l'originale fosse 0pt

% ===================================================================
%  INIZIO DOCUMENTO
% ===================================================================
\begin{document}

% -------------------------------------------------------------------
%  FRONTESPIZIO
% -------------------------------------------------------------------
\thispagestyle{empty}
\begin{titlepage}
    \centering
    \vspace*{1cm}
    \includegraphics[width=0.35\textwidth]{logo.png}\\[1cm]

     \vfill
    
    {\small UNIVERSITÀ DEGLI STUDI DI PADOVA \par}
    {\small CORSO DI LAUREA IN INFORMATICA (L-31) \par}
    \vspace{0.5cm}
    {\large Corso di Ingegneria del Software \par}
    {\small Anno Accademico 2025/2026 \par}
    \vfill
    
    {\Huge \bfseries Glossario \par}
        \vspace{1cm}
         {\Large Redattori: Giovanni Ponso, Davide Biasuzzi \par} 
    {\Large Approvato da: \par}
    \vfill

    {\Large \bfseries Gruppo: NightPRO}    \vspace{0.5cm}

    {\large \href{mailto:swe.nightpro@gmail.com}{swe.nightpro@gmail.com}}\\[2cm]

        {\large Data: 2025-11-05 \par}

     {\Large Versione: 1.0 \par} 

\end{titlepage}

%  SEZIONE: Tabella delle Versioni
% -------------------------------------------------------------------
\newpage
\pagestyle{fancy}
\phantomsection
\addcontentsline{toc}{section}{Tabella delle Versioni}
\section*{Tabella delle Versioni}
\vspace{0.2cm} 
\begin{center}
\resizebox{\textwidth}{!}{
\renewcommand{\arraystretch}{1.2}
\begin{tabular}{@{}llp{0.25\textwidth}p{0.45\textwidth}c@{}} 
\toprule
\textbf{Versione} & \textbf{Data} & \textbf{Autore/i} & \textbf{Descrizione delle Modifiche} & \textbf{Verificatore} \\
\midrule
1.0 & 2025-11-05 & Davide Biasuzzi & Inserimento definizioni iniziali mancanti & Francesco Zanella \\
0.1 & 2025-11-05 & Giovanni Ponso & Creazione bozza glossario e inserimento definizioni &  Francesco Zanella \\
\bottomrule
\end{tabular}
}
\end{center}


\newpage
\tableofcontents % Genera l'indice
\pagestyle{fancy}

% -------------------------------------------------------------------
%  INFORMAZIONI GENERALI
% -------------------------------------------------------------------
\newpage
\section{Informazioni Generali}

\subsection{Componenti del Gruppo}

\begin{table}[h!]
\centering
\renewcommand{\arraystretch}{1.2} % più spazio tra le righe
\begin{tabular}{@{}llc@{}}
\toprule
\textbf{Cognome} & \textbf{Nome} & \textbf{Matricola} \\
\midrule
Biasuzzi & Davide & 2111000 \\
Bilato & Leonardo & 2071084 \\
Zanella & Francesco & 2116442 \\
Romascu & Mihaela-Mariana & 2079726 \\
Ogniben & Michele & 2042325 \\
Perozzo & Samuele & 2110989 \\
Ponso & Giovanni & 2000558 \\
\bottomrule
\end{tabular}
\caption{Componenti del gruppo NightPRO.}
\end{table}

% -------------------------------------------------------------------
% INTRODUZIONE
% -------------------------------------------------------------------


\section{Introduzione}

\subsection{Obiettivo del documento}
Il presente glossario raccoglie i termini tecnici, specialistici o potenzialmente ambigui utilizzati nella documentazione del progetto.
Il suo scopo è garantire una comprensione uniforme del linguaggio adottato dal gruppo, fornendo definizioni chiare e non equivoche.
\subsection{Struttura del documento}
I termini sono organizzati in ordine alfabetico.  
Alla loro prima apparizione nei documenti ufficiali, essi vengono contrassegnati dal pedice \textsubscript{\textsc{g}}, che ne indica la presenza nel glossario.
Le occorrenze successive non riportano tale marcatura.

\appendix 

% A
\section*{A}
\addcontentsline{toc}{section}{A}

\subsection{Approvazione}
Fase formale del ciclo di vita di un documento, successiva alla verifica, in cui si certifica che il documento è completo, corretto e pronto per il rilascio ufficiale.
\vspace{0.5cm}

% B
\section*{B}
\addcontentsline{toc}{section}{B}

\subsection{Build}
Il processo automatico di compilazione del codice sorgente (es. file \texttt{.tex}) per generare un artefatto distribuibile (es. un file \texttt{.pdf}).
\vspace{0.5cm}

% C
\section*{C}
\addcontentsline{toc}{section}{C}

\subsection{Capitolato}
Documento fornito dal committente che specifica i requisiti, gli obiettivi e i vincoli del progetto da realizzare.
\vspace{0.5cm}

\subsection{CI (Continuous Integration)}
Acronimo di \textit{Continuous Integration} (Integrazione Continua). Pratica di sviluppo che consiste nell'automatizzare l'integrazione e la verifica (build e test) del codice o dei documenti ad ogni modifica, al fine di rilevare tempestivamente errori.
\vspace{0.5cm}

\subsection{Commit}
Una singola operazione di salvataggio registrata nel sistema di versionamento (come Git). Rappresenta un insieme di modifiche apportate ai file del repository.
\vspace{0.5cm}

% D
\section*{D}
\addcontentsline{toc}{section}{D}

\subsection{Diario della Riunione}
Sezione di un verbale che contiene il resoconto dettagliato della discussione, tipicamente suddiviso rispecchiando i punti dell'Ordine del Giorno.
\vspace{0.5cm}

\subsection{Docker}
Piattaforma software che permette di creare, distribuire ed eseguire applicazioni in ambienti isolati chiamati "container". Utilizzata nel progetto per garantire un ambiente di compilazione \LaTeX{} coerente e riproducibile.
\vspace{0.5cm}

% E
\section*{E}
\addcontentsline{toc}{section}{E}
% (Nessun termine)

% F
\section*{F}
\addcontentsline{toc}{section}{F}

\subsection{Frontespizio}
La prima pagina di un documento ufficiale che contiene i metadati identificativi: titolo, autori, gruppo, versione, data, e contesto (es. università, corso).
\vspace{0.5cm}

% G
\section*{G}
\addcontentsline{toc}{section}{G}

\subsection{Git}
Sistema di controllo versione distribuito (DVCS) creato da Linus Torvalds. 
È lo strumento software che traccia la cronologia delle modifiche ai file di progetto (sorgenti, documenti) e permette la collaborazione.
\vspace{0.5cm}

\subsection{GitHub}
Piattaforma web basata su Git per l'hosting di repository, la gestione del versionamento del codice e la collaborazione allo sviluppo software.
\vspace{0.5cm}

\subsection{GitHub Actions}
Servizio di automazione e CI/CD integrato in GitHub. Permette di definire \textit{workflow} automatici (es. per la compilazione e pubblicazione dei documenti) che si attivano in risposta a eventi nel repository.
\vspace{0.5cm}

\subsection{GitHub Pages}
Servizio di hosting per siti web statici fornito da GitHub. Utilizzato dal gruppo per pubblicare e rendere consultabile la documentazione di progetto.
\vspace{0.5cm}

\subsection{GitHub Projects}
Strumento di gestione progettuale (project management) integrato in GitHub, utilizzato per pianificare, organizzare e tracciare lo stato di avanzamento delle attività (task).
\vspace{0.5cm}

\subsection{Glossario}
Documento che raccoglie e definisce i termini tecnici, specialistici o potenzialmente ambigui utilizzati nel progetto, al fine di garantirne una comprensione uniforme.
\vspace{0.5cm}

\subsection{Google Meet}
Piattaforma di videoconferenza utilizzata dal gruppo per le riunioni sincrone e le attività collaborative che richiedono condivisione dello schermo.
\vspace{0.5cm}

% H
\section*{H}
\addcontentsline{toc}{section}{H}
% (Nessun termine)

% I
\section*{I}
\addcontentsline{toc}{section}{I}
% (Nessun termine)

% J
\section*{J}
\addcontentsline{toc}{section}{J}
% (Nessun termine)

% K
\section*{K}
\addcontentsline{toc}{section}{K}
% (Nessun termine)

% L
\section*{L}
\addcontentsline{toc}{section}{L}

\subsection{LaTeX}
Linguaggio di markup e sistema di preparazione di documenti ampiamente usato in ambito accademico per la sua elevata qualità tipografica e la gestione di formule complesse. È il formato sorgente per la documentazione ufficiale del progetto.
\vspace{0.5cm}

% M
\section*{M}
\addcontentsline{toc}{section}{M}
% (Nessun termine)

% N
\section*{N}
\addcontentsline{toc}{section}{N}

\subsection{Norme di Progetto}
Il documento (questo) che definisce il metodo di lavoro, le regole redazionali, gli strumenti e i processi di gestione adottati dal gruppo per garantire coerenza, tracciabilità e qualità.
\vspace{0.5cm}

% O
\section*{O}
\addcontentsline{toc}{section}{O}

\subsection{Ordine del Giorno}
Elenco degli argomenti (chiamato anche "Agenda") pianificati per la discussione durante una riunione. È una sezione obbligatoria dei verbali.
\vspace{0.5cm}

% P
\section*{P}
\addcontentsline{toc}{section}{P}

\subsection{PDF}
Acronimo di \textit{Portable Document Format}. Formato file utilizzato per la distribuzione e la consultazione dei documenti ufficiali, in quanto garantisce la non modificabilità e la coerenza di layout su diversi dispositivi.
\vspace{0.5cm}

\subsection{Pipeline}
Nel contesto della CI/CD, è la sequenza di passi (job) automatizzati eseguiti da un sistema di automazione (come GitHub Actions) per compiere un processo, ad esempio la \textit{pipeline di pubblicazione}. Sinonimo di Workflow.
\vspace{0.5cm}

\subsection{Pubblicazione}
Fase finale del ciclo di vita di un documento che consiste nel renderlo accessibile agli stakeholder, ad esempio caricandolo nell'archivio PDF sul sito web del gruppo.
\vspace{0.5cm}

% Q
\section*{Q}
\addcontentsline{toc}{section}{Q}
% (Nessun termine)

% R
\section*{R}
\addcontentsline{toc}{section}{R}

\subsection{Redazione}
Fase iniziale del ciclo di vita di un documento, durante la quale uno o più autori (redattori) creano e aggiornano i contenuti.
\vspace{0.5cm}

\subsection{Repository}
Archivio centrale gestito da un sistema di versionamento (come Git) che contiene tutti i file del progetto (codice sorgente, documenti) e la cronologia completa delle loro modifiche.
\vspace{0.5cm}

% S
\section*{S}
\addcontentsline{toc}{section}{S}

\subsection{Snake case}
Convenzione di nomenclatura dei file (es. \texttt{nome\_file.tex}) che prevede l'uso di sole lettere minuscole e la separazione delle parole tramite un trattino basso (underscore).
\vspace{0.5cm}

\subsection{Stakeholder}
Qualsiasi individuo o gruppo che ha un interesse nel progetto o è influenzato dal suo risultato (es. membri del team, committenti, docenti).
\vspace{0.5cm}

% T
\section*{T}
\addcontentsline{toc}{section}{T}

\subsection{Task}
Un'attività o un compito specifico e tracciabile che deve essere svolto. La gestione dei task avviene tramite GitHub Projects.
\vspace{0.5cm}

\subsection{Telegram}
Applicazione di messaggistica istantanea usata come canale di comunicazione principale dal gruppo.
\vspace{0.5cm}

\subsection{Template}
Un modello di documento (es. un file \texttt{.tex} preimpostato) utilizzato come base per la creazione di nuovi documenti, al fine di garantire l'omogeneità della struttura e del formato.
\vspace{0.5cm}

\subsection{Topic (Telegram)}
Funzionalità dei gruppi Telegram che permette di suddividere le conversazioni in argomenti specifici, garantendo maggiore ordine e tracciabilità delle discussioni.
\vspace{0.5cm}

% U
\section*{U}
\addcontentsline{toc}{section}{U}
% (Nessun termine)

% V
\section*{V}
\addcontentsline{toc}{section}{V}

\subsection{Verbali}
Documenti ufficiali che riportano le discussioni, le decisioni prese e le attività assegnate durante gli incontri del gruppo, al fine di tracciare l'evoluzione del progetto.
\vspace{0.5cm}

\subsection{Verifica}
Fase del ciclo di vita di un documento in cui un membro del gruppo, diverso dal redattore, controlla la correttezza formale, la qualità e la coerenza dei contenuti prima dell'approvazione o pubblicazione.
\vspace{0.5cm}

% W
\section*{W}
\addcontentsline{toc}{section}{W}

\subsection{Workflow}
Una sequenza configurabile di operazioni automatizzate. Nel contesto di GitHub Actions, definisce i passi per processi come la compilazione o la pubblicazione. Sinonimo di Pipeline.
\vspace{0.5cm}

% X
\section*{X}
\addcontentsline{toc}{section}{X}
% (Nessun termine)

% Y
\section*{Y}
\addcontentsline{toc}{section}{Y}
% (Nessun termine)

% Z
\section*{Z}
\addcontentsline{toc}{section}{Z}
% (Nessun termine)


\end{document}