\documentclass[a4paper, 11pt, oneside]{scrartcl} % Classe KOMA-Script

% --- Pacchetti Fondamentali ---
\usepackage[utf8]{inputenc}     % Codifica UTF-8
\usepackage[T1]{fontenc}        % Font encoding moderno
\usepackage[italian]{babel}     % Lingua italiana
\usepackage{lmodern}            % Font "Latin Modern"

% --- Grafica e Layout ---
\usepackage{graphicx}           % Per le immagini
\graphicspath{{../../assets/}}
\usepackage[a4paper, top=2.5cm, bottom=3cm, left=2.5cm, right=2.5cm]{geometry} % Margini
\usepackage{fancyhdr}           % Per header e footer personalizzati
\usepackage{microtype}          % Migliora la tipografia
\usepackage[svgnames]{xcolor}   % Colori

% --- Utility ---
\usepackage{booktabs}           % Tabelle più professionali
\usepackage{enumitem}           % Per personalizzare liste
\usepackage{hyperref}           % Rende i link cliccabili
\hypersetup{
    colorlinks=true,
    linkcolor=DarkBlue,
    filecolor=DarkBlue,      
    urlcolor=DarkBlue,
    citecolor=DarkBlue,
    pdftitle={Documento Progetto - NightPRO},
    pdfauthor={Gruppo NightPRO},
}


% ===================================================================
%  HEADER E FOOTER
% ===================================================================
\pagestyle{fancy}
\fancyhf{} % Pulisce i campi
\fancyhead[L]{\textbf{NightPRO – Progetto Ingegneria del Software}}
\fancyhead[R]{Anno Accademico 2025/2026}
\fancyfoot[C]{\thepage} % Numero di pagina al centro
\renewcommand{\headrulewidth}{0.4pt}
\renewcommand{\footrulewidth}{0pt}

% ===================================================================
%  INIZIO DOCUMENTO
% ===================================================================
\begin{document}

% -------------------------------------------------------------------
%  FRONTESPIZIO
% -------------------------------------------------------------------
\thispagestyle{empty}
\begin{titlepage}
    \centering
    \vspace*{1cm}
    \includegraphics[width=0.35\textwidth]{logo.png}\\[1cm]

     \vfill
    
    {\small UNIVERSITÀ DEGLI STUDI DI PADOVA \par}
    {\small CORSO DI LAUREA IN INFORMATICA (L-31) \par}
    \vspace{0.5cm}
    {\large Corso di Ingegneria del Software \par}
    {\small Anno Accademico 2025/2026 \par}
    \vfill
    
    {\Huge \bfseries Norme di Progetto \par}
        \vspace{1cm}
         {\Large Redattori: Giovanni Ponso, Davide Biasuzzi, Michele Ogniben \par} 
    {\Large Approvato da: Mihaela Mariana Romascu \par}
    \vfill

    {\Large \bfseries Gruppo: NightPRO}    \vspace{0.5cm}

    {\large \href{mailto:swe.nightpro@gmail.com}{swe.nightpro@gmail.com}}\\[2cm]

        {\large Data: 2025-11-11 \par}

     {\Large Versione: 1.1 \par} 

\end{titlepage}

%  SEZIONE: Tabella delle Versioni
% -------------------------------------------------------------------
\newpage
\pagestyle{fancy}
\phantomsection
\addcontentsline{toc}{section}{Tabella delle Versioni}
\section*{Tabella delle Versioni}
\vspace{0.2cm} 
\begin{center}
\resizebox{\textwidth}{!}{
\renewcommand{\arraystretch}{1.2}
\begin{tabular}{@{}llp{0.25\textwidth}p{0.45\textwidth}c@{}} 
\toprule
\textbf{Versione} & \textbf{Data} & \textbf{Autore/i} & \textbf{Descrizione delle Modifiche} & \textbf{Verificatore} \\
\midrule
1.1 & 2025-11-11 & M. Ogniben & Aggiunto paragrafo relativo ai Processi Organizzativi & Davide Biasuzzi\\
1.0 & 2025-11-05 & D. Biasuzzi & Modifica struttura del file e verifica & Francesco Zanella\\
0.1 & 2025-11-05 & G. Ponso; D. Biasuzzi & Creazione documento + redazione processi di supporto &  Francesco Zanella\\
\bottomrule
\end{tabular}
}
\end{center}


\newpage
\tableofcontents % Genera l'indice
\pagestyle{fancy}

% -------------------------------------------------------------------
%  INFORMAZIONI GENERALI
% -------------------------------------------------------------------
\newpage
\section{Informazioni Generali}

\subsection{Componenti del Gruppo}

\begin{table}[h!]
\centering
\renewcommand{\arraystretch}{1.2} % più spazio tra le righe
\begin{tabular}{@{}llc@{}}
\toprule
\textbf{Cognome} & \textbf{Nome} & \textbf{Matricola} \\
\midrule
Biasuzzi & Davide & 2111000 \\
Bilato & Leonardo & 2071084 \\
Zanella & Francesco & 2116442 \\
Romascu & Mihaela-Mariana & 2079726 \\
Ogniben & Michele & 2042325 \\
Perozzo & Samuele & 2110989 \\
Ponso & Giovanni & 2000558 \\
\bottomrule
\end{tabular}
\caption{Componenti del gruppo NightPRO.}
\end{table}

% -------------------------------------------------------------------
% INTRODUZIONE
% -------------------------------------------------------------------

\newpage
\section{Introduzione}
\label{sec:introduzione}

\subsection{Scopo del documento}
Questo documento definisce le Norme di Progetto{\scriptsize\raisebox{-0.5ex}{G}} del gruppo NightPRO per il corso di Ingegneria del Software (A.A.~2025/2026).
Stabilisce in forma univoca: metodo di lavoro, regole redazionali e criteri di gestione dei documenti, così da garantire coerenza, tracciabilità e chiarezza operativa.
La presente versione costituisce la base metodologica iniziale e sarà aggiornata con l’avanzare delle attività.
\subsection{Stato del progetto}
Al momento il capitolato{\scriptsize\raisebox{-0.5ex}{G}} non è assegnato. Il documento disciplina quindi l’assetto organizzativo e documentale preliminare: modalità di collaborazione, convenzioni redazionali e pubblicazione dei deliverable.
Le sezioni relative a prodotto, pianificazione dettagliata e processi di qualità, verifica e validazione saranno integrate nelle versioni successive, in coerenza con il capitolato che verrà attribuito.
\subsection{Glossario}

Per garantire chiarezza terminologica e uniformità nella comprensione dei concetti utilizzati, viene redatto un Glossario{\scriptsize\raisebox{-0.5ex}{G}} contenente i termini tecnici e potenzialmente ambigui.
Alla \textbf{prima occorrenza} nel documento, ogni termine seguito dal simbolo {\scriptsize\raisebox{-0.5ex}{G}} viene definito formalmente nel glossario.
Le occorrenze successive non riportano tale marcatura.

% -------------------------------------------------------------------
% PROCESSI DI SUPPORTO
% -------------------------------------------------------------------
\newpage
\section{Processi di Supporto}

% -------------------------------------------------------------------
%      STRUTTURA AMBIENTE DI SVILUPPO PROGETTO (FASE PRELIMINARE)
% -------------------------------------------------------------------
 
\subsection{Ambiente collaborativo e infrastruttura del progetto}

\subsubsection{Obiettivo}
L'obiettivo di questa sezione è definire l'insieme di strumenti e configurazioni adottate dal gruppo per supportare la collaborazione, la comunicazione, l’organizzazione e l’integrazione tecnica del progetto.
Pur trovandosi in fase preliminare, il team ha ritenuto fondamentale predisporre fin da subito un ambiente robusto e strutturato che garantisca tracciabilità, trasparenza, automazione e continuità operativa per l’intera durata del progetto.
\subsubsection{Comunicazione sincrona e asincrona}

\paragraph{Telegram}
Il gruppo utilizza Telegram{\scriptsize\raisebox{-0.5ex}{G}} come canale di comunicazione principale.
La scelta è motivata dalla disponibilità della funzionalità topic{\scriptsize\raisebox{-0.5ex}{G}}, che permette di suddividere le conversazioni per argomento (es. documentazione, riunioni, diario di bordo), garantendo ordine e tracciabilità.
\paragraph{Google Meet}
Per le riunioni sincrone viene adottato Google Meet{\scriptsize\raisebox{-0.5ex}{G}}, in quanto:
\begin{itemize}
    \item consente la partecipazione rapida e trasversale ai membri del gruppo;
\item offre funzionalità di condivisione schermo utili durante attività collaborative;
    \item supporta riunioni sia programmate che estemporanee.
\end{itemize}

\subsubsection{Gestione del repository e versionamento}
Il gruppo ha istituito un’organizzazione \textbf{GitHub}{\scriptsize\raisebox{-0.5ex}{G}} denominata \textbf{NightPRO}, all’interno della quale sono presenti due repository{\scriptsize\raisebox{-0.5ex}{G}} distinti:

\begin{itemize}
    \item \textbf{Documentazione}: dedicato alla gestione della documentazione di progetto, configurazione del sito pubblico e automazione CI{\scriptsize\raisebox{-0.5ex}{G}};
\item \textbf{Prodotto software}: inizialmente vuoto, verrà popolato quando il capitolato verrà assegnato.
\end{itemize}

La separazione è stata decisa per garantire indipendenza tra le attività documentali e quelle di sviluppo software, favorendo chiarezza, modularità e controllo delle versioni.
\paragraph{Struttura del repository documentazione}
La struttura adottata è la seguente:

\begin{verbatim}
.github/workflows/   # Workflow CI (build PDF, deploy sito)
src/                 # File .tex sorgenti dei documenti
docs/                # PDF generati automaticamente
site/                # Codice sorgente sito GitHub Pages
template/            # Template e documenti base
report.md            # Report compilazione automatica
\end{verbatim}

\subsubsection{Gestione delle attività}
Per la pianificazione, il monitoraggio delle attività e la tracciabilità dello stato di avanzamento, il gruppo utilizza \textbf{GitHub Projects}{\scriptsize\raisebox{-0.5ex}{G}}.
Questa piattaforma permette di:

\begin{itemize}
    \item assegnare responsabilità chiare ai membri del gruppo;
    \item monitorare l'avanzamento delle attività;
\item mantenere una visione condivisa delle priorità;
    \item aggiornare in tempo reale lo stato dei task{\scriptsize\raisebox{-0.5ex}{G}}.
\end{itemize}

In questa fase iniziale il flusso operativo è in definizione e verrà progressivamente raffinato in base alle esigenze progettuali e all’evoluzione delle attività.
L'obiettivo è arrivare a un sistema di lavoro che garantisca trasparenza, coordinamento ed efficienza, riducendo il rischio di sovrapposizioni o attività non monitorate.
\subsubsection{Sistema di pubblicazione e consultazione della documentazione}
Per garantire un accesso semplice, centralizzato e sempre aggiornato alla documentazione di progetto, è stato sviluppato un sito web statico pubblicato tramite \textbf{GitHub Pages}{\scriptsize\raisebox{-0.5ex}{G}}.
\paragraph{Obiettivo}
Rendere la documentazione facilmente consultabile da tutti gli stakeholder{\scriptsize\raisebox{-0.5ex}{G}} (membri del team, committenti e corpo docente), mantenendo coerenza, ordine e tracciabilità delle versioni.
\paragraph{Funzionalità principali}
Il sito offre:

\begin{itemize}
    \item navigazione dei documenti tramite struttura a cartelle collassabili;
\item motore di ricerca interno per nome, data o versione del documento;
\item possibilità di apertura o download diretto dei PDF;
    \item pagina informativa con i membri del gruppo e relativi riferimenti;
\item selezione del tema grafico chiaro, scuro o sistema.
\end{itemize}

\paragraph{Aggiornamento automatico}
La struttura del sito e l’elenco dei documenti vengono aggiornati automaticamente ad ogni esecuzione della pipeline{\scriptsize\raisebox{-0.5ex}{G}} di pubblicazione.
Durante la build{\scriptsize\raisebox{-0.5ex}{G}}, un processo analizza la cartella \texttt{docs/} e genera la mappa della documentazione pubblicata, garantendo che il sito rifletta sempre lo stato più recente e valido dei file.
Grazie a questa configurazione, la consultazione della documentazione è:
\begin{itemize}
    \item automatizzata,
    \item coerente con il repository principale,
    \item priva di interventi manuali,
    \item sempre aggiornata all’ultima versione approvata.
\end{itemize}

\subsubsection{Automazione tramite GitHub Actions}

Al fine di garantire coerenza, tracciabilità e aggiornamento continuo della documentazione, il gruppo ha configurato due workflow{\scriptsize\raisebox{-0.5ex}{G}} automatizzati tramite \textbf{GitHub Actions}{\scriptsize\raisebox{-0.5ex}{G}}: uno dedicato alla compilazione dei documenti \LaTeX{} e uno alla pubblicazione del sito web.
L’obiettivo è ridurre al minimo gli interventi manuali, eliminare errori operativi e mantenere un processo documentale affidabile, verificabile e riproducibile.
\paragraph{Compilazione automatica della documentazione}

Il workflow di build è attivato al verificarsi di modifiche ai file sorgenti \LaTeX{} presenti nella directory \texttt{src/}.
Il sistema esegue una serie di controlli e operazioni automatiche per garantire la coerenza tra sorgenti e PDF generati:

\begin{itemize}
    \item rileva le modifiche rispetto all’ultima build valida e identifica i file da ricompilare;
\item elimina PDF obsoleti o non più associati a sorgenti esistenti, escludendo i PDF firmati che vengono gestiti manualmente;
\item compila i progetti \LaTeX{} tramite \texttt{latexmk} in ambiente Docker{\scriptsize\raisebox{-0.5ex}{G}};
\item genera un file \texttt{report.md} contenente l'esito della compilazione per ciascun documento (con rispettivo link diretto ai pdf generati dalla build);
\item effettua automaticamente un commit{\scriptsize\raisebox{-0.5ex}{G}} dei soli file effettivamente aggiornati.
\end{itemize}

Questa procedura assicura che:

\begin{itemize}
    \item nella cartella \texttt{docs/} siano presenti solamente PDF aggiornati e validi;
\item nessun documento obsoleto rimanga nel repository;
    \item eventuali errori di compilazione impediscano la pubblicazione di versioni incoerenti.
\end{itemize}

\paragraph{Pubblicazione automatica della documentazione online}

Al termine di una compilazione \LaTeX{} completata con esito positivo, un secondo workflow provvede automaticamente alla pubblicazione della documentazione tramite GitHub Pages.
Il processo si occupa di:

\begin{itemize}
    \item sincronizzare la struttura del sito con la cartella \texttt{docs/};
\item includere automaticamente nuove versioni o nuovi documenti;
    \item rendere il contenuto immediatamente accessibile online.
\end{itemize}

La pubblicazione avviene senza intervento manuale e garantisce che solo materiale correttamente compilato e validato venga reso disponibile.
Questo meccanismo assicura un punto di accesso unico e costantemente aggiornato alla documentazione ufficiale del progetto.
% -------------------------------------------------------------------
%   REGOLE, CONVENZIONI E CICLO DI VITA DELLA DOCUMENTAZIONE
% -------------------------------------------------------------------

\subsection{Documentazione}

\subsubsection{Obiettivo}
La documentazione rappresenta un elemento fondamentale del progetto:  
definisce i processi, registra le decisioni, formalizza gli avanzamenti e costituisce un riferimento verificabile e condiviso.
Questa sezione definisce regole, convenzioni e ciclo di vita adottati dal gruppo per la produzione dei documenti ufficiali.
Poiché il progetto si trova nella fase di candidatura, tali norme costituiscono una baseline che sarà estesa con l’avanzare delle attività progettuali.
\subsubsection{Formati di riferimento}
L'intera documentazione ufficiale viene redatta in formato \LaTeX{}{\scriptsize\raisebox{-0.5ex}{G}} (\texttt{.tex}) e distribuita in formato PDF{\scriptsize\raisebox{-0.5ex}{G}} (\texttt{.pdf}).
\begin{itemize}
    \item Il formato \textbf{\LaTeX{}} garantisce qualità tipografica, modularità, controllabilità delle modifiche e standardizzazione.
\item Il formato \textbf{PDF} costituisce la versione ufficiale, consultabile e non modificabile.
\end{itemize}

Le due forme convivono con ruoli distinti:

\begin{center}
\begin{tabular}{|l|l|}
\hline
\textbf{File .tex} & Codice sorgente, modificabile, tracciabile \\ \hline
\textbf{File .pdf} & Documento ufficiale distribuito e verificabile \\ \hline
\end{tabular}
\end{center}

\subsubsection{Struttura dei Documenti}
Ogni documento prodotto segue una struttura coerente per garantire leggibilità e uniformità:

\begin{itemize}
    \item Frontespizio{\scriptsize\raisebox{-0.5ex}{G}} con dati e metadati ufficiali
    \item Tabella delle versioni e delle modifiche
    \item Indice dei contenuti
    \item Corpo del documento, articolato in sezioni e sottosezioni
    \item (Se necessario) Appendici, glossari, tabelle e riferimenti
\end{itemize}

Template{\scriptsize\raisebox{-0.5ex}{G}} condivisi sono forniti nel repository del progetto per garantire omogeneità.
\subsubsection{Convenzioni di Scrittura e Nomenclatura}
Per assicurare tracciabilità e organizzazione dei file, si adottano le seguenti convenzioni.
\paragraph{Nomi dei file}
La convenzione utilizzata è snake\_case{\scriptsize\raisebox{-0.5ex}{G}}. Quando rilevante, si includono versione e/o data:

\begin{itemize}
    \item La versione è indicata come \texttt{vX.Y}
    \item La data è riportata nel formato ISO \texttt{AAAA-MM-GG}
\end{itemize}

\noindent
\textbf{Esempi:}
\begin{itemize}
    \item \texttt{norme\_di\_progetto\_v0.1.tex}
    \item \texttt{verbale\_interno\_2025-10-29.tex}
\end{itemize}

\paragraph{Documenti firmati}
In caso di documenti soggetti a firma, la copia firmata mantiene lo stesso nome del documento ufficiale aggiungendo il suffisso \texttt{\_firmato} (o \texttt{\_signed}):

\begin{itemize}
    \item \texttt{norme\_di\_progetto\_v0.1\_firmato.pdf}
\end{itemize}

Questo consente distinzione chiara tra versione certificata e versioni operative.
\subsubsection{Ciclo di Vita dei Documenti}

Tutti i documenti prodotti dal gruppo seguono un processo di revisione volto a garantire qualità, coerenza e tracciabilità.
Il ciclo di vita adottato è iterativo e prevede controlli continui ad ogni modifica significativa.
Le fasi previste sono:

\begin{itemize}
    \item \textbf{Redazione}{\scriptsize\raisebox{-0.5ex}{G}}  
    Il documento viene creato o aggiornato da uno o più redattori.
    La stesura avviene in modo incrementale e versionato: ogni aggiornamento rilevante è committato e tracciato nel sistema di versionamento.
\item \textbf{Verifica}{\scriptsize\raisebox{-0.5ex}{G}}  
    Ogni commit contenente contenuto documentale richiede una verifica da parte di almeno un membro diverso dal redattore.
    La verifica assicura:
    \begin{itemize}
        \item coerenza con gli standard adottati;
\item correttezza dei contenuti;
        \item qualità formale e lessicale;
        \item aderenza alle norme di progetto.
\end{itemize}
    Un documento è considerato verificato quando tutte le modifiche proposte sono state esaminate e validate.
\item \textbf{Approvazione}{\scriptsize\raisebox{-0.5ex}{G}} *(solo per i documenti ufficiali)* Per i documenti formali del progetto (es. Lettera di Presentazione, Valutazione Capitolati, Preventivo dei Costi) è prevista una fase di approvazione finale, successiva alla verifica.
    L'approvazione certifica che il documento è completo, corretto e pronto per essere rilasciato nella sua versione ufficiale.
\item \textbf{Pubblicazione}{\scriptsize\raisebox{-0.5ex}{G}}  
    Dopo la verifica (per verbali) o dopo l'approvazione (per documenti ufficiali), il documento viene inserito nell'archivio PDF (con il sistema automatico descritto in precedenza), indicizzato e pubblicato sul portale del gruppo.
\end{itemize}

\subsubsection{Verbali}
I verbali{\scriptsize\raisebox{-0.5ex}{G}} sono i documenti che riportano le discussioni e le decisioni prese durante gli incontri ufficiali del gruppo.
Hanno lo scopo di tracciare l'evoluzione del progetto e formalizzare gli impegni presi.
Ogni verbale prodotto dal gruppo NightPRO deve essere strutturato nelle seguenti sezioni principali:

\begin{itemize}
    \item \textbf{Sezione 1: Informazioni Generali}
    Contiene i metadati della riunione.
    È suddivisa in:
    \begin{itemize}
        \item Componenti del Gruppo: La tabella standard con l'elenco dei membri.
\item Dettagli Riunione: Elenco puntato con Data, Ora, Luogo, Partecipanti (con eventuali assenti), Redatto da, Verificato da e Versione del verbale.
\end{itemize}

    \item \textbf{Sezione 2: Ordine del Giorno{\scriptsize\raisebox{-0.5ex}{G}} (Agenda)}
    Un elenco puntato (\texttt{\textbackslash itemize}) che elenca tutti gli argomenti pianificati per la discussione.
\item \textbf{Sezione 3: Diario della Riunione}{\scriptsize\raisebox{-0.5ex}{G}}
    Il resoconto dettagliato della discussione.
    Questa sezione è suddivisa in sottosezioni (\texttt{\textbackslash subsection}) che rispecchiano i punti dell'Ordine del Giorno, riportando le analisi e i fatti emersi.
\item \textbf{Sezione 4: Decisioni Prese}
    Un elenco numerato (\texttt{\textbackslash enumerate}) che riassume in modo chiaro e sintetico tutte le decisioni ufficiali deliberate dal gruppo durante l'incontro.
\item \textbf{Sezione 5: Attività da Svolgere (To-Do)}
    Una tabella riepilogativa (\texttt{\textbackslash table}) che assegna compiti specifici ai membri del gruppo.
    Deve includere le colonne: Attività, Assegnatario/i e Scadenza.
\end{itemize}

% -------------------------------------------------------------------
% PROCESSI ORGANIZZATIVI
% -------------------------------------------------------------------

\newpage
\section{Processi organizzativi}
\label{sec:processi_organizzativi}

\subsection{Finalità}
Questa sezione stabilisce le norme e le procedure che regolano l'organizzazione interna del gruppo.
Definisce i meccanismi di coordinamento, comunicazione e gestione delle risorse umane necessari per condurre efficacemente il progetto lungo tutto il suo sviluppo.
Le regole qui definite riguardano l'assegnazione dei ruoli, la pianificazione e il monitoraggio delle attività, la gestione degli incontri, i canali di comunicazione, nonché i processi di apprendimento e di ottimizzazione dei metodi di lavoro.

\subsection{Ambito di applicazione}
I dettagli operativi e le tempistiche specifiche relative ai processi organizzativi sono riportati nel documento \emph{Piano di Progetto}{\scriptsize\raisebox{-0.5ex}{G}}.
Le norme qui presentate coprono i seguenti ambiti:
\begin{itemize}
    \item Ruoli e responsabilità dei membri del team;
    \item Comunicazione interna ed esterna;
    \item Pianificazione e tracciamento delle attività;
    \item Sviluppo delle competenze;
    \item Ottimizzazione dei processi.
\end{itemize}

\subsection{Ruoli e responsabilità}
La distribuzione dei ruoli tra i membri del gruppo è di competenza del \emph{Responsabile di Progetto}, che deve assicurare una rotazione equa delle responsabilità.
Nel corso del progetto, ogni componente del team dovrà ricoprire almeno una volta ciascuno dei ruoli previsti.
La copertura simultanea di tutti i ruoli non è sempre necessaria, in quanto dipende dalla fase di sviluppo in cui si trova il progetto.

\paragraph{Politica di rotazione dei ruoli}
Per garantire che tutti i membri del team acquisiscano esperienza in ogni ruolo previsto, il gruppo ha definito una politica di rotazione strutturata.
Durante le prime sette settimane di progetto (Fase Iniziale), si adotta una rotazione a cadenza settimanale, con l'obiettivo di assicurare che ogni membro abbia ricoperto tutti i ruoli disponibili.
Una volta completato il ciclo di rotazione iniziale (Fase a Regime, post-settimana 7), la cadenza passa a bisettimanale.
Il team si riserva la facoltà di adottare cadenze differenti esclusivamente per gestire occasioni eccezionali o specifiche necessità progettuali urgenti.

Di seguito vengono descritti i sei ruoli previsti e le relative responsabilità.

\subsubsection{Responsabile di Progetto}
Il Responsabile di Progetto funge da interfaccia principale tra il gruppo e le parti esterne, in particolare con il \emph{committente}{\scriptsize\raisebox{-0.5ex}{G}} e l'\emph{azienda proponente}{\scriptsize\raisebox{-0.5ex}{G}}.
È responsabile dell'approvazione delle decisioni strategiche e del coordinamento generale delle attività del team.

\paragraph{Compiti principali}
Il Responsabile di Progetto si occupa della definizione degli obiettivi, della pianificazione delle scadenze e dell'allocazione delle risorse disponibili.
Coordina il lavoro del team, gestisce le risorse umane e assegna i compiti, assicurando che ogni membro abbia chiarezza sulle proprie responsabilità.
Monitora costantemente l'avanzamento del progetto, verificando che le attività vengano completate nei tempi stabiliti.
Si occupa inoltre dell'identificazione, analisi e gestione dei rischi potenziali.
Gestisce tutte le comunicazioni con il team, gli stakeholder, il committente e l'azienda proponente, organizzando e presiedendo gli incontri interni ed esterni.
Ha il compito di approvare la documentazione ufficiale prodotta dal gruppo, l'offerta economica presentata al committente e i task completati e verificati.
Suddivide le attività complessive in singole issue{\scriptsize\raisebox{-0.5ex}{G}} o task e ne gestisce l'assegnazione ai membri del gruppo.

\subsubsection{Amministratore}
L'Amministratore ha la responsabilità di controllare e amministrare l'ambiente di lavoro del gruppo, garantendo che strumenti e infrastrutture siano sempre funzionanti e accessibili.

\paragraph{Compiti principali}
L'Amministratore si occupa della salvaguardia e della gestione della documentazione di progetto, curando il sistema di archiviazione e versionamento sia per la documentazione che per il codice sorgente.
Amministra l'infrastruttura tecnologica e gli strumenti utilizzati dal team, mantenendo l'ambiente di sviluppo efficiente.
Fornisce supporto tecnico ai membri del gruppo, risolvendo errori e gestendo le segnalazioni di malfunzionamenti.
Si dedica all'automazione dei processi ricorrenti, identifica opportunità di miglioramento e implementa soluzioni per ottimizzare l'ambiente di lavoro.
Esegue il controllo delle versioni e delle configurazioni del prodotto software, gestendo il sistema di versionamento e configurazione.
Redige e attua i piani e le procedure per la gestione della qualità, assicurando l'efficacia e l'efficienza dei processi.
Fornisce inoltre supporto per la gestione delle risorse e delle comunicazioni del progetto.

\subsubsection{Analista}
L'Analista si occupa dell'analisi dei requisiti{\scriptsize\raisebox{-0.5ex}{G}} e della definizione delle specifiche del progetto, traducendo le esigenze del cliente in requisiti formali e verificabili.

\paragraph{Compiti principali}
L'Analista raccoglie e analizza i requisiti del cliente, interpretando i bisogni del proponente e trasformandoli in aspettative chiare che il gruppo deve soddisfare per realizzare un prodotto di qualità professionale.
Definisce le \emph{specifiche funzionali}{\scriptsize\raisebox{-0.5ex}{G}} e \emph{tecniche}{\scriptsize\raisebox{-0.5ex}{G}} del prodotto, sviluppa una modellazione concettuale del sistema e organizza i requisiti in categorie logiche.
Collabora strettamente con il progettista per individuare soluzioni che soddisfino i requisiti individuati.
Redige il documento \emph{Analisi dei Requisiti}{\scriptsize\raisebox{-0.5ex}{G}}, descrivendo in dettaglio i servizi che il sistema deve fornire e garantendo che tutti i requisiti siano espressi in modo chiaro, completo e non ambiguo.
Studia approfonditamente il problema e il \emph{contesto applicativo}{\scriptsize\raisebox{-0.5ex}{G}}, valutando la complessità del dominio e identificando sia i requisiti espliciti che quelli impliciti.
Assicura che i requisiti siano definiti con sufficiente precisione per evitare ambiguità nelle fasi successive di progettazione e sviluppo.

\subsubsection{Progettista}
Il Progettista ha il compito di trasformare i requisiti emersi dalla fase di analisi in un'\emph{architettura}{\scriptsize\raisebox{-0.5ex}{G}} software coerente e realizzabile.

\paragraph{Compiti principali}
Il Progettista progetta l'architettura del sistema in modo che soddisfi tutti i requisiti identificati, privilegiando soluzioni con elevata manutenibilità e ridotto \emph{accoppiamento}{\scriptsize\raisebox{-0.5ex}{G}} tra i componenti.
Prende decisioni riguardanti gli aspetti tecnici e tecnologici del progetto, selezionando le tecnologie più appropriate per garantire efficacia ed efficienza.
Valuta e seleziona eventuali \emph{pattern architetturali}{\scriptsize\raisebox{-0.5ex}{G}} da adottare e sviluppa lo \emph{schema UML}{\scriptsize\raisebox{-0.5ex}{G}} delle classi che modella la struttura del sistema.
Garantisce che la soluzione proposta sia economicamente sostenibile e mantenibile, rispettando i vincoli di budget definiti nel preventivo.
Collabora con gli analisti per comprendere appieno i requisiti e con i programmatori per guidare l'implementazione delle soluzioni tecniche.
Coordina le attività di progettazione per assicurare che il prodotto finale soddisfi tutti i requisiti.
Si dedica inoltre alla ricerca e all'approfondimento delle conoscenze tecniche, esplorando strumenti e tecnologie innovative che possano migliorare l'architettura del sistema.

\subsubsection{Programmatore}
Il Programmatore realizza l'implementazione concreta delle soluzioni tecniche definite dal progettista, trasformando le specifiche di progettazione in codice funzionante.

\paragraph{Compiti principali}
Il Programmatore scrive codice che aderisce rigorosamente alle specifiche di progettazione, applicando le "\emph{best practices}{\scriptsize\raisebox{-0.5ex}{G}}" consolidate nel settore per garantire qualità e leggibilità.
Risolve i problemi tecnici che emergono durante lo sviluppo, trovando soluzioni efficaci e compatibili con l'architettura definita.
Esegue test sul proprio codice per verificare che funzioni correttamente e soddisfi i requisiti specificati.
Scrive codice ben documentato, mantenibile e correttamente versionato, facilitando così le attività di verifica e manutenzione.
Redige la documentazione necessaria per la comprensione e l'utilizzo del codice prodotto.
Collabora attivamente con il progettista e gli altri membri del team per implementare correttamente l'architettura definita nella fase di progettazione.

\subsubsection{Verificatore}
Il Verificatore controlla il lavoro prodotto dagli altri membri del gruppo, assicurando che rispetti gli standard di qualità e le norme di progetto stabilite.

\paragraph{Compiti principali}
Il Verificatore esamina ogni file caricato in un \emph{branch}{\scriptsize\raisebox{-0.5ex}{G}} protetto della repository per verificarne la conformità alle Norme di Progetto.
Durante la revisione di una \emph{pull request}{\scriptsize\raisebox{-0.5ex}{G}}, controlla i file modificati o aggiunti per individuare errori ortografici, sintattici, logici e problemi di build.
Esamina i prodotti in fase di revisione utilizzando le tecniche e gli strumenti definiti nelle Norme di Progetto, verificando che siano conformi ai requisiti funzionali e di qualità.
Segnala eventuali errori o non conformità riscontrati durante la verifica, fornendo \emph{feedback}{\scriptsize\raisebox{-0.5ex}{G}} dettagliati e costruttivi a chi ha prodotto il lavoro.
Assicura che la qualità di quanto prodotto sia conforme agli standard imposti, verificando la conformità in ogni fase del ciclo di vita del prodotto.
Redige la parte retrospettiva del \emph{Piano di Qualifica}{\scriptsize\raisebox{-0.5ex}{G}}, documentando le verifiche e le prove effettuate.
Mantiene una sorveglianza continua durante l'intera durata del progetto, garantendo che tutte le attività rispettino il livello di qualità atteso.

\subsection{Comunicazione e incontri}
La gestione efficace della comunicazione e degli incontri è fondamentale per garantire il coordinamento tra i membri del team e con le parti esterne.
Questa sezione definisce i canali e le modalità di comunicazione adottate dal gruppo, distinguendo tra comunicazioni interne ed esterne.

\subsubsection{Canali di comunicazione interna}
Le comunicazioni tra i membri del gruppo avvengono principalmente attraverso due strumenti, scelti in base alle esigenze specifiche.

\paragraph{Telegram}
Telegram è il canale principale per le comunicazioni asincrone interne.
Viene utilizzato per scambi rapidi di messaggi testuali e vocali, nonché per la pianificazione e l'organizzazione degli incontri interni.
La scelta di questo strumento è stata motivata dalla disponibilità della funzionalità topic, che consente di organizzare le conversazioni per argomento, migliorando l'ordine e la tracciabilità delle discussioni.

\paragraph{Google Meet}
Google Meet viene utilizzato per le riunioni sincrone interne quando è necessario un confronto diretto tra i membri del gruppo.
Questo strumento permette la partecipazione simultanea di tutti i membri e offre funzionalità di condivisione schermo particolarmente utili durante attività collaborative.

\subsubsection{Canali di comunicazione esterna}
Le comunicazioni con il proponente e i committenti sono gestite dal Responsabile di Progetto e avvengono attraverso canali dedicati.

\paragraph{Google Meet}
Google Meet è utilizzato per le riunioni esterne con il proponente e i committenti.
Queste riunioni possono essere richieste da entrambe le parti in base alle necessità del progetto.
Al termine di ogni riunione esterna viene redatto un verbale per documentare gli argomenti trattati e le decisioni prese.

\paragraph{Email}
L'email viene utilizzata per comunicazioni formali e dettagliate con le parti esterne.
L'indirizzo email del gruppo (\texttt{swe.nightpro@gmail.com}) è condiviso tra tutti i membri del team per garantire trasparenza e accessibilità alle comunicazioni ufficiali.

\subsubsection{Organizzazione degli incontri}
Gli incontri del gruppo si distinguono in interni ed esterni, ciascuno con modalità e finalità specifiche.

\paragraph{Riunioni interne}
Le riunioni interne coinvolgono esclusivamente i membri del gruppo e si svolgono con cadenza regolare, tipicamente almeno una volta a settimana.
Qualsiasi membro del gruppo può richiedere un incontro aggiuntivo al Responsabile di Progetto, che provvederà a organizzarlo in base alla disponibilità di tutti i partecipanti.
Le riunioni si svolgono prevalentemente in modalità virtuale tramite Google Meet.

Per garantire efficienza e produttività, ogni riunione interna segue una struttura ben definita.
Prima dell'incontro viene preparata una scaletta con i principali punti da discutere.
Durante la riunione si discute del lavoro svolto da ogni membro dall'ultimo incontro e si affrontano i punti previsti nella scaletta, con spazio per il confronto su eventuali dubbi o problematiche emerse.
Infine, si pianificano le attività da svolgere per ogni membro fino al prossimo incontro.

Al termine di ogni riunione interna, viene designato un membro del gruppo incaricato di redigere il verbale, contenente una descrizione dei punti principali discussi durante l'incontro.

\paragraph{Riunioni esterne}
Le riunioni esterne coinvolgono i membri del gruppo insieme ai referenti aziendali o ai committenti.
La frequenza e le modalità di svolgimento di questi incontri verranno stabilite in accordo con il proponente.
Entrambe le parti possono richiedere incontri aggiuntivi quando necessario.
Le riunioni esterne si svolgono principalmente in modalità virtuale tramite Google Meet.
Al termine di ogni riunione esterna viene redatto un verbale che documenta i momenti salienti e le decisioni prese; quando necessario, il verbale viene sottoposto ai referenti per approvazione.

\subsubsection{Responsabilità durante gli incontri}
Durante gli incontri, sia il Responsabile di Progetto che i partecipanti hanno responsabilità specifiche da rispettare.

\paragraph{Responsabilità del Responsabile di Progetto}
Il Responsabile di Progetto ha il compito di pianificare l'ordine del giorno delle riunioni, comunicare tempestivamente eventuali variazioni orarie e verificare la presenza dei membri.
Durante la riunione, guida le discussioni in modo ordinato ed efficace, assicurando che tutti i punti previsti vengano affrontati.
Al termine dell'incontro, nomina un segretario incaricato di redigere il verbale e, una volta completato, lo approva formalmente.

\paragraph{Responsabilità dei partecipanti}
I partecipanti alle riunioni si impegnano a partecipare puntualmente, comunicando tempestivamente eventuali ritardi o assenze.
Durante gli incontri devono partecipare attivamente alle discussioni e mantenere un comportamento professionale e rispettoso.

\subsection{Pianificazione e tracciamento delle attività}
Una gestione efficace dei compiti e dei task è essenziale per mantenere l'organizzazione del progetto, garantendo tracciabilità, qualità e rispetto delle tempistiche.
Questa sezione illustra l'approccio metodologico adottato dal team e descrive il processo che regola lo svolgimento di ciascun task.

\subsubsection{Approccio metodologico}
Il team adotta il metodo di sviluppo \emph{Agile}{\scriptsize\raisebox{-0.5ex}{G}} per organizzare e pianificare le attività progettuali.
Questo approccio favorisce una gestione flessibile e iterativa, incoraggiando la collaborazione continua tra i membri del team e una comunicazione trasparente con il committente.

I task vengono definiti, pianificati e assegnati utilizzando \textbf{GitHub Projects}{\scriptsize\raisebox{-0.5ex}{G}}, uno strumento che agevola la condivisione delle responsabilità e il lavoro collaborativo.
Il sistema permette di visualizzare lo stato delle attività e di tracciare l'avanzamento, individuando rapidamente eventuali blocchi o ritardi.
Il team valuta periodicamente i risultati ottenuti e pianifica i miglioramenti per i cicli successivi, garantendo un adattamento flessibile ai cambiamenti nei requisiti e promuovendo il coinvolgimento attivo di tutti i partecipanti.

\subsubsection{Processo di gestione dei task}
Il \emph{ciclo di vita}{\scriptsize\raisebox{-0.5ex}{G}} di ciascun task segue un processo strutturato, gestito attraverso GitHub Projects per assicurare coerenza e tracciabilità.

Il processo inizia con la \textbf{creazione} del task da parte del Responsabile, che definisce su GitHub Projects tutti i campi essenziali: titolo, descrizione, priorità, assegnatari e stima temporale, associandolo eventualmente a una milestone{\scriptsize\raisebox{-0.5ex}{G}}.

Nella fase di \textbf{assegnazione}, i task vengono distribuiti ai membri del team in base alle loro competenze e al carico di lavoro corrente.
Se un task rimane non assegnato, i membri possono prendersene carico autonomamente.

Durante l'\textbf{esecuzione}, l'assegnatario aggiorna lo stato del task (ad esempio da "To Do" a "In Progress") e lavora su un branch dedicato, garantendo una gestione separata e tracciabile delle modifiche.

Una volta completato, il task entra in fase di \textbf{revisione}.
Il Verificatore esamina il lavoro prodotto, segnala eventuali modifiche necessarie tramite commenti o review e valida il task se conforme agli standard stabiliti.

Infine, nella fase di \textbf{accettazione}, il Responsabile approva il task completato, esegue il merge del branch quando pertinente e aggiorna lo stato del task a "Completato".

Questo processo strutturato consente al team di gestire le attività con precisione, tracciando ogni modifica e mantenendo una visione chiara e condivisa dell'avanzamento.

\subsection{Sviluppo delle competenze}
Il processo di formazione e aggiornamento dei membri del team è fondamentale per garantire che tutti possiedano le competenze necessarie per svolgere efficacemente le proprie attività.

\subsubsection{Obiettivi formativi}
Il processo di formazione mira a sviluppare le competenze necessarie per la produzione della documentazione e per la realizzazione del prodotto software richiesto.
Gli obiettivi formativi includono:
\begin{itemize}
    \item l'acquisizione di una conoscenza solida del linguaggio \LaTeX{};
    \item lo sviluppo di dimestichezza con i linguaggi di programmazione, le librerie e gli strumenti necessari per il prodotto software assegnato dal proponente;
    \item la familiarizzazione con l'ambiente di lavoro relativo al capitolato di interesse.
\end{itemize}

\subsubsection{Modalità di apprendimento}
Il gruppo adotta un approccio flessibile alla formazione, permettendo a ciascun membro di scegliere il metodo di apprendimento più adatto alle proprie esigenze.
Questa libertà consente di ampliare la visione del gruppo sulle tecnologie utilizzate e migliorare le scelte implementative attraverso l'esperienza diretta.
La rotazione periodica dei ruoli, come definita nella sezione "Ruoli e responsabilità", garantisce che ogni membro del team acquisisca esperienza pratica in tutti i ruoli previsti.

\subsection{Ottimizzazione dei processi}
Il gruppo adotta un approccio di miglioramento continuo per ottimizzare costantemente i processi di lavoro e la qualità dei prodotti.

\subsubsection{Principio del miglioramento continuo}
Durante lo svolgimento delle attività e l'elaborazione della documentazione, il team si impegna a seguire costantemente il principio di miglioramento continuo.
L'obiettivo è identificare proattivamente opportunità di ottimizzazione nei processi, nei ruoli e nelle attività, sviluppando soluzioni innovative per affrontare le sfide che emergono durante il progetto.

Il miglioramento continuo si configura come un ciclo iterativo che permette al team di adattarsi dinamicamente alle esigenze mutevoli del progetto.
Questo approccio garantisce una crescita costante e un incremento dell'efficienza nelle attività svolte, caratterizzando ogni fase del processo dalla pianificazione all'esecuzione, inclusa la documentazione.

\subsubsection{Meccanismi di implementazione}
L'implementazione del miglioramento continuo avviene attraverso meccanismi strutturati che coinvolgono tutto il team.

Le problematiche riscontrate durante lo sviluppo e le soluzioni adottate vengono sistematicamente documentate e analizzate.
Questo permette una gestione consapevole e mirata delle sfide incontrate, evitando la ripetizione di errori già commessi e fornendo un repertorio di soluzioni efficaci per situazioni simili.

Durante le riunioni periodiche, il team discute le attività svolte nell'ultimo periodo, coinvolgendo tutti i membri nell'identificazione delle aree di successo e delle opportunità di miglioramento.
L'obiettivo è formulare azioni correttive concrete da implementare, promuovendo un feedback costante e un adattamento continuo che migliora progressivamente le prestazioni del team.

Le azioni correttive stabilite durante le riunioni vengono implementate e successivamente valutate per la loro efficacia.
I risultati di questa valutazione vengono esaminati durante le riunioni successive, creando un ciclo virtuoso di apprendimento e miglioramento.
Le decisioni prese e le azioni correttive vengono tracciate attraverso il sistema di gestione dei task (GitHub Projects) e documentate nei verbali delle riunioni, garantendo tracciabilità e continuità nel processo di ottimizzazione.

\end{document}