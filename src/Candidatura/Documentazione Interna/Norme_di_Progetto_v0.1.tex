\documentclass[a4paper, 11pt, oneside]{scrartcl} % Classe KOMA-Script

% --- Pacchetti Fondamentali ---
\usepackage[utf8]{inputenc}     % Codifica UTF-8
\usepackage[T1]{fontenc}        % Font encoding moderno
\usepackage[italian]{babel}     % Lingua italiana
\usepackage{lmodern}            % Font "Latin Modern"

% --- Grafica e Layout ---
\usepackage{graphicx}           % Per le immagini
\graphicspath{{../../assets/}}
\usepackage[a4paper, top=2.5cm, bottom=3cm, left=2.5cm, right=2.5cm]{geometry} % Margini
\usepackage{fancyhdr}           % Per header e footer personalizzati
\usepackage{microtype}          % Migliora la tipografia
\usepackage[svgnames]{xcolor}   % Colori

% --- Utility ---
\usepackage{booktabs}           % Tabelle più professionali
\usepackage{enumitem}           % Per personalizzare liste
\usepackage{hyperref}           % Rende i link cliccabili
\hypersetup{
    colorlinks=true,
    linkcolor=DarkBlue,
    filecolor=DarkBlue,      
    urlcolor=DarkBlue,
    citecolor=DarkBlue,
    pdftitle={Documento Progetto - NightPRO},
    pdfauthor={Gruppo NightPRO},
}


% ===================================================================
%  HEADER E FOOTER
% ===================================================================
\pagestyle{fancy}
\fancyhf{} % Pulisce i campi
\fancyhead[L]{\textbf{NightPRO – Progetto Ingegneria del Software}}
\fancyhead[R]{Anno Accademico 2025/2026}
\fancyfoot[C]{\thepage} % Numero di pagina al centro
\renewcommand{\headrulewidth}{0.4pt}
\renewcommand{\footrulewidth}{0pt}

% ===================================================================
%  INIZIO DOCUMENTO
% ===================================================================
\begin{document}

% -------------------------------------------------------------------
%  FRONTESPIZIO
% -------------------------------------------------------------------
\thispagestyle{empty}
\begin{titlepage}
    \centering
    \vspace*{1cm}
    \includegraphics[width=0.35\textwidth]{logo.png}\\[1cm]

     \vfill
    
    {\small UNIVERSITÀ DEGLI STUDI DI PADOVA \par}
    {\small CORSO DI LAUREA IN INFORMATICA (L-31) \par}
    \vspace{0.5cm}
    {\large Corso di Ingegneria del Software \par}
    {\small Anno Accademico 2025/2026 \par}
    \vfill
    
    {\Huge \bfseries Norme di Progetto \par}
        \vspace{1cm}
         {\Large Redattori: Giovanni Ponso, Davide Biasuzzi \par} 
    {\Large Verificato da:  \par} 
    {\Large Approvato da: \par}
    \vfill

    {\Large \bfseries Gruppo: NightPRO}    \vspace{0.5cm}

    {\large \href{mailto:swe.nightpro@gmail.com}{swe.nightpro@gmail.com}}\\[2cm]

        {\large Data: 2025-11-05 \par}

     {\Large Versione: 0.1 \par} 

\end{titlepage}

%  SEZIONE: Tabella delle Versioni
% -------------------------------------------------------------------
\newpage
\pagestyle{fancy}
\phantomsection
\addcontentsline{toc}{section}{Tabella delle Versioni}
\section*{Tabella delle Versioni}
\vspace{0.2cm} 
\begin{center}
\resizebox{\textwidth}{!}{
\renewcommand{\arraystretch}{1.2}
\begin{tabular}{@{}llp{0.25\textwidth}p{0.45\textwidth}c@{}} 
\toprule
\textbf{Versione} & \textbf{Data} & \textbf{Autore/i} & \textbf{Descrizione delle Modifiche} & \textbf{Verificatore} \\
\midrule
0.1 & 2025-11-05 & G. Ponso; D. Biasuzzi & Creazione documento + redazione processi di supporto &  - \\
\bottomrule
\end{tabular}
}
\end{center}


\newpage
\tableofcontents % Genera l'indice
\pagestyle{fancy}

% -------------------------------------------------------------------
%  INFORMAZIONI GENERALI
% -------------------------------------------------------------------
\newpage
\section{Informazioni Generali}

\subsection{Componenti del Gruppo}

\begin{table}[h!]
\centering
\renewcommand{\arraystretch}{1.2} % più spazio tra le righe
\begin{tabular}{@{}llc@{}}
\toprule
\textbf{Cognome} & \textbf{Nome} & \textbf{Matricola} \\
\midrule
Biasuzzi & Davide & 2111000 \\
Bilato & Leonardo & 2071084 \\
Zanella & Francesco & 2116442 \\
Romascu & Mihaela-Mariana & 2079726 \\
Ogniben & Michele & 2042325 \\
Perozzo & Samuele & 2110989 \\
Ponso & Giovanni & 2000558 \\
\bottomrule
\end{tabular}
\caption{Componenti del gruppo NightPRO.}
\end{table}

% -------------------------------------------------------------------
% INTRODUZIONE
% -------------------------------------------------------------------

\newpage
\section{Introduzione}
\label{sec:introduzione}

\subsection{Scopo del documento}
Questo documento definisce le Norme di Progetto del gruppo NightPRO per il corso di Ingegneria del Software (A.A.~2025/2026). Stabilisce in forma univoca: metodo di lavoro, regole redazionali e criteri di gestione dei documenti, così da garantire coerenza, tracciabilità e chiarezza operativa. La presente versione costituisce la base metodologica iniziale e sarà aggiornata con l’avanzare delle attività.

\subsection{Stato del progetto}
Al momento il capitolato non è assegnato. Il documento disciplina quindi l’assetto organizzativo e documentale preliminare: modalità di collaborazione, convenzioni redazionali e pubblicazione dei deliverable. Le sezioni relative a prodotto, pianificazione dettagliata e processi di qualità, verifica e validazione saranno integrate nelle versioni successive, in coerenza con il capitolato che verrà attribuito.

\subsection{Glossario}

Per garantire chiarezza terminologica e uniformità nella comprensione dei concetti utilizzati, viene redatto un Glossario contenente i termini tecnici e potenzialmente ambigui.

Alla \textbf{prima occorrenza} nel documento, ogni termine seguito dal simbolo {\scriptsize\raisebox{-0.5ex}{G}} viene definito formalmente nel glossario. Le occorrenze successive non riportano tale marcatura.

% -------------------------------------------------------------------
% PROCESSI DI SUPPORTO
% -------------------------------------------------------------------
\newpage
\section{Processi di Supporto}

% -------------------------------------------------------------------
%      STRUTTURA AMBIENTE DI SVILUPPO PROGETTO (FASE PRELIMINARE)
% -------------------------------------------------------------------
 
\subsection{Ambiente collaborativo e infrastruttura del progetto}

\subsubsection{Obiettivo}
L'obiettivo di questa sezione è definire l'insieme di strumenti e configurazioni adottate dal gruppo per supportare la collaborazione, la comunicazione, l’organizzazione e l’integrazione tecnica del progetto.  
Pur trovandosi in fase preliminare, il team ha ritenuto fondamentale predisporre fin da subito un ambiente robusto e strutturato che garantisca tracciabilità, trasparenza, automazione e continuità operativa per l’intera durata del progetto.

\subsubsection{Comunicazione sincrona e asincrona}

\paragraph{Telegram}
Il gruppo utilizza Telegram come canale di comunicazione principale.  
La scelta è motivata dalla disponibilità della funzionalità topic, che permette di suddividere le conversazioni per argomento (es. documentazione, riunioni, diario di bordo), garantendo ordine e tracciabilità.

\paragraph{Google Meet}
Per le riunioni sincrone viene adottato Google Meet, in quanto:
\begin{itemize}
    \item consente la partecipazione rapida e trasversale ai membri del gruppo;
    \item offre funzionalità di condivisione schermo utili durante attività collaborative;
    \item supporta riunioni sia programmate che estemporanee.
\end{itemize}

\subsubsection{Gestione del repository e versionamento}
Il gruppo ha istituito un’organizzazione \textbf{GitHub} denominata \textbf{NightPRO}, all’interno della quale sono presenti due repository distinti:

\begin{itemize}
    \item \textbf{Documentazione}: dedicato alla gestione della documentazione di progetto, configurazione del sito pubblico e automazione CI;
    \item \textbf{Prodotto software}: inizialmente vuoto, verrà popolato quando il capitolato verrà assegnato.
\end{itemize}

La separazione è stata decisa per garantire indipendenza tra le attività documentali e quelle di sviluppo software, favorendo chiarezza, modularità e controllo delle versioni.

\paragraph{Struttura del repository documentazione}
La struttura adottata è la seguente:

\begin{verbatim}
.github/workflows/   # Workflow CI (build PDF, deploy sito)
src/                 # File .tex sorgenti dei documenti
docs/                # PDF generati automaticamente
site/                # Codice sorgente sito GitHub Pages
template/            # Template e documenti base
report.md            # Report compilazione automatica
\end{verbatim}

\subsubsection{Gestione delle attività}
Per la pianificazione, il monitoraggio delle attività e la tracciabilità dello stato di avanzamento, il gruppo utilizza \textbf{GitHub Projects}.

Questa piattaforma permette di:

\begin{itemize}
    \item assegnare responsabilità chiare ai membri del gruppo;
    \item monitorare l'avanzamento delle attività;
    \item mantenere una visione condivisa delle priorità;
    \item aggiornare in tempo reale lo stato dei task.
\end{itemize}

In questa fase iniziale il flusso operativo è in definizione e verrà progressivamente raffinato in base alle esigenze progettuali e all’evoluzione delle attività.  
L'obiettivo è arrivare a un sistema di lavoro che garantisca trasparenza, coordinamento ed efficienza, riducendo il rischio di sovrapposizioni o attività non monitorate.

\subsubsection{Sistema di pubblicazione e consultazione della documentazione}
Per garantire un accesso semplice, centralizzato e sempre aggiornato alla documentazione di progetto, è stato sviluppato un sito web statico pubblicato tramite \textbf{GitHub Pages}.

\paragraph{Obiettivo}
Rendere la documentazione facilmente consultabile da tutti gli stakeholder (membri del team, committenti e corpo docente), mantenendo coerenza, ordine e tracciabilità delle versioni.

\paragraph{Funzionalità principali}
Il sito offre:

\begin{itemize}
    \item navigazione dei documenti tramite struttura a cartelle collassabili;
    \item motore di ricerca interno per nome, data o versione del documento;
    \item possibilità di apertura o download diretto dei PDF;
    \item pagina informativa con i membri del gruppo e relativi riferimenti;
    \item selezione del tema grafico chiaro, scuro o sistema.
\end{itemize}

\paragraph{Aggiornamento automatico}
La struttura del sito e l’elenco dei documenti vengono aggiornati automaticamente ad ogni esecuzione della pipeline di pubblicazione.  
Durante la build, un processo analizza la cartella \texttt{docs/} e genera la mappa della documentazione pubblicata, garantendo che il sito rifletta sempre lo stato più recente e valido dei file.

Grazie a questa configurazione, la consultazione della documentazione è:
\begin{itemize}
    \item automatizzata,
    \item coerente con il repository principale,
    \item priva di interventi manuali,
    \item sempre aggiornata all’ultima versione approvata.
\end{itemize}

\subsubsection{Automazione tramite GitHub Actions}

Al fine di garantire coerenza, tracciabilità e aggiornamento continuo della documentazione, il gruppo ha configurato due workflow automatizzati tramite \textbf{GitHub Actions}: uno dedicato alla compilazione dei documenti \LaTeX{} e uno alla pubblicazione del sito web.

L’obiettivo è ridurre al minimo gli interventi manuali, eliminare errori operativi e mantenere un processo documentale affidabile, verificabile e riproducibile.

\paragraph{Compilazione automatica della documentazione}

Il workflow di build è attivato al verificarsi di modifiche ai file sorgenti \LaTeX{} presenti nella directory \texttt{src/}.  
Il sistema esegue una serie di controlli e operazioni automatiche per garantire la coerenza tra sorgenti e PDF generati:

\begin{itemize}
    \item rileva le modifiche rispetto all’ultima build valida e identifica i file da ricompilare;
    \item elimina PDF obsoleti o non più associati a sorgenti esistenti, escludendo i PDF firmati che vengono gestiti manualmente;
    \item compila i progetti \LaTeX{} tramite \texttt{latexmk} in ambiente Docker;
    \item genera un file \texttt{report.md} contenente l'esito della compilazione per ciascun documento (con rispettivo link diretto ai pdf generati dalla build);
    \item effettua automaticamente un commit dei soli file effettivamente aggiornati.
\end{itemize}

Questa procedura assicura che:

\begin{itemize}
    \item nella cartella \texttt{docs/} siano presenti solamente PDF aggiornati e validi;
    \item nessun documento obsoleto rimanga nel repository;
    \item eventuali errori di compilazione impediscano la pubblicazione di versioni incoerenti.
\end{itemize}

\paragraph{Pubblicazione automatica della documentazione online}

Al termine di una compilazione \LaTeX{} completata con esito positivo, un secondo workflow provvede automaticamente alla pubblicazione della documentazione tramite GitHub Pages.

Il processo si occupa di:

\begin{itemize}
    \item sincronizzare la struttura del sito con la cartella \texttt{docs/};
    \item includere automaticamente nuove versioni o nuovi documenti;
    \item rendere il contenuto immediatamente accessibile online.
\end{itemize}

La pubblicazione avviene senza intervento manuale e garantisce che solo materiale correttamente compilato e validato venga reso disponibile.  
Questo meccanismo assicura un punto di accesso unico e costantemente aggiornato alla documentazione ufficiale del progetto.

% -------------------------------------------------------------------
%   REGOLE, CONVENZIONI E CICLO DI VITA DELLA DOCUMENTAZIONE
% -------------------------------------------------------------------

\subsection{Documentazione}

\subsubsection{Obiettivo}
La documentazione rappresenta un elemento fondamentale del progetto:  
definisce i processi, registra le decisioni, formalizza gli avanzamenti e costituisce un riferimento verificabile e condiviso.

Questa sezione definisce regole, convenzioni e ciclo di vita adottati dal gruppo per la produzione dei documenti ufficiali.  
Poiché il progetto si trova nella fase di candidatura, tali norme costituiscono una baseline che sarà estesa con l’avanzare delle attività progettuali.

\subsubsection{Formati di riferimento}
L'intera documentazione ufficiale viene redatta in formato \LaTeX{} (\texttt{.tex}) e distribuita in formato PDF (\texttt{.pdf}).

\begin{itemize}
    \item Il formato \textbf{\LaTeX{}} garantisce qualità tipografica, modularità, controllabilità delle modifiche e standardizzazione.
    \item Il formato \textbf{PDF} costituisce la versione ufficiale, consultabile e non modificabile.
\end{itemize}

Le due forme convivono con ruoli distinti:

\begin{center}
\begin{tabular}{|l|l|}
\hline
\textbf{File .tex} & Codice sorgente, modificabile, tracciabile \\ \hline
\textbf{File .pdf} & Documento ufficiale distribuito e verificabile \\ \hline
\end{tabular}
\end{center}

\subsubsection{Struttura dei Documenti}
Ogni documento prodotto segue una struttura coerente per garantire leggibilità e uniformità:

\begin{itemize}
    \item Frontespizio con dati e metadati ufficiali
    \item Tabella delle versioni e delle modifiche
    \item Indice dei contenuti
    \item Corpo del documento, articolato in sezioni e sottosezioni
    \item (Se necessario) Appendici, glossari, tabelle e riferimenti
\end{itemize}

Template condivisi sono forniti nel repository del progetto per garantire omogeneità.

\subsubsection{Convenzioni di Scrittura e Nomenclatura}
Per assicurare tracciabilità e organizzazione dei file, si adottano le seguenti convenzioni.

\paragraph{Nomi dei file}
La convenzione utilizzata è snake\_case. Quando rilevante, si includono versione e/o data:

\begin{itemize}
    \item La versione è indicata come \texttt{vX.Y}
    \item La data è riportata nel formato ISO \texttt{AAAA-MM-GG}
\end{itemize}

\noindent
\textbf{Esempi:}
\begin{itemize}
    \item \texttt{norme\_di\_progetto\_v0.1.tex}
    \item \texttt{verbale\_interno\_2025-10-29.tex}
\end{itemize}

\paragraph{Documenti firmati}
In caso di documenti soggetti a firma, la copia firmata mantiene lo stesso nome del documento ufficiale aggiungendo il suffisso \texttt{\_firmato} (o \texttt{\_signed}):

\begin{itemize}
    \item \texttt{norme\_di\_progetto\_v0.1\_firmato.pdf}
\end{itemize}

Questo consente distinzione chiara tra versione certificata e versioni operative.

\subsubsection{Ciclo di Vita dei Documenti}

Tutti i documenti prodotti dal gruppo seguono un processo di revisione volto a garantire qualità, coerenza e tracciabilità.  
Il ciclo di vita adottato è iterativo e prevede controlli continui ad ogni modifica significativa.

Le fasi previste sono:

\begin{itemize}
    \item \textbf{Redazione}  
    Il documento viene creato o aggiornato da uno o più redattori.  
    La stesura avviene in modo incrementale e versionato: ogni aggiornamento rilevante è committato e tracciato nel sistema di versionamento.

    \item \textbf{Verifica}  
    Ogni commit contenente contenuto documentale richiede una verifica da parte di almeno un membro diverso dal redattore.  
    La verifica assicura:
    \begin{itemize}
        \item coerenza con gli standard adottati;
        \item correttezza dei contenuti;
        \item qualità formale e lessicale;
        \item aderenza alle norme di progetto.
    \end{itemize}
    Un documento è considerato verificato quando tutte le modifiche proposte sono state esaminate e validate.

    \item \textbf{Approvazione} *(solo per i documenti ufficiali)*  
    Per i documenti formali del progetto (es. Lettera di Presentazione, Valutazione Capitolati, Preventivo dei Costi) è prevista una fase di approvazione finale, successiva alla verifica.  
    L'approvazione certifica che il documento è completo, corretto e pronto per essere rilasciato nella sua versione ufficiale.

    \item \textbf{Pubblicazione}  
    Dopo la verifica (per verbali) o dopo l'approvazione (per documenti ufficiali), il documento viene inserito nell'archivio PDF (con il sistema automatico descritto in precedenza), indicizzato e pubblicato sul portale del gruppo.
\end{itemize}

\subsubsection{Verbali}
I verbali sono i documenti che riportano le discussioni e le decisioni prese durante gli incontri ufficiali del gruppo. Hanno lo scopo di tracciare l'evoluzione del progetto e formalizzare gli impegni presi.

Ogni verbale prodotto dal gruppo NightPRO deve essere strutturato nelle seguenti sezioni principali:

\begin{itemize}
    \item \textbf{Sezione 1: Informazioni Generali}
    Contiene i metadati della riunione. È suddivisa in:
    \begin{itemize}
        \item Componenti del Gruppo: La tabella standard con l'elenco dei membri.
        \item Dettagli Riunione: Elenco puntato con Data, Ora, Luogo, Partecipanti (con eventuali assenti), Redatto da, Verificato da e Versione del verbale.
    \end{itemize}

    \item \textbf{Sezione 2: Ordine del Giorno (Agenda)}
    Un elenco puntato (\texttt{\textbackslash itemize}) che elenca tutti gli argomenti pianificati per la discussione.

    \item \textbf{Sezione 3: Diario della Riunione}
    Il resocon
    to dettagliato della discussione. Questa sezione è suddivisa in sottosezioni (\texttt{\textbackslash subsection}) che rispecchiano i punti dell'Ordine del Giorno, riportando le analisi e i fatti emersi.

    \item \textbf{Sezione 4: Decisioni Prese}
    Un elenco numerato (\texttt{\textbackslash enumerate}) che riassume in modo chiaro e sintetico tutte le decisioni ufficiali deliberate dal gruppo durante l'incontro.

    \item \textbf{Sezione 5: Attività da Svolgere (To-Do)}
    Una tabella riepilogativa (\texttt{\textbackslash table}) che assegna compiti specifici ai membri del gruppo. Deve includere le colonne: Attività, Assegnatario/i e Scadenza.
\end{itemize}

% -------------------------------------------------------------------
% PROCESSI ORGANIZZATIVI
% -------------------------------------------------------------------


%\newpage
%\section{Processi organizzativi}
%(da fare)


\end{document}
