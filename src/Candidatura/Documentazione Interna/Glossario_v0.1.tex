\documentclass[a4paper, 11pt, oneside]{scrartcl} % Classe KOMA-Script

% --- Pacchetti Fondamentali ---
\usepackage[utf8]{inputenc}     % Codifica UTF-8
\usepackage[T1]{fontenc}        % Font encoding moderno
\usepackage[italian]{babel}     % Lingua italiana
\usepackage{lmodern}            % Font "Latin Modern"

% --- Grafica e Layout ---
\usepackage{graphicx}           % Per le immagini
\graphicspath{{../../assets/}}
\usepackage[a4paper, top=2.5cm, bottom=3cm, left=2.5cm, right=2.5cm]{geometry} % Margini
\usepackage{fancyhdr}           % Per header e footer personalizzati
\usepackage{microtype}          % Migliora la tipografia
\usepackage[svgnames]{xcolor}   % Colori

% --- Utility ---
\usepackage{booktabs}           % Tabelle più professionali
\usepackage{enumitem}           % Per personalizzare liste
\usepackage{hyperref}           % Rende i link cliccabili
\hypersetup{
    colorlinks=true,
    linkcolor=DarkBlue,
    filecolor=DarkBlue,      
    urlcolor=DarkBlue,
    citecolor=DarkBlue,
    pdftitle={Documento Progetto - NightPRO},
    pdfauthor={Gruppo NightPRO},
}


% ===================================================================
%  HEADER E FOOTER
% ===================================================================
\pagestyle{fancy}
\fancyhf{} % Pulisce i campi
\fancyhead[L]{\textbf{NightPRO – Progetto Ingegneria del Software}}
\fancyhead[R]{Anno Accademico 2025/2026}
\fancyfoot[C]{\thepage} % Numero di pagina al centro
\renewcommand{\headrulewidth}{0.4pt}
\renewcommand{\footrulewidth}{0pt}

% ===================================================================
%  INIZIO DOCUMENTO
% ===================================================================
\begin{document}

% -------------------------------------------------------------------
%  FRONTESPIZIO
% -------------------------------------------------------------------
\thispagestyle{empty}
\begin{titlepage}
    \centering
    \vspace*{1cm}
    \includegraphics[width=0.35\textwidth]{logo.png}\\[1cm]

     \vfill
    
    {\small UNIVERSITÀ DEGLI STUDI DI PADOVA \par}
    {\small CORSO DI LAUREA IN INFORMATICA (L-31) \par}
    \vspace{0.5cm}
    {\large Corso di Ingegneria del Software \par}
    {\small Anno Accademico 2025/2026 \par}
    \vfill
    
    {\Huge \bfseries Glossario \par}
        \vspace{1cm}
         {\Large Redattori: Giovanni Ponso \par} 
    {\Large Verificato da:  \par} 
    {\Large Approvato da: \par}
    \vfill

    {\Large \bfseries Gruppo: NightPRO}    \vspace{0.5cm}

    {\large \href{mailto:swe.nightpro@gmail.com}{swe.nightpro@gmail.com}}\\[2cm]

        {\large Data: 2025-11-05 \par}

     {\Large Versione: 0.1 \par} 

\end{titlepage}

%  SEZIONE: Tabella delle Versioni
% -------------------------------------------------------------------
\newpage
\pagestyle{fancy}
\phantomsection
\addcontentsline{toc}{section}{Tabella delle Versioni}
\section*{Tabella delle Versioni}
\vspace{0.2cm} 
\begin{center}
\resizebox{\textwidth}{!}{
\renewcommand{\arraystretch}{1.2}
\begin{tabular}{@{}llp{0.25\textwidth}p{0.45\textwidth}c@{}} 
\toprule
\textbf{Versione} & \textbf{Data} & \textbf{Autore/i} & \textbf{Descrizione delle Modifiche} & \textbf{Verificatore} \\
\midrule
0.1 & 2025-11-05 & Giovanni Ponso & Creazione bozza glossario &  - \\
\bottomrule
\end{tabular}
}
\end{center}


\newpage
\tableofcontents % Genera l'indice
\pagestyle{fancy}

% -------------------------------------------------------------------
%  INFORMAZIONI GENERALI
% -------------------------------------------------------------------
\newpage
\section{Informazioni Generali}

\subsection{Componenti del Gruppo}

\begin{table}[h!]
\centering
\renewcommand{\arraystretch}{1.2} % più spazio tra le righe
\begin{tabular}{@{}llc@{}}
\toprule
\textbf{Cognome} & \textbf{Nome} & \textbf{Matricola} \\
\midrule
Biasuzzi & Davide & 2111000 \\
Bilato & Leonardo & 2071084 \\
Zanella & Francesco & 2116442 \\
Romascu & Mihaela-Mariana & 2079726 \\
Ogniben & Michele & 2042325 \\
Perozzo & Samuele & 2110989 \\
Ponso & Giovanni & 2000558 \\
\bottomrule
\end{tabular}
\caption{Componenti del gruppo NightPRO.}
\end{table}

% -------------------------------------------------------------------
% INTRODUZIONE
% -------------------------------------------------------------------


\section{Introduzione}

\subsection{Obiettivo del documento}
Il presente glossario raccoglie i termini tecnici, specialistici o potenzialmente ambigui utilizzati nella documentazione del progetto.  
Il suo scopo è garantire una comprensione uniforme del linguaggio adottato dal gruppo, fornendo definizioni chiare e non equivoche.

\subsection{Struttura del documento}
I termini sono organizzati in ordine alfabetico.  
Alla loro prima apparizione nei documenti ufficiali, essi vengono contrassegnati dal pedice \textsubscript{\textsc{g}}, che ne indica la presenza nel glossario.  
Le occorrenze successive non riportano tale marcatura.

\appendix 

% A
\section*{A}
\addcontentsline{toc}{section}{A}

\subsection{API} 
Inserimento di test

% B
\section*{B}
\addcontentsline{toc}{section}{B}

% C
\section*{C}
\addcontentsline{toc}{section}{C}

% D
\section*{D}
\addcontentsline{toc}{section}{D}

% E
\section*{E}
\addcontentsline{toc}{section}{E}

% F
\section*{F}
\addcontentsline{toc}{section}{F}

% G
\section*{G}
\addcontentsline{toc}{section}{G}

% H
\section*{H}
\addcontentsline{toc}{section}{H}

% I
\section*{I}
\addcontentsline{toc}{section}{I}

% J
\section*{J}
\addcontentsline{toc}{section}{J}

% K
\section*{K}
\addcontentsline{toc}{section}{K}

% L
\section*{L}
\addcontentsline{toc}{section}{L}

% M
\section*{M}
\addcontentsline{toc}{section}{M}

% N
\section*{N}
\addcontentsline{toc}{section}{N}

% O
\section*{O}
\addcontentsline{toc}{section}{O}

% P
\section*{P}
\addcontentsline{toc}{section}{P}

% Q
\section*{Q}
\addcontentsline{toc}{section}{Q}

% R
\section*{R}
\addcontentsline{toc}{section}{R}

% S
\section*{S}
\addcontentsline{toc}{section}{S}

% T
\section*{T}
\addcontentsline{toc}{section}{T}

% U
\section*{U}
\addcontentsline{toc}{section}{U}

% V
\section*{V}
\addcontentsline{toc}{section}{V}

% W
\section*{W}
\addcontentsline{toc}{section}{W}

% X
\section*{X}
\addcontentsline{toc}{section}{X}

% Y
\section*{Y}
\addcontentsline{toc}{section}{Y}

% Z
\section*{Z}
\addcontentsline{toc}{section}{Z}


\end{document}