\documentclass[a4paper, 11pt, oneside]{scrartcl} % Classe KOMA-Script

% --- Pacchetti Fondamentali ---
\usepackage[utf8]{inputenc}      % Codifica UTF-8
\usepackage[T1]{fontenc}         % Font encoding moderno
\usepackage[italian]{babel}      % Lingua italiana
\usepackage{lmodern}             % Font "Latin Modern"
\usepackage{scrhack}
\usepackage{placeins}
\usepackage{array}
\usepackage{tabularx}
\usepackage{pgf-pie}             % Pacchetto per grafici a torta

% --- Grafica e Layout ---
\usepackage{graphicx}            % Per le immagini
\graphicspath{{../../assets/},{../assets/},{assets/}}
\usepackage[a4paper, top=2.5cm, bottom=3cm, left=2.5cm, right=2.5cm]{geometry} % Margini
\usepackage{fancyhdr}            % Per header e footer personalizzati
\usepackage{microtype}           % Migliora la tipografia
\usepackage[svgnames, table]{xcolor}
\usepackage{booktabs}            % Per le linee professionali
\usepackage{caption}
\usepackage{hhline}

% --- Utility ---
\usepackage{enumitem}            % Per personalizzare liste
\usepackage{hyperref}            % Rende i link cliccabili

% --- Comandi Personalizzati ---
\newcommand{\riskheader}[1]{%
    \multicolumn{2}{@{}l@{}}{\textbf{#1}} \\
    \midrule
}

\newcolumntype{L}{>{\raggedright\arraybackslash}X}
\newcolumntype{R}{>{\raggedleft\arraybackslash}X}

\hypersetup{
    colorlinks=true,
    linkcolor=DarkBlue,
    filecolor=DarkBlue,      
    urlcolor=DarkBlue,
    citecolor=DarkBlue,
    pdftitle={Piano di Progetto - NightPRO},
    pdfauthor={Gruppo NightPRO},
}

% ===================================================================
%  IMPOSTAZIONE HEADER E FOOTER
% ===================================================================
\pagestyle{fancy}
\fancyhf{} % Pulisce tutti i campi
\fancyhead[L]{NightPRO - Progetto Ingegneria del Software}
\fancyhead[R]{Anno Accademico 2025/2026}
\fancyfoot[C]{\thepage} % Numero di pagina al centro in basso
\renewcommand{\headrulewidth}{0.4pt} % Linea sottile sotto l'header
\renewcommand{\footrulewidth}{0pt}

% ===================================================================
%  INIZIO DEL DOCUMENTO
% ===================================================================
\begin{document}

% -------------------------------------------------------------------
%  SEZIONE: intestazione_titolo
% -------------------------------------------------------------------
\thispagestyle{empty}
\begin{titlepage}
    \centering
    
\begin{figure}
    \centering
    \includegraphics[width=0.4\textwidth]{logo.png}
\end{figure}

    \vfill
    
    {\small UNIVERSITÀ DEGLI STUDI DI PADOVA \par}
    {\small CORSO DI LAUREA IN INFORMATICA (L-31) \par}
    \vspace{0.5cm}
    {\large Corso di Ingegneria del Software \par}
    {\small Anno Accademico 2025/2026 \par}

    \vfill
    
    {\Huge \bfseries Piano di Progetto \par}
    
    \vspace{1cm}
    
    {\Large Redattori: Francesco Zanella, Giovanni Ponso, Michele Ogniben, Mihaela-Mariana Romascu, Samuele Perozzo \par}

    \vfill
    
    {\Large \bfseries Gruppo: NightPRO \par}
    \vspace{0.5cm}
    {\large \href{mailto:swe.nightpro@gmail.com}{swe.nightpro@gmail.com} \par}
    
    \vfill
    
    {\large Data: 2025-12-12 \par}
    {\Large Versione: 0.5 \par} 

\end{titlepage}

% SEZIONE: Tabella delle Versioni
% -------------------------------------------------------------------
\newpage
\pagestyle{fancy}
\phantomsection 
\addcontentsline{toc}{section}{Tabella delle Versioni}
\section*{Tabella delle Versioni}
\vspace{0.2cm}

\begin{center}
\renewcommand{\arraystretch}{1.2}
\begin{tabularx}{\textwidth}{@{}p{0.08\textwidth}p{0.12\textwidth}p{0.20\textwidth}X>{\centering\arraybackslash}p{0.15\textwidth}@{}}
\toprule
\textbf{Versione} & \textbf{Data} & \textbf{Autore/i} & \textbf{Descrizione delle Modifiche} & \textbf{Verificatore} \\
\midrule
0.5 & 2025-12-12 & Samuele Perozzo & Redazione Periodo 5 & Mihaela-Mariana Romascu \\
0.4 & 2025-12-05 & Leonardo Bilato & Redazione Periodo 4 & Giovanni Ponso \\
0.3 & 2025-11-29 & Giovanni Ponso & Redazione Periodo 3 & Leonardo Bilato \\
0.2 & 2025-11-28 & Mihaela-Mariana Romascu, Samuele Perozzo, Michele Ogniben, Francesco Zanella & Redazione Periodo 1 e Periodo 2 & Leonardo Bilato \\
0.1 & 2025-11-24 & Giovanni Ponso, Michele Ogniben, Francesco Zanella & Creazione struttura documento. Redazione di: Introduzione, Analisi dei rischi, Calendario del progetto, Stima dei costi di realizzazione e Modello di Sviluppo & Leonardo Bilato \\
\bottomrule
\end{tabularx}
\end{center}

% -------------------------------------------------------------------
%  SEZIONE: indice
% -------------------------------------------------------------------
\newpage
\tableofcontents 
\pagestyle{fancy} 

% -------------------------------------------------------------------
%  INFORMAZIONI GENERALI
% -------------------------------------------------------------------
\newpage
\section*{Informazioni Generali}
\addcontentsline{toc}{section}{Informazioni Generali}

\subsection*{Componenti del Gruppo}

\begin{table}[h!]
\centering
\renewcommand{\arraystretch}{1.2} 
\begin{tabular}{@{}llc@{}}
\toprule
\textbf{Cognome} & \textbf{Nome} & \textbf{Matricola} \\
\midrule
Biasuzzi & Davide & 2111000 \\
Bilato & Leonardo & 2071084 \\
Zanella & Francesco & 2116442 \\
Romascu & Mihaela-Mariana & 2079726 \\
Ogniben & Michele & 2042325 \\
Perozzo & Samuele & 2110989 \\
Ponso & Giovanni & 2000558 \\
\bottomrule
\end{tabular}
\caption{Componenti del gruppo NightPRO.}
\end{table}

\newpage
\section{Introduzione}

\subsection{Scopo del Documento}
Il presente documento ha lo scopo di definire la pianificazione strategica e operativa del progetto \textit{SmartOrder}, delineando le attività, le scadenze e le risorse necessarie per il raggiungimento degli obiettivi prefissati. 
In particolare, il documento approfondisce:
\begin{itemize}
    \item L'analisi dei rischi e le relative strategie di mitigazione{\scriptsize\raisebox{-0.5ex}{G}};
    \item Il modello di sviluppo adottato{\scriptsize\raisebox{-0.5ex}{G}} e la metodologia di gestione dei processi;
    \item La pianificazione temporale delle attività in relazione alle scadenze (Milestone{\scriptsize\raisebox{-0.5ex}{G}});
    \item La preventivazione e il consuntivo{\scriptsize\raisebox{-0.5ex}{G}} delle risorse economiche e orarie per ciascun periodo di lavoro.
\end{itemize}
Questo piano è da intendersi come uno strumento dinamico, soggetto ad aggiornamenti periodici basati sull'avanzamento reale dei lavori e sui feedback ricevuti.

\subsection{Scopo del Prodotto}
Il Capitolato{\scriptsize\raisebox{-0.5ex}{G}} C8 – \textbf{SmartOrder}, proposto da Ergon Informatica, ha come obiettivo la realizzazione di una piattaforma intelligente per l’automazione del processo di gestione degli ordini cliente.
Il sistema è progettato per ricevere input multimodali (testo, audio, immagini) ed elaborarli automaticamente tramite tecniche di Intelligenza Artificiale{\scriptsize\raisebox{-0.5ex}{G}}, Machine Learning{\scriptsize\raisebox{-0.5ex}{G}} e NLP{\scriptsize\raisebox{-0.5ex}{G}}.
La finalità è convertire richieste non strutturate in ordini strutturati pronti per l'inserimento nei sistemi ERP{\scriptsize\raisebox{-0.5ex}{G}} aziendali, riducendo l'intervento umano e aumentando l'efficienza operativa.

\subsection{Glossario}
Al fine di evitare ambiguità, i termini tecnici o specifici del dominio sono raccolti nel documento \textit{Glossario}. Nel presente testo, tali termini sono contrassegnati con una “G” in pedice alla loro prima occorrenza (es. Agile{\scriptsize\raisebox{-0.5ex}{G}}).

\subsection{Riferimenti}
\subsubsection{Riferimenti normativi}
\begin{itemize}
    \item \href{https://github.com/swenightpro/Documentazione/blob/main/docs/RTB/Documentazione%20Interna/}{Norme di Progetto}
    \item \href{https://www.math.unipd.it/~tullio/IS-1/2025/Progetto/C8.pdf}{Capitolato C8: SmartOrder}
    \item \href{https://www.math.unipd.it/~tullio/IS-1/2025/Dispense/PD1.pdf}{Regolamento progetto didattico}
\end{itemize}

\subsubsection{Riferimenti informativi}
\begin{itemize}
    \item \href{https://github.com/swenightpro/Documentazione/blob/main/docs/RTB/Documentazione%20Esterna}{Glossario}
    \item \href{https://www.math.unipd.it/~tullio/IS-1/2025/Dispense/T02.pdf}{T2: Ciclo di vita del software}
    \item \href{https://www.math.unipd.it/~tullio/IS-1/2025/Dispense/T04.pdf}{T4: Gestione di Progetto}
\end{itemize}

\newpage
\section{Analisi dei rischi}
\label{sec:analisi_rischi}
La gestione dei rischi rappresenta un elemento fondamentale nella conduzione di progetti software, in quanto consente di anticipare e mitigare potenziali problematiche che potrebbero compromettere il successo dell'iniziativa. L'identificazione precoce dei rischi e la definizione di strategie di mitigazione appropriate permettono di ridurre l'impatto negativo di eventi imprevisti, garantendo il rispetto delle scadenze, il controllo dei costi e il mantenimento degli standard qualitativi.

Il presente documento illustra la metodologia di gestione dei rischi adottata dal gruppo per il progetto SmartOrder. Vengono presentati i principali rischi identificati, classificati per categoria, insieme alle relative strategie di prevenzione e mitigazione. L'obiettivo è fornire un quadro completo e strutturato che supporti il team nel prendere decisioni informate e nel rispondere efficacemente alle sfide che potrebbero emergere durante lo sviluppo del progetto.

\subsection{Metodologia di gestione}

Il gruppo adotta un approccio strutturato alla gestione dei rischi, articolato in cinque fasi che vengono applicate in modo continuativo durante tutto il ciclo di vita del progetto:

\begin{description}
    \item[Identificazione] Individuazione sistematica delle fonti di rischio attraverso l'analisi delle attività pianificate, degli strumenti e delle tecnologie adottate, nonché delle dinamiche organizzative del gruppo. Questa fase mira a creare un inventario completo dei potenziali problemi che potrebbero influenzare il progetto.
    
    \item[Analisi] Valutazione approfondita di ciascun rischio per determinarne la probabilità di occorrenza e il grado di pericolosità. L'analisi fornisce le basi per decisioni informate sulle strategie di trattamento più appropriate, permettendo di allocare le risorse in modo efficace.
    
    \item[Valutazione] Determinazione delle priorità e definizione dell'ordine di attuazione delle misure di mitigazione. Questa fase consente di concentrare gli sforzi sulle minacce più rilevanti, ottimizzando l'utilizzo delle risorse disponibili.
    
    \item[Gestione] Implementazione di misure preventive e azioni di mitigazione mirate. La fase di gestione traduce le analisi precedenti in interventi concreti, che possono includere misure preventive, trasferimento del rischio o adozione di piani di contingenza.
    
    \item[Monitoraggio e revisione] Controllo periodico dell'efficacia delle soluzioni adottate e identificazione di nuovi rischi emergenti. Il monitoraggio continuo garantisce che la gestione dei rischi resti allineata agli obiettivi del progetto e si adatti alle condizioni mutevoli del contesto di sviluppo.
\end{description}

\subsection{Sistema di classificazione}

Per garantire una tracciabilità efficace e una gestione organizzata dei rischi, è stato definito un sistema di codifica che consente di identificare rapidamente la natura e la categoria di ciascun rischio. La convenzione adottata segue il formato:

\begin{center}
\textbf{R[Tipo][Indice]}
\end{center}

dove:
\begin{itemize}
    \item \textbf{R} indica che si tratta di un rischio;
    \item \textbf{Tipo} identifica la categoria di appartenenza:
    \begin{itemize}
        \item \textbf{T}: rischi tecnologici, relativi all'utilizzo di strumenti, tecnologie o infrastrutture;
        \item \textbf{O}: rischi organizzativi, concernenti la gestione del progetto, la comunicazione e il coordinamento del gruppo;
        \item \textbf{P}: rischi personali, legati alla disponibilità, alle competenze o agli impegni dei singoli membri del team.
    \end{itemize}
    \item \textbf{Indice} è un numero progressivo che identifica univocamente il rischio all'interno della categoria.
\end{itemize}

Questa classificazione facilita la consultazione rapida del documento e permette di tracciare l'evoluzione dei rischi nel tempo, supportando il processo di monitoraggio e revisione.

\subsection{Rischi tecnologici}

\subsubsection{RT1: Complessità nell'apprendimento di nuove tecnologie}
\label{sec:rt1}

Il capitolato C8 (SmartOrder) richiede l'utilizzo di tecnologie avanzate per l'elaborazione del linguaggio naturale, l'estrazione di entità e la classificazione tramite modelli di intelligenza artificiale. Una parte del gruppo potrebbe non possedere esperienza pregressa sufficiente con queste tecnologie, rendendo necessario un periodo di apprendimento che potrebbe impattare sui tempi di sviluppo.

\begin{table}[!ht]
\centering
\renewcommand{\arraystretch}{1.4}
\begin{tabularx}{\textwidth}{@{}>{\bfseries}lL@{}}
\toprule
\riskheader{RT1: Complessità nell'apprendimento di nuove tecnologie}
\textbf{Codice} & RT1 \\
\addlinespace[0.3em]
\textbf{Probabilità} & Alta \\
\addlinespace[0.3em]
\textbf{Pericolosità} & Alta \\
\addlinespace[0.3em]
\textbf{Conseguenze} & Ritardi nella pianificazione dovuti al tempo necessario per acquisire familiarità con le tecnologie da parte dei membri meno esperti. Possibile riduzione della qualità iniziale delle implementazioni. Incremento del carico di lavoro per i membri del gruppo impegnati nell'apprendimento. Necessità di dedicare tempo aggiuntivo alla formazione, sottraendo risorse alle attività di sviluppo. \\
\addlinespace[0.3em]
\textbf{Mitigazione} & Organizzare sessioni di formazione interne e workshop dedicati all'esplorazione delle tecnologie. Sfruttare le risorse formative fornite dal proponente{\scriptsize\raisebox{-0.5ex}{G}} e creare documentazione interna condivisa per raccogliere soluzioni e best practices. Predisporre una fase iniziale di prototipazione per testare le tecnologie in un contesto controllato prima dell'integrazione nel prodotto finale. Limitare l'MVP{\scriptsize\raisebox{-0.5ex}{G}} iniziale alla sola modalità testuale, come confermato dall'incontro con il proponente, per gestire la complessità in modo progressivo. \\
\bottomrule
\end{tabularx}
\caption{RT1: Complessità nell'apprendimento di nuove tecnologie}
\end{table}

\subsubsection{RT2: Mancanza di documentazione o risorse per le tecnologie adottate}

Alcune tecnologie emergenti potrebbero presentare documentazione incompleta o risorse di apprendimento limitate, rendendo più complesso il processo di comprensione e implementazione.

\begin{table}[!ht]
\centering
\renewcommand{\arraystretch}{1.4}
\begin{tabularx}{\textwidth}{@{}>{\bfseries}lL@{}}
\toprule
\riskheader{RT2: Mancanza di documentazione o risorse per le tecnologie adottate}
\textbf{Codice} & RT2 \\
\addlinespace[0.3em]
\textbf{Probabilità} & Media \\
\addlinespace[0.3em]
\textbf{Pericolosità} & Alta \\
\addlinespace[0.3em]
\textbf{Conseguenze} & Prolungamento dei tempi di apprendimento e sviluppo. Possibile necessità di ricerche approfondite o sperimentazioni autonome per risolvere problematiche specifiche. Aumento del rischio di errori di implementazione. \\
\addlinespace[0.3em]
\textbf{Mitigazione} & Coinvolgere il proponente per ottenere supporto diretto, risorse aggiuntive o contatti con esperti nel dominio. Valutare l'adozione di alternative tecnologiche con documentazione più completa qualora la tecnologia non risulti strettamente indispensabile. Mantenere una comunicazione costante con il proponente per segnalare tempestivamente eventuali difficoltà. \\
\bottomrule
\end{tabularx}
\caption{RT2: Mancanza di documentazione o risorse per le tecnologie adottate}
\end{table}

\subsubsection{RT3: Problemi con strumenti di terze parti}

Il progetto si avvale di strumenti software di terze parti, librerie e servizi esterni. Malfunzionamenti, bug{\scriptsize\raisebox{-0.5ex}{G}} o cambiamenti nelle API{\scriptsize\raisebox{-0.5ex}{G}} di questi strumenti potrebbero generare ritardi o compromettere la funzionalità del sistema.

\begin{table}[!ht]
\centering
\renewcommand{\arraystretch}{1.4}
\begin{tabularx}{\textwidth}{@{}>{\bfseries}lL@{}}
\toprule
\riskheader{RT3: Problemi con strumenti di terze parti}
\textbf{Codice} & RT3 \\
\addlinespace[0.3em]
\textbf{Probabilità} & Media \\
\addlinespace[0.3em]
\textbf{Pericolosità} & Media \\
\addlinespace[0.3em]
\textbf{Conseguenze} & Ritardi nello sviluppo dovuti alla necessità di risolvere problemi esterni o trovare alternative. Possibile impatto negativo sulla qualità e sulle funzionalità del prodotto finale. Costi aggiuntivi per la risoluzione o la sostituzione degli strumenti problematici. \\
\addlinespace[0.3em]
\textbf{Mitigazione} & Monitorare attentamente il funzionamento degli strumenti durante lo sviluppo e consultare i bug tracking system{\scriptsize\raisebox{-0.5ex}{G}} relativi. Mantenere gli strumenti aggiornati con le ultime patch{\scriptsize\raisebox{-0.5ex}{G}} di sicurezza. Valutare alternative robuste e predisporre un piano di ripristino rapido che includa la possibilità di tornare a versioni precedenti se necessario. \\
\bottomrule
\end{tabularx}
\caption{RT3: Problemi con strumenti di terze parti}
\end{table}

\subsubsection{RT4: Perdita o danneggiamento di file}

Esiste il rischio che file importanti vengano persi o danneggiati a causa di malfunzionamenti hardware, errori umani o problemi con il sistema di versionamento{\scriptsize\raisebox{-0.5ex}{G}}.

\begin{table}[!ht]
\centering
\renewcommand{\arraystretch}{1.4}
\begin{tabularx}{\textwidth}{@{}>{\bfseries}lL@{}}
\toprule
\riskheader{RT4: Perdita o danneggiamento di file}
\textbf{Codice} & RT4 \\
\addlinespace[0.3em]
\textbf{Probabilità} & Bassa \\
\addlinespace[0.3em]
\textbf{Pericolosità} & Media \\
\addlinespace[0.3em]
\textbf{Conseguenze} & Perdita di lavoro svolto, necessità di ricreare documenti o codice. Ritardi nello sviluppo e possibile compromissione della continuità del progetto. \\
\addlinespace[0.3em]
\textbf{Mitigazione} & Utilizzare un sistema di versionamento robusto che consenta di tracciare e recuperare agevolmente versioni precedenti dei file. Eseguire commit{\scriptsize\raisebox{-0.5ex}{G}} frequenti e backup regolari. Mantenere una struttura organizzata del repository per facilitare il recupero delle informazioni. \\
\bottomrule
\end{tabularx}
\caption{RT4: Perdita o danneggiamento di file}
\end{table}

\subsection{Rischi organizzativi}

\subsubsection{RO1: Comunicazione interna inefficace}

La comunicazione tra i membri del gruppo potrebbe non essere sufficientemente chiara, tempestiva o completa, generando disallineamenti e malintesi che compromettono la qualità del lavoro.

\begin{table}[!ht]
\centering
\renewcommand{\arraystretch}{1.4}
\begin{tabularx}{\textwidth}{@{}>{\bfseries}lL@{}}
\toprule
\riskheader{RO1: Comunicazione interna inefficace}
\textbf{Codice} & RO1 \\
\addlinespace[0.3em]
\textbf{Probabilità} & Media \\
\addlinespace[0.3em]
\textbf{Pericolosità} & Alta \\
\addlinespace[0.3em]
\textbf{Conseguenze} & I membri del team potrebbero non essere consapevoli di problemi emergenti o decisioni prese. Mancanza di coordinamento che può portare a soluzioni non allineate o a duplicazione di lavoro. Riduzione dell'efficienza complessiva e possibile compromissione della qualità del prodotto finale. Ritardi nella risoluzione di problematiche che potrebbero essere affrontate tempestivamente con una comunicazione efficace. \\
\addlinespace[0.3em]
\textbf{Mitigazione} & Definire canali di comunicazione chiari e strutturati, utilizzando strumenti formali (GitHub Projects per la gestione delle attività, Telegram{\scriptsize\raisebox{-0.5ex}{G}} con topic per organizzare le conversazioni per argomento) e informali. Organizzare riunioni periodiche per fare il punto sui progressi e risolvere eventuali problematiche. Mantenere una documentazione condivisa e centralizzata facilmente consultabile da tutti i membri, dove registrare aggiornamenti, decisioni e soluzioni ai problemi. Assicurare che tutti siano allineati sugli obiettivi, le priorità e le scadenze del progetto. \\
\bottomrule
\end{tabularx}
\caption{RO1: Comunicazione interna inefficace}
\end{table}

\subsubsection{RO2: Confusione sulle responsabilità e sui ruoli}

La mancanza di chiarezza sui compiti assegnati e sui ruoli di ciascun membro potrebbe generare sovrapposizioni, lacune o mancanza di coordinamento.

\begin{table}[!ht]
\centering
\renewcommand{\arraystretch}{1.4}
\begin{tabularx}{\textwidth}{@{}>{\bfseries}lL@{}}
\toprule
\riskheader{RO2: Confusione sulle responsabilità e sui ruoli}
\textbf{Codice} & RO2 \\
\addlinespace[0.3em]
\textbf{Probabilità} & Media \\
\addlinespace[0.3em]
\textbf{Pericolosità} & Alta \\
\addlinespace[0.3em]
\textbf{Conseguenze} & Attività non completate o duplicate. Conflitti interni e inefficienze nella gestione delle risorse. Ritardi nel progetto dovuti alla mancanza di coordinamento. \\
\addlinespace[0.3em]
\textbf{Mitigazione} & Definire in modo chiaro le responsabilità di ciascun membro fin dall'inizio, utilizzando strumenti di gestione del progetto (GitHub Projects) per tracciare i compiti assegnati e monitorare i progressi. Documentare la politica di rotazione dei ruoli e assicurarsi che tutti i membri ne siano consapevoli. Il Responsabile di Progetto deve verificare periodicamente che le assegnazioni siano chiare e che non vi siano sovrapposizioni. \\
\bottomrule
\end{tabularx}
\caption{RO2: Confusione sulle responsabilità e sui ruoli}
\end{table}

\subsubsection{RO3: Ritardi nella pianificazione e nel rispetto delle scadenze}

Una pianificazione imprecisa o irrealistica potrebbe portare a sottovalutare il tempo necessario per completare determinate attività, causando slittamenti nelle scadenze.

\begin{table}[!ht]
\centering
\renewcommand{\arraystretch}{1.4}
\begin{tabularx}{\textwidth}{@{}>{\bfseries}lL@{}}
\toprule
\riskheader{RO3: Ritardi nella pianificazione e nel rispetto delle scadenze}
\textbf{Codice} & RO3 \\
\addlinespace[0.3em]
\textbf{Probabilità} & Alta \\
\addlinespace[0.3em]
\textbf{Pericolosità} & Alta \\
\addlinespace[0.3em]
\textbf{Conseguenze} & Sforamento dei tempi preventivati per la realizzazione dei compiti. Possibile compromissione della qualità del lavoro a causa della fretta per recuperare i ritardi. Allocazione non ottimale delle risorse con conseguente impatto negativo sul rendimento complessivo. \\
\addlinespace[0.3em]
\textbf{Mitigazione} & Comprendere chiaramente le priorità del progetto per evitare di sprecare tempo su attività secondarie. Effettuare una pianificazione accurata che tenga conto dell'inesperienza iniziale del gruppo e includa margini di sicurezza. Monitorare continuamente i progressi attraverso GitHub Projects e riunioni periodiche, identificando tempestivamente eventuali slittamenti. Il Responsabile di Progetto deve essere informato immediatamente di eventuali difficoltà nel rispettare le scadenze, così da poter riassegnare le attività o estendere i tempi previsti. \\
\bottomrule
\end{tabularx}
\caption{RO3: Ritardi nella pianificazione e nel rispetto delle scadenze}
\end{table}

\subsubsection{RO4: Problemi nella gestione del versionamento}

La gestione delle versioni dei documenti e del codice potrebbe risultare complessa o generare confusione, specialmente durante le fasi di sviluppo parallelo.

\begin{table}[!ht]
\centering
\renewcommand{\arraystretch}{1.4}
\begin{tabularx}{\textwidth}{@{}>{\bfseries}lL@{}}
\toprule
\riskheader{RO4: Problemi nella gestione del versionamento}
\textbf{Codice} & RO4 \\
\addlinespace[0.3em]
\textbf{Probabilità} & Media \\
\addlinespace[0.3em]
\textbf{Pericolosità} & Media \\
\addlinespace[0.3em]
\textbf{Conseguenze} & Confusione sulla versione corrente dei documenti. Possibile perdita di modifiche o conflitti durante il merge{\scriptsize\raisebox{-0.5ex}{G}}. Difficoltà nel tracciare l'evoluzione del progetto e nel recuperare versioni precedenti. \\
\addlinespace[0.3em]
\textbf{Mitigazione} & Stabilire convenzioni chiare per il versionamento dei documenti e del codice, documentate nelle Norme di Progetto. Utilizzare branch{\scriptsize\raisebox{-0.5ex}{G}} separati per le modifiche in corso e assicurarsi che ogni commit sia verificato prima del merge. Implementare workflow{\scriptsize\raisebox{-0.5ex}{G}} automatizzati tramite GitHub Actions{\scriptsize\raisebox{-0.5ex}{G}} per garantire coerenza. Consultare il docente per le best practices consigliate quando necessario. \\
\bottomrule
\end{tabularx}
\caption{RO4: Problemi nella gestione del versionamento}
\end{table}

\subsubsection{RO5: Comunicazione inefficace con il proponente}

La comunicazione con l'azienda proponente potrebbe non essere sempre efficace o tempestiva, generando dubbi e ritardi nella risoluzione di questioni tecniche o organizzative.

\begin{table}[!ht]
\centering
\renewcommand{\arraystretch}{1.4}
\begin{tabularx}{\textwidth}{@{}>{\bfseries}lL@{}}
\toprule
\riskheader{RO5: Comunicazione inefficace con il proponente}
\textbf{Codice} & RO5 \\
\addlinespace[0.3em]
\textbf{Probabilità} & Media \\
\addlinespace[0.3em]
\textbf{Pericolosità} & Media \\
\addlinespace[0.3em]
\textbf{Conseguenze} & Risposte assenti o incomplete che non contribuiscono alla risoluzione di dubbi o domande. Diminuzione della frequenza degli incontri. Possibile disallineamento tra le aspettative del proponente e il lavoro svolto dal gruppo. \\
\addlinespace[0.3em]
\textbf{Mitigazione} & Il Responsabile di Progetto deve comunicare tempestivamente la situazione al proponente, cercando di trovare una soluzione condivisa. Se non si riesce a risolvere il problema direttamente, richiedere l'intervento del committente{\scriptsize\raisebox{-0.5ex}{G}}. Stabilire fin dall'inizio modalità di comunicazione chiare (sincrona e asincrona) e frequenza degli incontri. Documentare tutte le comunicazioni importanti nei verbali esterni. \\
\bottomrule
\end{tabularx}
\caption{RO5: Comunicazione inefficace con il proponente}
\end{table}

\subsection{Rischi personali}

\subsubsection{RP1: Mancata continuità del progetto}

Interruzioni nel flusso di lavoro causate da assenze impreviste, malattie, impegni accademici o personali dei membri del gruppo potrebbero compromettere la continuità e la produttività del progetto.

\begin{table}[!ht]
\centering
\renewcommand{\arraystretch}{1.4}
\begin{tabularx}{\textwidth}{@{}>{\bfseries}lL@{}}
\toprule
\riskheader{RP1: Mancata continuità del progetto}
\textbf{Codice} & RP1 \\
\addlinespace[0.3em]
\textbf{Probabilità} & Alta \\
\addlinespace[0.3em]
\textbf{Pericolosità} & Alta \\
\addlinespace[0.3em]
\textbf{Conseguenze} & Riduzione della disponibilità di tempo per lavorare sul progetto. Procrastinazione e mancanza di pianificazione che possono portare a inefficienze nell'utilizzo del tempo. Eventi imprevisti come malattie o emergenze possono causare assenze improvvise, rallentando ulteriormente il lavoro e influendo sulla capacità del team di rispettare i tempi stabiliti. \\
\addlinespace[0.3em]
\textbf{Mitigazione} & Pianificare con anticipo e creare un programma di lavoro realistico che tenga conto degli impegni accademici e personali. Utilizzare strumenti di gestione del tempo come calendari condivisi e pianificazioni settimanali per monitorare i progressi e identificare tempestivamente eventuali slittamenti. Prevedere margini di tempo extra per imprevisti. I membri del gruppo devono comunicare tempestivamente al Responsabile eventuali assenze o indisponibilità, specificando la ragione e fornendo una stima del periodo. Il Responsabile redistribuirà le attività dei membri assenti agli altri membri del team, o le ripianificherà per il periodo successivo se la redistribuzione rischia di sovraccaricare il team. \\
\bottomrule
\end{tabularx}
\caption{RP1: Mancata continuità del progetto}
\end{table}

\subsubsection{RP2: Inesperienza nell'esecuzione di attività specifiche}

Il team potrebbe trovarsi ad affrontare compiti o attività che richiedono competenze specifiche o esperienza pregressa di cui potrebbe non essere in possesso, rallentando il completamento delle attività.

\begin{table}[!ht]
\centering
\renewcommand{\arraystretch}{1.4}
\begin{tabularx}{\textwidth}{@{}>{\bfseries}lL@{}}
\toprule
\riskheader{RP2: Inesperienza nell'esecuzione di attività specifiche}
\textbf{Codice} & RP2 \\
\addlinespace[0.3em]
\textbf{Probabilità} & Alta \\
\addlinespace[0.3em]
\textbf{Pericolosità} & Media \\
\addlinespace[0.3em]
\textbf{Conseguenze} & Rallentamento nel completamento delle attività dovuto alla necessità di acquisire competenze mancanti. Possibili errori iniziali che richiedono correzioni successive. Incremento del tempo necessario per raggiungere gli obiettivi prefissati. \\
\addlinespace[0.3em]
\textbf{Mitigazione} & I membri del team devono notificare tempestivamente al Responsabile eventuali difficoltà riscontrate durante l'esecuzione di un'attività, con particolare attenzione alle attività in cui manca esperienza. Identificare rapidamente le lacune di conoscenza e fornire formazione o risorse aggiuntive al team. Coinvolgere il proponente e il committente per ottenere consulenza quando possibile. In caso di ritardi significativi, rivedere il piano di progetto e riallocare risorse. La rotazione dei ruoli prevista nelle Norme di Progetto favorisce l'acquisizione di competenze trasversali da parte di tutti i membri. \\
\bottomrule
\end{tabularx}
\caption{RP2: Inesperienza nell'esecuzione di attività specifiche}
\end{table}

\subsubsection{RP3: Non conformità rispetto agli impegni dichiarati}

Se i membri del team non adempiono agli impegni presi o non rispettano le scadenze concordate, il progetto potrebbe subire ritardi o compromettere la qualità finale.

\begin{table}[!ht]
\centering
\renewcommand{\arraystretch}{1.4}
\begin{tabularx}{\textwidth}{@{}>{\bfseries}lL@{}}
\toprule
\riskheader{RP3: Non conformità rispetto agli impegni dichiarati}
\textbf{Codice} & RP3 \\
\addlinespace[0.3em]
\textbf{Probabilità} & Media \\
\addlinespace[0.3em]
\textbf{Pericolosità} & Media \\
\addlinespace[0.3em]
\textbf{Conseguenze} & Perdita di fiducia da parte del proponente e del committente. Costi e tempi aggiuntivi per rimediare ai problemi. Diminuzione della produttività e potenziale perdita di coesione all'interno del gruppo. \\
\addlinespace[0.3em]
\textbf{Mitigazione} & Adottare una gestione rigorosa del progetto con comunicazione continua e monitoraggio costante dei progressi per garantire il rispetto degli impegni dichiarati. Utilizzare GitHub Projects per tracciare le attività e identificare tempestivamente eventuali ritardi. Il Responsabile di Progetto deve verificare periodicamente lo stato di avanzamento e intervenire prontamente in caso di problemi. Promuovere un ambiente di lavoro collaborativo dove tutti i membri si sentono responsabili del successo del progetto. \\
\bottomrule
\end{tabularx}
\caption{RP3: Non conformità rispetto agli impegni dichiarati}
\end{table}

\subsection{Monitoraggio e aggiornamento}

Il processo di gestione dei rischi non è statico ma richiede un monitoraggio continuo e aggiornamenti periodici. Durante le riunioni interne del gruppo, vengono periodicamente rivisti i rischi identificati, valutata l'efficacia delle strategie di mitigazione adottate e identificati eventuali nuovi rischi emersi durante lo sviluppo del progetto.

Le modifiche ai rischi o alle strategie di mitigazione vengono documentate e aggiornate nel presente documento, garantendo che la gestione dei rischi resti allineata all'evoluzione del progetto e alle condizioni mutevoli del contesto di sviluppo.

\newpage

\newpage
\section{Calendario del progetto}

\subsection{Introduzione}
Il calendario illustra le date previste per le revisioni di avanzamento, definite in base alla pianificazione strategica e alle scadenze del corso. Tali date rappresentano delle \textit{milestone} fondamentali per la verifica del lavoro svolto.

\subsection{Calendario Revisioni Avanzamento Progetto - Prima stesura 30/10/2025}

\begin{table}[!ht]
\centering
\renewcommand{\arraystretch}{1.5}
\begin{tabularx}{0.8\textwidth}{L c}
\toprule
\textbf{Revisione} & \textbf{Data stimata} \\
\midrule
Requirements and Technology Baseline{\scriptsize\raisebox{-0.5ex}{G}} (RTB) & ? \\
\midrule
Product Baseline (PB) & 21/03/2026 \\
\bottomrule
\end{tabularx}
\caption{Calendario delle revisioni - Prima stesura}
\end{table}

\newpage
\section{Stima dei costi di realizzazione}

\subsection{Introduzione}
La stima dei costi di realizzazione definisce il budget preventivato per il completamento del progetto. Tale calcolo si basa sulla distribuzione dei ruoli, sulle ore allocate e sui costi orari stabiliti dal regolamento didattico.
Poiché il preventivo{\scriptsize\raisebox{-0.5ex}{G}} potrà subire aggiornamenti nelle stesure successive (esclusivamente al ribasso o a parità di saldo), questo valore rappresenta un vincolo economico rigido: qualora le proiezioni future indicassero un potenziale superamento di tale soglia, il gruppo interverrà tempestivamente concordando con il proponente una rimodulazione del perimetro dei requisiti per garantire il rispetto dei vincoli di budget.

\subsection{Preventivo Costi - Prima stesura 30/10/2025}
\begin{table}[!ht]
\centering
\renewcommand{\arraystretch}{1.4}
\begin{tabularx}{\textwidth}{L c c c}
\toprule
\textbf{Ruolo} & \textbf{Costo orario (€/h)} & \textbf{Ore previste per ruolo (h)} & \textbf{Costo per ruolo (€)} \\
\midrule
Responsabile & 30 & 63 & 1.890,00 \\
Amministratore & 20 & 63 & 1.260,00 \\
Analista & 25 & 126 & 3.150,00 \\
Progettista & 25 & 126 & 3.150,00 \\
Programmatore & 15 & 157 & 2.355,00 \\
Verificatore & 15 & 95 & 1.425,00 \\
\midrule
\textbf{TOTALE} & - & \textbf{630} & \textbf{13.230,00} \\
\bottomrule
\end{tabularx}
\caption{Distribuzione dei costi e delle ore per ruolo - Prima stesura}
\end{table}

\begin{figure}[!ht]
\centering
\begin{tikzpicture}
\pie[
    sum=12850,
    text=pin,
    radius=2.5,
    color={blue!40, red!40, orange!40, yellow!50, cyan!50, purple!40}
]{
  1890/Responsabile,
  1260/Amministratore,
  3150/Analista,
  3150/Progettista,
  2355/Programmatore,
  1425/Verificatore
}
\end{tikzpicture}
\caption{Grafico a torta della distribuzione dei costi per ruolo - Prima stesura}
\end{figure}

\newpage
\subsection{Preventivo Costi - Seconda stesura 04/11/2025}
\label{sec:preventivo_v2}
A seguito di una revisione delle necessità operative, il gruppo ha ritenuto opportuno effettuare una rimodulazione delle risorse assegnate ai diversi ruoli.
L'intervento mira a garantire un bilanciamento più efficace tra le attività di analisi, sviluppo e verifica, allineando l'impegno previsto alle specifiche esigenze tecniche emerse in fase di studio preliminare, mantenendo inalterato il monte ore complessivo.
\begin{table}[!ht]
\centering
\renewcommand{\arraystretch}{1.4}
\begin{tabularx}{\textwidth}{L c c c}
\toprule
\textbf{Ruolo} & \textbf{Costo orario (€/h)} & \textbf{Ore previste per ruolo (h)} & \textbf{Costo per ruolo (€)} \\
\midrule
Responsabile & 30 & 63 & 1.890,00 \\
Amministratore & 20 & 63 & 1.260,00 \\
Analista & 25 & 107 & 2.675,00 \\
Progettista & 25 & 107 & 2.675,00 \\
Programmatore & 15 & 157 & 2.355,00 \\
Verificatore & 15 & 133 & 1.995,00 \\
\midrule
\textbf{TOTALE} & - & \textbf{630} & \textbf{12.850,00} \\
\bottomrule
\end{tabularx}
\caption{Distribuzione dei costi e delle ore per ruolo - Seconda stesura}
\end{table}

\begin{figure}[!ht]
\centering
\begin{tikzpicture}
\pie[
    sum=12850,
    text=pin,
    radius=2.5,
    color={blue!40, red!40, orange!40, yellow!50, cyan!50, purple!40}
]{
  1890/Responsabile,
  1260/Amministratore,
  2675/Analista,
  2675/Progettista,
  2355/Programmatore,
  1995/Verificatore
}
\end{tikzpicture}
\caption{Grafico a torta della distribuzione dei costi per ruolo - Seconda stesura}
\end{figure}

\newpage
\section{Modello di Sviluppo}

Per la gestione del progetto \textit{SmartOrder}, il gruppo NightPRO ha scelto di adottare un modello di sviluppo \textbf{Agile}, facendo specifico riferimento al framework{\scriptsize\raisebox{-0.5ex}{G}} \textbf{Scrum{\scriptsize\raisebox{-0.5ex}{G}}}.
Tale scelta è motivata dalla necessità di gestire un progetto innovativo con requisiti che potrebbero evolvere in base ai feedback dell'azienda proponente, garantendo al contempo un controllo costante sull'avanzamento e sulla qualità del prodotto.

\subsection{Fasi di progetto}
Il ciclo di vita del progetto si articola in tre macro-fasi principali, ciascuna caratterizzata da specifici obiettivi:
\begin{itemize}
    \item \textbf{Candidatura:} È la fase preliminare, antecedente l'avvio operativo dello sviluppo, dedicata all'analisi comparata dei capitolati proposti e alla stima iniziale dei costi e delle risorse necessarie. In questo stadio il gruppo definisce l'assetto organizzativo e seleziona il progetto da realizzare.

    \item \textbf{RTB (Requirements and Technology Baseline):} Questa fase si concentra sull'acquisizione del dominio e sul consolidamento delle basi tecnologiche. Sebbene il gruppo utilizzi il concetto di \textbf{Sprint}{\scriptsize\raisebox{-0.5ex}{G}} per scandire il tempo e gli obiettivi, in questa fase non si applica un ciclo Scrum puro poiché il \textbf{Product Backlog}{\scriptsize\raisebox{-0.5ex}{G}} è ancora in fase di definizione e consolidamento. L'attività principale è lo studio dei requisiti dell'azienda proponente e l'analisi delle tecnologie, che culmina nella realizzazione di un \textbf{Proof of Concept (PoC){\scriptsize\raisebox{-0.5ex}{G}}}.

    \item \textbf{PB (Product Baseline):} È la fase di sviluppo intensivo, volta alla realizzazione del prodotto software vero e proprio. Il lavoro si focalizza sulla costruzione del sistema fino al raggiungimento di un \textbf{Minimum Viable Product (MVP)}, che implementa le funzionalità core concordate con il proponente.
\end{itemize}

\subsection{Adozione del framework Scrum}
Il gruppo ha adattato i cerimoniali Scrum al contesto accademico per massimizzare l'efficienza operativa. Il ciclo di lavoro si ripete ad ogni Sprint (Periodo) secondo il seguente schema:
\begin{itemize}
    \item \textbf{Sprint Planning}{\scriptsize\raisebox{-0.5ex}{G}}: All'inizio di ogni periodo, il gruppo seleziona le attività da svolgere basandosi sulle priorità emerse, definendo lo \textit{Sprint Backlog}{\scriptsize\raisebox{-0.5ex}{G}} corrente.
    \item \textbf{Stand-up Meeting Asincrono:} Per garantire un allineamento costante compatibile con gli impegni accademici, il coordinamento avviene tramite un canale di messaggistica dedicato. Ogni membro riporta le attività svolte e gli eventuali impedimenti con una cadenza indicativa di circa 48 ore (o secondo necessità).
    \item \textbf{Sprint Review{\scriptsize\raisebox{-0.5ex}{G}} e Retrospective{\scriptsize\raisebox{-0.5ex}{G}}:} Al termine dello sprint, si svolge un incontro di chiusura unificato per verificare il lavoro svolto rispetto agli obiettivi (Review) e analizzare il processo interno per definire miglioramenti futuri (Retrospective).
\end{itemize}

\subsection{Gestione dei Ruoli e Rotazione}
Per garantire che ogni componente del gruppo acquisisca competenze trasversali e una visione completa del progetto, è stata definita una politica di rotazione dei ruoli basata sulla durata degli sprint:
\begin{itemize}
    \item \textbf{Fase Iniziale (Prime 7 settimane):} Gli sprint hanno una durata ridotta di \textbf{una settimana}. La rotazione dei ruoli avviene ad ogni sprint, permettendo a tutti i membri di sperimentare velocemente le diverse responsabilità e consolidare il metodo di lavoro.
    \item \textbf{Fase a Regime (Dalla settimana 8):} Una volta stabilizzati i processi, la durata degli sprint e della rotazione si estende a \textbf{due settimane}. Questa cadenza è ideale per le fasi di sviluppo più intenso, dove è necessaria maggiore continuità operativa per completare task complessi di codifica e progettazione.
\end{itemize}

\subsection{Gestione e Monitoraggio delle Attività}
Il monitoraggio delle attività è supportato dall'utilizzo di \textbf{GitHub Projects}, che funge da strumento per la gestione visiva dei task (Kanban Board), garantendo trasparenza sullo stato di avanzamento dei lavori verso tutti gli stakeholder interni ed esterni.

\subsection{Struttura dei rendiconti di periodo}
Per ogni periodo rendicontato in questo documento, verranno presentate sistematicamente le seguenti sezioni per monitorare l'avanzamento:

\begin{description}
    \item[Pianificazione Operativa ed Economica (Sprint Planning)] \hfill \\
    In questa sezione vengono definiti gli obiettivi strategici e le stime per il periodo. Include:
    \begin{itemize}
        \item \textbf{Attività da svolgere (Sprint Backlog):} L'elenco puntuale dei task (studio, stesura documenti, PoC) selezionati dal Product Backlog.
        \item \textbf{Preventivo:} La stima delle ore per ruolo pianificate per completare le attività del periodo.
    \end{itemize}

    \item[Rendicontazione delle Risorse] \hfill \\
    Analisi quantitativa a consuntivo dell'impegno orario ed economico. Include:
    \begin{itemize}
        \item \textbf{Consuntivo:} Le ore effettivamente impiegate nel periodo corrente.
        \item \textbf{Riepilogo Ore Individuali:} Tabella di monitoraggio dell'impegno orario cumulativo per ciascun membro.
        \item \textbf{Bilancio di Periodo:} Tabella riepilogativa che confronta le risorse disponibili all'inizio del periodo con quelle consumate, calcolando il residuo finale.
    \end{itemize}

    \item[Considerazioni finali (Sprint Review e Retrospective)] \hfill \\
    Analisi qualitativa dell'andamento del periodo, divisa in:
    \begin{itemize}
        \item \textbf{Esito Sprint Review:} Verifica del raggiungimento degli obiettivi e completamento degli artefatti (documentali o software).
        \item \textbf{Esito Sprint Retrospective:} Analisi critica del metodo di lavoro, gestione dei rischi emersi e definizione di azioni correttive per il periodo successivo.
    \end{itemize}
\end{description}
\newpage


\section{Periodi di Avanzamento}

% ===================================================================
%  PERIODO 1
% ===================================================================
\subsection{Primo periodo: 06/11/2025 - 14/11/2025}

\subsubsection*{Pianificazione (Sprint Planning)}

%%%%%%%%%%%%%%%%%%%%% breve introduzione del periodo %%%%%%%%%%%%%%%%%%%%%%%
Durante questo primo sprint il gruppo ha concentrato le proprie attività
sull’acquisizione delle conoscenze necessarie per affrontare efficacemente la milestone \textit{RTB (Requirements and Technology Baseline)}.  
In particolare ci siamo dedicati allo studio dei requisiti di progetto (organizzando anche una riunione con l'azienda proponente) oltre che alla stesura dei documenti Norme di progetto e del Glossario.

%%%%%%%%%%%%%%%%%%%%% TABELLA Attività pianificate %%%%%%%%%%%%%%%%%%%%%%%
\begin{table}[!ht]
\vspace{0.2cm}
\centering
{\large \textbf{Attività pianificate} (Sprint Backlog)} \\
\vspace{0.3cm}
\renewcommand{\arraystretch}{1.4}
\rowcolors{2}{white}{green!50!black!20!white}
\begin{tabularx}{\textwidth}{|X|c|} 
\hline
\rowcolor{white} 
\textbf{Descrizione Attività} & \textbf{Ruolo Assegnato} \\
\hline
Stesura Glossario & Analista \\
Stesura Norme di Progetto & Amministratore \\
Aggiornare e sistemare sito web dedicato alla documenazione & Programmatore \\
Predisporre repository per la fase RTB & Amministratore \\
Studio preliminare dei requisiti di progetto & Analista \\
Organizzare e stendere verbale dell'incontro con l'azienda proponente per definire requisiti e modalità di comunicazione & Responsabile  \\
Redarre verbale riunione interna periodica e diario di bordo & Responsabile \\
\hline
\end{tabularx}
\end{table}
\FloatBarrier

%%%%%%%%%%%%%%%%%%%%% TABELLA Preventivo %%%%%%%%%%%%%%%%%%%%%%%
\begin{table}[!ht]
\centering
{\large \textbf{Preventivo}} \\
\vspace{0.3cm}
\renewcommand{\arraystretch}{1.4}
\rowcolors{2}{gray!10}{white}
\begin{tabular}{|l|c|c|c|c|c|c|>{\cellcolor{green!50!black!20!white}}c|}
\hline
Membro           & Resp. & Amm. & Anal. & Proget. & Prog. & Ver. & Totale \\
\hline
Davide           &   2   &  2   &   -   &    -    &   -   &   -  &   4    \\
Leonardo         &   -   &  -   &   -   &    -    &   3   &   -  &   3    \\
Francesco        &   -   &  -   &   -   &    -    &   -   &   3  &   3    \\
Mihaela-Mariana  &   -   &  -   &   3   &    -    &   -   &   -  &   3    \\
Michele          &   -   &  -   &   3   &    -    &   -   &   -  &   3    \\
Samuele          &   -   &  -   &   3   &    -    &   -   &   -  &   3    \\
Giovanni         &   -   &  1   &   -   &    -    &   3   &   -  &   4    \\
\hline  
\hiderowcolors  
\rowcolor{green!50!black!20!white}  
Totale           &   2   &  3   &   9   &    0    &   6   &   3  &   23    \\
\hline
\end{tabular}
\end{table}
\FloatBarrier
\newpage

%%%%%%%%%%%%%%%%%%%%%%%%%%%%%%%%%%%%%%%%%%%%%%%%%%%%%%%%%%%%%%%%

\subsubsection*{Rendicontazione delle Risorse}

%%%%%%%%%%%%%%%%%%%%% TABELLA Consuntivo %%%%%%%%%%%%%%%%%%%%%%%
\begin{table}[!ht]
\centering
{\large \textbf{Consuntivo}} \\
\vspace{0.3cm}
\renewcommand{\arraystretch}{1.4}
\rowcolors{2}{gray!10}{white}
\begin{tabular}{|l|c|c|c|c|c|c|>{\cellcolor{blue!10}}c|}
\hline
Membro           & Resp. & Amm. & Anal. & Proget. & Prog. & Ver. & Totale \\
\hline
Davide           &   3   &  3   &   -   &    -    &   -   &   1  &   7    \\
Leonardo         &   -   &  -   &   -   &    -    &   3   &   1  &   4    \\
Francesco        &   -   &  1   &   1   &    -    &   -   &   3  &   5    \\
Mihaela-Mariana  &   -   &  -   &   4   &    -    &   -   &   -  &   4    \\
Michele          &   -   &  3   &   -   &    -    &   -   &   -  &   3    \\
Samuele          &   -   &  -   &   4   &    -    &   -   &   -  &   4    \\
Giovanni         &   -   &  2   &   -   &    -    &   3   &   1  &   6    \\
\hline  
\hiderowcolors  
\rowcolor{blue!10}  
Totale           &   3   &  9   &   9   &    0    &   6   &   6  &   33    \\
\hline
\end{tabular}
\end{table}
\FloatBarrier

%%%%%%%%%%%%%%%%%%%%% TABELLA Riepilogo ore individuali %%%%%%%%%%%%%%%%%%%%%%%
\begin{table}[!ht]
\centering
{\large \textbf{Riepilogo ore individuali}} \\
\vspace{0.3cm}
\renewcommand{\arraystretch}{1.4}
\rowcolors{2}{gray!10}{white}
\begin{tabular}{|l|c|c|>{\cellcolor{blue!10}}c|}
\hline
    Membro              & Ore Pregresse & Ore Periodo & Ore Totali \\
\hline
Biasuzzi Davide         & 0             & 7           & 7          \\
Bilato Leonardo         & 0             & 4           & 4          \\
Zanella Francesco       & 0             & 5           & 5          \\
Romascu Mihaela-Mariana & 0             & 4           & 4          \\
Ogniben Michele         & 0             & 3           & 3          \\
Perozzo Samuele         & 0             & 4           & 4          \\
Ponso Giovanni          & 0             & 6           & 6          \\
\hline
\end{tabular}
\end{table}
\FloatBarrier


%%%%%%%%%%%%%%%%%%%%% TABELLA Bilancio economico %%%%%%%%%%%%%%%%%%%%%%%
\begin{table}[!ht]
\centering
{\large \textbf{Bilancio economico} (basato su \hyperref[sec:preventivo_v2]{Preventivo Costi - Seconda stesura})} \\
\vspace{0.3cm}
\label{tab:bilancio_p1}
\renewcommand{\arraystretch}{1.4} 
\definecolor{light-gray}{gray}{0.95}
\newcolumntype{Y}{>{\centering\arraybackslash}p{1.3cm}}
\rowcolors{3}{gray!10}{white}
\begin{tabular}{|l|Y|Y|Y|Y|>{\cellcolor{blue!10}}Y|>{\cellcolor{blue!10}}Y|} 
\hline
\multicolumn{1}{|c|}{Ruolo} & \multicolumn{2}{c|}{Residuo Inizio} & \multicolumn{2}{c|}{Consuntivo} & \multicolumn{2}{c|}{\cellcolor{blue!10}Residuo Fine}\\
& h & € & h & € & \cellcolor{blue!10}h & \cellcolor{blue!10}€ \\
\hline
%        Ruolo        & residuo inizio [ h  |   €   ]  & consuntivo [ h | € ]  & residuo fine [ h  |   €  ] \\
Responsabile (30€/h)  &                 63  & 1890    &             3 & 90     &               60  & 1800  \\
Amministratore (20€/h)&                 63  & 1260    &             9 & 180    &               54  & 1080  \\
Analista (25€/h)      &                 107 & 2675    &             9 & 225    &               98  & 2450  \\
Progettista (25€/h)   &                 107 & 2675    &             0 & 0      &               107 & 2675  \\
Programmatore (15€/h) &                 157 & 2355    &             6 & 90     &               151 & 2265  \\
Verificatore (15€/h)  &                 133 & 1995    &             6 & 90     &               127 & 1905  \\
\hline
\rowcolor{blue!10} 
Totale                &                 630 &   12850 &            33 & 675    &               597 & 12175 \\
\hline
\end{tabular}
\end{table}
\FloatBarrier
\newpage

%%%%%%%%%%%%%%%%%%%%% Valutazione e miglioramento  %%%%%%%%%%%%%%%%%%%%%%%
\subsubsection*{Considerazioni finali (Sprint Review e Retrospective)}
Il gruppo ha rispettato gli obiettivi previsti per questo primo periodo di lavoro. Le ore impiegate risultano superiori al preventivo iniziale. Considerando la natura introduttiva delle attività svolte e la necessità di studio preliminare ci aspettavamo di disallinearci con le stime iniziali.
Le attività documentali e organizzative sono state completate secondo quanto pianificato, garantendo una base solida per proseguire con il periodo successivo.
\newpage
% ==================================================================
%  FINE PERIODO 1
% ==================================================================

% ===================================================================
%  PERIODO 2
% ===================================================================
\subsection{Secondo periodo: 15/11/2025 - 21/11/2025}

\subsubsection*{Pianificazione (Sprint Planning)}

%%%%%%%%%%%%%%%%%%%%% breve introduzione del periodo %%%%%%%%%%%%%%%%%%%%%%%
Nel corso di questa settimana il gruppo si é dedicato alla stesura del documento Piano di Progetto e alla scrittura di una prima bozza dell’Analisi dei Requisiti. Il lavoro si è concentrato sulla definizione accurata delle funzionalità richieste e sulla pianificazione delle successive fasi operative, così da garantire una preparazione adeguata per il raggiungimento della milestone RTB.

%%%%%%%%%%%%%%%%%%%%% TABELLA Attività pianificate %%%%%%%%%%%%%%%%%%%%%%%
\begin{table}[!ht]
\vspace{0.2cm}
\centering
{\large \textbf{Attività pianificate} (Sprint Backlog)} \\
\vspace{0.3cm}
\renewcommand{\arraystretch}{1.4}
\rowcolors{2}{white}{green!50!black!20!white}
\begin{tabularx}{\textwidth}{|X|c|} 
\hline
\rowcolor{white} 
\textbf{Descrizione Attività} & \textbf{Ruolo Assegnato} \\
\hline
Aggiornare Glossario con le modifiche discusse & Analista \\
Abbozzare una primissima lista dei requisiti di progetto & Analista \\
Iniziare la stesura di Piano di Progetto & Progettista \\
Redarre verbale riunione interna periodica e diario di bordo & Responsabile \\
\hline
\end{tabularx}
\end{table}
\FloatBarrier

%%%%%%%%%%%%%%%%%%%%% TABELLA Preventivo %%%%%%%%%%%%%%%%%%%%%%%
\begin{table}[!ht]
\centering
{\large \textbf{Preventivo}} \\
\vspace{0.3cm}
\renewcommand{\arraystretch}{1.4}
\rowcolors{2}{gray!10}{white}
\begin{tabular}{|l|c|c|c|c|c|c|>{\cellcolor{green!50!black!20!white}}c|}
\hline
Membro           & Resp. & Amm. & Anal. & Proget. & Prog. & Ver. & Totale \\
\hline
Davide           &   3   &  1   &   -   &    -    &   -   &   -  &   4    \\
Leonardo         &   -   &  -   &   2   &    -    &   -   &   -  &   2   \\
Francesco        &   -   &  -   &   -   &    2    &   -   &   1  &   3   \\
Mihaela-Mariana  &   -   &  -   &   2   &    -    &   -   &   -  &   2    \\
Michele          &   -   &  -   &   -   &    2    &   -   &   1  &   3    \\
Samuele          &   -   &  -   &   2   &    -    &   -   &   -  &   2    \\
Giovanni         &   -   &  -   &   -   &    2    &   -   &   -  &   2    \\
\hline  
\hiderowcolors  
\rowcolor{green!50!black!20!white}  
Totale           &   3   &  1   &   6   &    6    &   0   &   2  &   18   \\
\hline
\end{tabular}
\end{table}
\FloatBarrier
\newpage

%%%%%%%%%%%%%%%%%%%%%%%%%%%%%%%%%%%%%%%%%%%%%%%%%%%%%%%%%%%%%%%%

\subsubsection*{Rendicontazione delle Risorse}

%%%%%%%%%%%%%%%%%%%%% TABELLA Consuntivo %%%%%%%%%%%%%%%%%%%%%%%
\begin{table}[!ht]
\centering
{\large \textbf{Consuntivo}} \\
\vspace{0.3cm}
\renewcommand{\arraystretch}{1.4}
\rowcolors{2}{gray!10}{white}
\begin{tabular}{|l|c|c|c|c|c|c|>{\cellcolor{blue!10}}c|}
\hline
Membro           & Resp. & Amm. & Anal. & Proget. & Prog. & Ver. & Totale \\
\hline
Davide           &   3   &  1   &   -   &    -    &   -   &   -  &   4    \\
Leonardo         &   -   &  -   &   2   &    -    &   -   &   -  &   2   \\
Francesco        &   -   &  -   &   -   &    2    &   -   &   2  &   4    \\
Mihaela-Mariana  &   -   &  -   &   2   &    -    &   -   &   -  &   2    \\
Michele          &   -   &  -   &   -   &    3    &   -   &   1  &   4    \\
Samuele          &   -   &  -   &   2   &    -    &   -   &   -  &   2    \\
Giovanni         &   -   &  -   &   -   &    3    &   -   &   -  &   3    \\
\hline  
\hiderowcolors
\rowcolor{blue!10}
Totale           &   3   &  1   &   6   &    8    &   0   &   3  &   21   \\
\hline
\end{tabular}
\end{table}
\FloatBarrier

%%%%%%%%%%%%%%%%%%%%% TABELLA Riepilogo ore individuali %%%%%%%%%%%%%%%%%%%%%%%
\begin{table}[!ht]
\centering
{\large \textbf{Riepilogo ore individuali}} \\
\vspace{0.3cm}
\renewcommand{\arraystretch}{1.4}
\rowcolors{2}{gray!10}{white}
\begin{tabular}{|l|c|c|>{\cellcolor{blue!10}}c|}
\hline
    Membro              & Ore Pregresse & Ore Periodo & Ore Totali \\
\hline
Biasuzzi Davide         & 7             & 4           & 11          \\
Bilato Leonardo         & 4             & 2           & 6           \\
Zanella Francesco       & 5             & 4           & 9          \\
Romascu Mihaela-Mariana & 4             & 2           & 6           \\
Ogniben Michele         & 3             & 4           & 7           \\
Perozzo Samuele         & 4             & 2           & 6           \\
Ponso Giovanni          & 6             & 3           & 9          \\
\hline
\end{tabular}
\end{table}
\FloatBarrier


%%%%%%%%%%%%%%%%%%%%% TABELLA Bilancio economico %%%%%%%%%%%%%%%%%%%%%%%
\begin{table}[!ht]
\centering
{\large \textbf{Bilancio economico} (basato su \hyperref[sec:preventivo_v2]{Preventivo Costi - Seconda stesura})} \\
\vspace{0.3cm}
\label{tab:bilancio_p2}
\renewcommand{\arraystretch}{1.4} 
\definecolor{light-gray}{gray}{0.95}
\newcolumntype{Y}{>{\centering\arraybackslash}p{1.3cm}}
\rowcolors{3}{gray!10}{white}
\begin{tabular}{|l|Y|Y|Y|Y|>{\cellcolor{blue!10}}Y|>{\cellcolor{blue!10}}Y|} 
\hline
\multicolumn{1}{|c|}{Ruolo} & \multicolumn{2}{c|}{Residuo Inizio} & \multicolumn{2}{c|}{Consuntivo} & \multicolumn{2}{c|}{\cellcolor{blue!10}Residuo Fine}\\
& h & € & h & € & \cellcolor{blue!10}h & \cellcolor{blue!10}€ \\
\hline
%        Ruolo        & residuo inizio [ h  |   €   ] & consuntivo [ h| € ]   & residuo fine [ h  |   €  ] \\
Responsabile (30€/h)  &                 60  & 1800    &             3 & 90    &               57  & 1710  \\
Amministratore (20€/h)&                 54  & 1080    &             1 & 20    &               53  & 1060  \\
Analista (25€/h)      &                 98  & 2450    &             6 & 150   &               92  & 2300  \\
Progettista (25€/h)   &                 107 & 2675    &             8 & 200   &               99  & 2475  \\
Programmatore (15€/h) &                 151 & 2265    &             0 & 0     &               151 & 2265  \\
Verificatore (15€/h)  &                 127 & 1905    &             3 & 45    &               124 & 1860  \\
\hline
\rowcolor{blue!10} 
Totale                &                 597 & 12175   &            21 &   505 &               576 & 11670 \\
\hline
\end{tabular}
\end{table}
\FloatBarrier
\newpage

%%%%%%%%%%%%%%%%%%%%% Valutazione e miglioramento  %%%%%%%%%%%%%%%%%%%%%%%
\subsubsection*{Considerazioni finali (Sprint Review e Retrospective)}
Il gruppo ha portato avanti in modo efficace le attività previste per questo secondo periodo, consolidando la documentazione necessaria in vista della milestone RTB. L’allineamento interno e la chiarezza raggiunta nei documenti confermano la solidità dell’organizzazione del lavoro. In questo periodo le ore impiegate risultano coerenti con quanto pianificato.
\newpage
% ==================================================================
%  FINE PERIODO 2
% ==================================================================

% ==================================================================
%  INIZIO PERIODO 3
% ==================================================================

\subsection{Terzo periodo: 22/11/2025 - 28/11/2025}

\subsubsection*{Pianificazione (Sprint Planning)}

%%%%%%%%%%%%%%%%%%%%% breve introduzione del periodo %%%%%%%%%%%%%%%%%%%%%%%
Durante questo terzo periodo, l'attività del gruppo si è concentrata prevalentemente sulla prima stesura del documento \textit{Analisi dei Requisiti}, approfondendo la specifica dei casi d'uso e dei requisiti funzionali emersi. \\Parallelamente i membri del gruppo si sono suddivisi i compiti per aggiornare il \textit{Piano di Progetto}, ricostruendo il consuntivo delle attività svolte nei periodi precedenti e raffinando la pianificazione temporale ed economica fino alla data corrente. L'obiettivo è stato allineare la documentazione di gestione con l'effettivo avanzamento dei lavori in vista della milestone \textit{RTB}.

%%%%%%%%%%%%%%%%%%%%% TABELLA Attività pianificate %%%%%%%%%%%%%%%%%%%%%%%
\begin{table}[!ht]
\vspace{0.2cm}
\centering
{\large \textbf{Attività pianificate} (Sprint Backlog)} \\
\vspace{0.3cm}
\renewcommand{\arraystretch}{1.4}
\rowcolors{2}{white}{green!50!black!20!white}
\begin{tabularx}{\textwidth}{|X|c|} 
\hline
\rowcolor{white} 
\textbf{Descrizione Attività} & \textbf{Ruolo Assegnato} \\
\hline
Scrivere una prima bozza di analisi dei requisiti & Analista \\
Terminare "introduzione" docuemento Piano di progetto & Progettista \\
Scrivere Periodo 1 in Piano di Progetto & Progettista \\
Scrivere Periodo 2 in Piano di Progetto & Progettista \\ 
Redarre verbale riunione interna periodica, diario di bordo e rendicontazione periodo attuale & Responsabile \\
\hline
\end{tabularx}
\end{table}
\FloatBarrier

%%%%%%%%%%%%%%%%%%%%% TABELLA Preventivo %%%%%%%%%%%%%%%%%%%%%%%
\begin{table}[!ht]
\centering
{\large \textbf{Preventivo}} \\
\vspace{0.3cm}
\renewcommand{\arraystretch}{1.4}
\rowcolors{2}{gray!10}{white}
\begin{tabular}{|l|c|c|c|c|c|c|>{\cellcolor{green!50!black!20!white}}c|}
\hline
Membro           & Resp. & Amm. & Anal. & Proget. & Prog. & Ver. & Totale \\
\hline
Davide           &   -   &  -   &   2   &    -    &   -   &   -  &   2    \\
Leonardo         &   -   &  -   &   -   &    -    &   -   &   2  &   2    \\
Francesco        &   -   &  -   &   -   &    2    &   -   &   -  &   2    \\
Mihaela-Mariana  &   -   &  -   &   2   &    2    &   -   &   -  &   4    \\
Michele          &   -   &  -   &   -   &    2    &   -   &   -  &   2    \\
Samuele          &   -   &  -   &   2   &    2    &   -   &   -  &   4    \\
Giovanni         &   2   &  -   &   -   &    1    &   -   &   -  &   3    \\
\hline  
\hiderowcolors  
\rowcolor{green!50!black!20!white}  
Totale           &   2   &  -   &   6   &    9   &   -   &   2  &  19    \\
\hline
\end{tabular}
\end{table}
\FloatBarrier
\newpage

%%%%%%%%%%%%%%%%%%%%%%%%%%%%%%%%%%%%%%%%%%%%%%%%%%%%%%%%%%%%%%%%

\subsubsection*{Rendicontazione delle Risorse}

%%%%%%%%%%%%%%%%%%%%% TABELLA Consuntivo %%%%%%%%%%%%%%%%%%%%%%%
\begin{table}[!ht]
\centering
{\large \textbf{Consuntivo}} \\
\vspace{0.3cm}

\renewcommand{\arraystretch}{1.4}
\rowcolors{2}{gray!10}{white}
\begin{tabular}{|l|c|c|c|c|c|c|>{\cellcolor{blue!10}}c|}
\hline
Membro           & Resp. & Amm. & Anal. & Proget. & Prog. & Ver. & Totale \\
\hline
Davide           &   -   &  -   &   3   &    -    &   -   &   -  &   3    \\
Leonardo         &   -   &  -   &   -   &    -    &   -   &   4  &   4    \\
Francesco        &   -   &  -   &   -   &    2    &   -   &   -  &   2    \\
Mihaela-Mariana  &   -   &  -   &   1   &    3    &   -   &   -  &   4    \\
Michele          &   -   &  -   &   -   &    2    &   -   &   -  &   2    \\
Samuele          &   -   &  -   &   2   &    2    &   -   &   -  &   4    \\
Giovanni         &   3   &  -   &   -   &    1    &   -   &   -  &   4    \\
\hline
\hiderowcolors
\rowcolor{blue!10}
Totale           &   3   &  -   &   6   &    10    &   -   &   4  &  23    \\
\hline
\end{tabular}
\end{table}
\FloatBarrier
%%%%%%%%%%%%%%%%%%%%% TABELLA Riepilogo ore individuali %%%%%%%%%%%%%%%%%%%%%%%

\begin{table}[!ht]
\centering
{\large \textbf{Riepilogo ore individuali}} \\
\vspace{0.3cm}
\renewcommand{\arraystretch}{1.4}
\rowcolors{2}{gray!10}{white}
\begin{tabular}{|l|c|c|>{\cellcolor{blue!10}}c|}
\hline
    Membro              & Ore Pregresse & Ore Periodo & Ore Totali \\
\hline
Biasuzzi Davide         & 11            & 3           & 14          \\
Bilato Leonardo         & 6             & 4           & 10          \\
Zanella Francesco       & 9             & 2           & 11          \\
Romascu Mihaela-Mariana & 6             & 4           & 10          \\
Ogniben Michele         & 7             & 2           & 9          \\
Perozzo Samuele         & 6             & 4           & 10         \\
Ponso Giovanni          & 9             & 4           & 13         \\
\hline
\end{tabular}
\end{table}
\FloatBarrier

%%%%%%%%%%%%%%%%%%%%% TABELLA Bilancio economico %%%%%%%%%%%%%%%%%%%%%%%

\begin{table}[!ht]
\centering
{\large \textbf{Bilancio economico} (basato su \hyperref[sec:preventivo_v2]{Preventivo Costi - Seconda stesura})} \\
\vspace{0.3cm}
\label{tab:bilancio_p1} 
\renewcommand{\arraystretch}{1.4} 
\definecolor{light-gray}{gray}{0.95}
\newcolumntype{Y}{>{\centering\arraybackslash}p{1.3cm}}
\rowcolors{3}{gray!10}{white}
\begin{tabular}{|l|Y|Y|Y|Y|>{\cellcolor{blue!10}}Y|>{\cellcolor{blue!10}}Y|} 
\hline
\multicolumn{1}{|c|}{Ruolo} & \multicolumn{2}{c|}{Residuo Inizio} & \multicolumn{2}{c|}{Consuntivo} & \multicolumn{2}{c|}{\cellcolor{blue!10}Residuo Fine}\\
& h & € & h & € & \cellcolor{blue!10}h & \cellcolor{blue!10}€ \\
\hline
%        Ruolo        & residuo inizio [ h  |   €   ]  & consuntivo [ h  | € ]  & residuo fine [   h  |   €  ] \\
Responsabile (30€/h)  &                 57  & 1710     &              3  & 90     &               54  & 1620  \\
Amministratore (20€/h)&                 53  & 1060     &              0  & 0      &               53  & 1060  \\
Analista (25€/h)      &                 92  & 2300     &              6  & 150    &               86  & 2150  \\
Progettista (25€/h)   &                 99  & 2475     &              10 & 250    &               89  & 2225  \\
Programmatore (15€/h) &                 151 & 2265     &              0  & 0      &               151 & 2265  \\
Verificatore (15€/h)  &                 124 & 1860     &              4  & 60     &               120 & 1800  \\
\hline
\rowcolor{blue!10} 
Totale                &                 576 & 11670    &              23 & 550    &                553 & 11120 \\
\hline
\end{tabular}
\end{table}
\FloatBarrier
\newpage

%%%%%%%%%%%%%%%%%%%%% Valutazione e miglioramento  %%%%%%%%%%%%%%%%%%%%%%%
\subsubsection*{Considerazioni finali (Sprint Review e Retrospective)}
Il gruppo ha ottemperato positivamente agli obiettivi del terzo periodo. Sebbene il monte ore consuntivato abbia superato quello preventivato, discrepanza che verrà mitigata in futuro tramite stime più conservative, l'efficienza operativa è stata buona.
\newpage
% ==================================================================
%  FINE PERIODO 3
% ==================================================================

% ==================================================================
%  INIZIO PERIODO 4
% ==================================================================

\subsection{Quarto periodo: 29/11/2025 - 05/12/2025}

\subsubsection*{Pianificazione (Sprint Planning)}

%%%%%%%%%%%%%%%%%%%%% breve introduzione del periodo %%%%%%%%%%%%%%%%%%%%%%%
Durante questo periodo il gruppo dovrà aggiornare Norme di progetto con i compiti precisi del responsabile discussi in riunione, inoltre  inizierà lo studio preliminare del PoC. È stato previsto di organizzare una chiamata con Ergon per chiarire i dubbi sorti durante l'analisi dei requisiti, ma a causa di impossibilità da parte del referente Ergon, questa è stata posticipata al periodo successivo. Abbiamo quindi deciso di procedere con attività alternative, ovvero l'implementazione dell'indice gulpease e di languagetool nella repo della documentazione. Infine verrá riorganizzato Github Projects per assicurare un workflow fluido ed efficiente del team.

%%%%%%%%%%%%%%%%%%%%% TABELLA Attività pianificate %%%%%%%%%%%%%%%%%%%%%%%


\begin{figure}[!ht]
    \vspace{0.2cm}
    \centering
    {\large \textbf{Attività pianificate} (Sprint Backlog)} \\
    \vspace{0.2cm}
    \centering
    \includegraphics[width=0.7\linewidth]{activity_sprint_4.png}
\end{figure}
\FloatBarrier

%%%%%%%%%%%%%%%%%%%%% TABELLA Preventivo %%%%%%%%%%%%%%%%%%%%%%%
\begin{table}[!ht]
\centering
{\large \textbf{Preventivo}} \\
\vspace{0.3cm}
\renewcommand{\arraystretch}{1.4}
\rowcolors{2}{gray!10}{white}
\begin{tabular}{|l|c|c|c|c|c|c|>{\cellcolor{green!50!black!20!white}}c|}
\hline
Membro           & Resp. & Amm. & Anal. & Proget. & Prog. & Ver. & Totale \\
\hline
Davide           &   -   &  1   &   -   &    -    &   3   &   -  &   4    \\
Leonardo         &   3   &  -   &   -   &    -    &   -   &   -  &   3    \\
Francesco        &   -   &  -   &   1.5   &    -    &   -   &   2  &   3.5    \\
Mihaela-Mariana  &   -   &  2   &   -   &    -    &   -   &   -  &   2    \\
Michele          &   -   &  2   &   -   &    -    &   -   &   -  &   2    \\
Samuele          &   -   &  -   &   -   &    3    &   -   &   -  &   3    \\
Giovanni         &   -   &  1   &   -   &    -    &   -   &   2  &   3    \\
\hline  
\hiderowcolors  
\rowcolor{green!50!black!20!white}  
Totale           &   3   &  6   &   1.5   &    3    &   3   &   4  &   20.5    \\
\hline
\end{tabular}
\end{table}
\FloatBarrier
\newpage


%%%%%%%%%%%%%%%%%%%%%%%%%%%%%%%%%%%%%%%%%%%%%%%%%%%%%%%%%%%%%%%%

\subsubsection*{Rendicontazione delle Risorse}

%%%%%%%%%%%%%%%%%%%%% TABELLA Consuntivo %%%%%%%%%%%%%%%%%%%%%%%
\begin{table}[!ht]
\centering
{\large \textbf{Consuntivo}} \\
\vspace{0.3cm}

\renewcommand{\arraystretch}{1.4}
\rowcolors{2}{gray!10}{white}
\begin{tabular}{|l|c|c|c|c|c|c|>{\cellcolor{blue!10}}c|}
\hline
Membro           & Resp. & Amm. & Anal. & Proget. & Prog. & Ver. & Totale \\
\hline
Davide           &   -   &   1  &   -   &    -    &   4   &   -  &   5    \\
Leonardo         &   3   &   -  &   -   &    -    &   -   &   -  &   3    \\
Francesco        &   -   &   -  &   3   &    -    &   -   &   -  &   3    \\
Mihaela-Mariana  &   -   &   2  &   -   &    -    &   -   &   -  &   2    \\
Michele          &   -   &   2  &   -   &    -    &   -   &   -  &   2    \\
Samuele          &   -   &   -  &   -   &    2    &   -   &   -  &   2    \\
Giovanni         &   -   &   2  &   -   &    -    &   -   &   2  &   4    \\
\hline
\hiderowcolors
\rowcolor{blue!10}
Totale           &   3   &   7  &   3   &    2    &   4   &   2  &   21    \\
\hline
\end{tabular}
\end{table}
\FloatBarrier

%%%%%%%%%%%%%%%%%%%%% TABELLA Riepilogo ore individuali %%%%%%%%%%%%%%%%%%%%%%%
\begin{table}[!ht]
\centering
{\large \textbf{Riepilogo ore individuali}} \\
\vspace{0.3cm}
\renewcommand{\arraystretch}{1.4}
\rowcolors{2}{gray!10}{white}
\begin{tabular}{|l|c|c|>{\cellcolor{blue!10}}c|}
\hline
    Membro              & Ore Pregresse & Ore Periodo & Ore Totali \\
\hline
Biasuzzi Davide         & 14             & 5           & 19          \\
Bilato Leonardo         & 10             & 3           & 13          \\
Zanella Francesco       & 11             & 3           & 14          \\
Romascu Mihaela-Mariana & 10             & 2           & 12          \\
Ogniben Michele         & 9             & 2          & 11          \\
Perozzo Samuele         & 10             & 2           & 12          \\
Ponso Giovanni          & 13             & 4           & 17          \\
\hline
\end{tabular}
\end{table}
\FloatBarrier

%%%%%%%%%%%%%%%%%%%%% TABELLA Bilancio economico %%%%%%%%%%%%%%%%%%%%%%%
\begin{table}[!ht]
\centering
{\large \textbf{Bilancio economico} (basato su \hyperref[sec:preventivo_v2]{Preventivo Costi - Seconda stesura})} \\
\vspace{0.3cm}
\label{tab:bilancio_p1} 
\renewcommand{\arraystretch}{1.4} 
\definecolor{light-gray}{gray}{0.95}
\newcolumntype{Y}{>{\centering\arraybackslash}p{1.3cm}}
\rowcolors{3}{gray!10}{white}
\begin{tabular}{|l|Y|Y|Y|Y|>{\cellcolor{blue!10}}Y|>{\cellcolor{blue!10}}Y|} 
\hline
\multicolumn{1}{|c|}{Ruolo} & \multicolumn{2}{c|}{Residuo Inizio} & \multicolumn{2}{c|}{Consuntivo} & \multicolumn{2}{c|}{\cellcolor{blue!10}Residuo Fine}\\
& h & € & h & € & \cellcolor{blue!10}h & \cellcolor{blue!10}€ \\
\hline
%        Ruolo        & residuo inizio [ h  |   €   ]  & consuntivo [ h  | € ]  & residuo fine [   h  |   €  ] \\
Responsabile (30€/h)  &                 54  & 1620    &             3 & 90    &               51  & 1530  \\
Amministratore (20€/h)&                 53  & 1060    &             7 & 140    &               46  & 920  \\
Analista (25€/h)      &                 86 & 2150    &             3 & 75   &               83 & 2075  \\
Progettista (25€/h)   &                 89 & 2225    &             2 & 50   &               87 & 2175  \\
Programmatore (15€/h) &                 151 & 2265    &             4 & 60    &               147 & 2205  \\
Verificatore (15€/h)  &                 120 & 1800    &             2 & 30    &               118 & 1770  \\
\hline
\rowcolor{blue!10} 
Totale                &                 553 &   11120 &            21 &   445 &               532 & 10675 \\
\hline
\end{tabular}
\end{table}
\FloatBarrier
\newpage

%%%%%%%%%%%%%%%%%%%%% Valutazione e miglioramento  %%%%%%%%%%%%%%%%%%%%%%%
\subsubsection*{Considerazioni finali (Sprint Review e Retrospective)}
Il gruppo ha rispettato gli obiettivi previsti per questo periodo. Le ore impiegate risultano sufficientemente coerenti con il preventivo iniziale.
\newpage

% ==================================================================
%  FINE PERIODO 4
% ==================================================================

% ==================================================================
%   INIZIO PERIODO 5
% ==================================================================

\subsection{Quinto periodo: 06/12/2025 - 12/12/2025}

\subsubsection*{Pianificazione (Sprint Planning)}

%%%%%%%%%%%%%%%%%%%%% breve introduzione del periodo %%%%%%%%%%%%%%%%%%%%%%%
Durante questo periodo il gruppo si è concentrato sulle attività preparatorie per la realizzazione del Proof of Concept (PoC). L'attività principale ha riguardato lo studio delle tecnologie individuate: React per la parte di applicazione web lato utente e l'approfondimento sui Large Language Models (LLM). Parallelamente, gli analisti hanno proseguito con la stesura definitiva degli Use Case nell'Analisi dei Requisiti. Il Responsabile ha curato l'aggiornamento del Piano di Progetto fissando la milestone per la RTB, mentre lato programmazione è stata finalizzata la configurazione degli strumenti di supporto alla documentazione (LanguageTool e workflow).

%%%%%%%%%%%%%%%%%%%%% TABELLA Attività pianificate %%%%%%%%%%%%%%%%%%%%%%%

\begin{figure}[!ht]
    \vspace{0.2cm}
    \centering
    {\large \textbf{Attività pianificate} (Sprint Backlog)} \\
    \vspace{0.2cm}
    \centering
    \includegraphics[width=0.7\linewidth]{activity_sprint_5.png}
\end{figure}
\FloatBarrier

%%%%%%%%%%%%%%%%%%%%% TABELLA Preventivo %%%%%%%%%%%%%%%%%%%%%%%
\begin{table}[!ht]
\centering
{\large \textbf{Preventivo}} \\
\vspace{0.3cm}
\renewcommand{\arraystretch}{1.4}
\rowcolors{2}{gray!10}{white}
\begin{tabular}{|l|c|c|c|c|c|c|>{\cellcolor{green!50!black!20!white}}c|}
\hline
Membro           & Resp. & Amm. & Anal. & Proget. & Prog. & Ver. & Totale \\
\hline
Davide           &   -   &  -   &   -   &    -    &   2   &   -  &   2    \\
Leonardo         &   -   &  -   &   -   &    2    &   -   &   -  &   2    \\
Francesco        &   -   &  -   &   2   &    -    &   -   &   -  &   2    \\
Mihaela-Mariana  &   -   &  -   &   -   &    -    &   -   &   2  &   2    \\
Michele          &   -   &  -   &   -   &    2    &   -   &   -  &   2    \\
Samuele          &   4   &  -   &   -   &    -    &   -   &   -  &   4    \\
Giovanni         &   -   &  -   &   2   &    -    &   -   &   -  &   2    \\
\hline  
\hiderowcolors  
\rowcolor{green!50!black!20!white}  
Totale           &   4   &  -   &   4   &    4    &   2   &   2  &   16    \\
\hline
\end{tabular}
\end{table}
\FloatBarrier
\newpage


%%%%%%%%%%%%%%%%%%%%%%%%%%%%%%%%%%%%%%%%%%%%%%%%%%%%%%%%%%%%%%%%

\subsubsection*{Rendicontazione delle Risorse}

%%%%%%%%%%%%%%%%%%%%% TABELLA Consuntivo %%%%%%%%%%%%%%%%%%%%%%%
\begin{table}[!ht]
\centering
{\large \textbf{Consuntivo}} \\
\vspace{0.3cm}

\renewcommand{\arraystretch}{1.4}
\rowcolors{2}{gray!10}{white}
\begin{tabular}{|l|c|c|c|c|c|c|>{\cellcolor{blue!10}}c|}
\hline
Membro           & Resp. & Amm. & Anal. & Proget. & Prog. & Ver. & Totale \\
\hline
Davide           &   -   &   -  &   -   &    -    &   2   &   -  &   2    \\
Leonardo         &   -   &   -  &   -   &    3    &   -   &   -  &   3    \\
Francesco        &   -   &   -  &   2   &    -    &   -   &   -  &   2    \\
Mihaela-Mariana  &   -   &   -  &   -   &    -    &   -   &   2  &   2    \\
Michele          &   -   &   -  &   -   &    4    &   -   &   -  &   4    \\
Samuele          &   4   &   -  &   -   &    -    &   -   &   -  &   4    \\
Giovanni         &   -   &   -  &   2   &    -    &   -   &   -  &   2    \\
\hline
\hiderowcolors
\rowcolor{blue!10}
Totale           &   4   &   -  &   4   &    7    &   2   &   2  &   19    \\
\hline
\end{tabular}
\end{table}
\FloatBarrier

%%%%%%%%%%%%%%%%%%%%% TABELLA Riepilogo ore individuali %%%%%%%%%%%%%%%%%%%%%%%
\begin{table}[!ht]
\centering
{\large \textbf{Riepilogo ore individuali}} \\
\vspace{0.3cm}
\renewcommand{\arraystretch}{1.4}
\rowcolors{2}{gray!10}{white}
\begin{tabular}{|l|c|c|>{\cellcolor{blue!10}}c|}
\hline
    Membro              & Ore Pregresse & Ore Periodo & Ore Totali \\
\hline
Biasuzzi Davide         & 19             & 2           & 21          \\
Bilato Leonardo         & 13             & 3           & 16          \\
Zanella Francesco       & 14             & 2           & 16          \\
Romascu Mihaela-Mariana & 12             & 2           & 14          \\
Ogniben Michele         & 11             & 4           & 15          \\
Perozzo Samuele         & 12             & 4           & 16          \\
Ponso Giovanni          & 17             & 2           & 19          \\
\hline
\end{tabular}
\end{table}
\FloatBarrier

%%%%%%%%%%%%%%%%%%%%% TABELLA Bilancio economico %%%%%%%%%%%%%%%%%%%%%%%
\begin{table}[!ht]
\centering
{\large \textbf{Bilancio economico} (basato su \hyperref[sec:preventivo_v2]{Preventivo Costi - Seconda stesura})} \\
\vspace{0.3cm}
\label{tab:bilancio_p5} 
\renewcommand{\arraystretch}{1.4} 
\definecolor{light-gray}{gray}{0.95}
\newcolumntype{Y}{>{\centering\arraybackslash}p{1.3cm}}
\rowcolors{3}{gray!10}{white}
\begin{tabular}{|l|Y|Y|Y|Y|>{\cellcolor{blue!10}}Y|>{\cellcolor{blue!10}}Y|} 
\hline
\multicolumn{1}{|c|}{Ruolo} & \multicolumn{2}{c|}{Residuo Inizio} & \multicolumn{2}{c|}{Consuntivo} & \multicolumn{2}{c|}{\cellcolor{blue!10}Residuo Fine}\\
& h & € & h & € & \cellcolor{blue!10}h & \cellcolor{blue!10}€ \\
\hline
%        Ruolo        & residuo inizio [ h  |   €    ]  & consuntivo [ h  | € ]  & residuo fine [   h  |   €   ] \\
Responsabile (30€/h)  &                 51  & 1530    &             4 & 120   &                 47  & 1410  \\
Amministratore (20€/h)&                 46  & 920     &             0 & 0     &                 46  & 920   \\
Analista (25€/h)      &                 83 & 2075     &             4 & 100   &                 79 & 1975  \\
Progettista (25€/h)   &                 87 & 2175     &             7 & 175   &                 80 & 2000  \\
Programmatore (15€/h) &                 147 & 2205    &             2 & 30    &                 145 & 2175  \\
Verificatore (15€/h)  &                 118 & 1770    &             2 & 30    &                 116 & 1740  \\
\hline
\rowcolor{blue!10} 
Totale                &                 532 &   10675 &            19 &   455 &                513 & 10220 \\
\hline
\end{tabular}
\end{table}
\FloatBarrier
\newpage

%%%%%%%%%%%%%%%%%%%%% Valutazione e miglioramento  %%%%%%%%%%%%%%%%%%%%%%%
\subsubsection*{Considerazioni finali (Sprint Review e Retrospective)}
Il gruppo ha raggiunto gli obiettivi prefissati. Le ore del Responsabile coincidono perfettamente con quanto preventivato. Si nota invece un leggero scostamento in positivo delle ore dei Progettisti (7 ore consuntivate contro le 4 preventivate): tale aumento è giustificato dalla necessità di approfondire lo studio del framework React e l'integrazione degli LLM, competenze ritenute fondamentali per ridurre i rischi tecnici durante la fase di codifica del PoC. Le altre attività sono rimaste in linea con le stime.
\newpage

% ==================================================================
%   FINE PERIODO 5
% ==================================================================

\end{document}