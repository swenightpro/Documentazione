\documentclass[a4paper,11pt,oneside]{scrartcl}

% --- Pacchetti Fondamentali ---
\usepackage[utf8]{inputenc}
\usepackage[T1]{fontenc}
\usepackage[italian]{babel}
\usepackage{lmodern}
\usepackage{scrhack}
\usepackage{placeins}
\usepackage{array}
\usepackage{tabularx}
\usepackage{pgf-pie}
\usepackage{amsmath, amssymb, mathtools}
\usepackage{float}

% --- Grafica e Layout ---
\usepackage{graphicx}
\graphicspath{{../../assets/}{../assets/}{assets/}}
\usepackage[a4paper, top=2.5cm, bottom=3cm, left=2.5cm, right=2.5cm]{geometry}
\usepackage{fancyhdr}
\usepackage{microtype}
\usepackage[svgnames,table]{xcolor}
\usepackage{booktabs}
\usepackage{caption}
\usepackage{hhline}

% --- Utility ---
\usepackage{enumitem}
\usepackage{hyperref}

% --- Comandi Personalizzati ---
\newcommand{\riskheader}[1]{%
    \multicolumn{2}{@{}l@{}}{\textbf{#1}} \\ \midrule
}


\newcolumntype{L}{>{\raggedright\arraybackslash}X}
\newcolumntype{R}{>{\raggedleft\arraybackslash}X}
\newcolumntype{C}{>{\centering\arraybackslash}X}

\hypersetup{
    colorlinks=true,
    linkcolor=DarkBlue,
    filecolor=DarkBlue,      
    urlcolor=DarkBlue,
    citecolor=DarkBlue,
    pdftitle={Piano di Qualifica - NightPRO},
    pdfauthor={Gruppo NightPRO},
}

% ===================================================================
%  HEADER E FOOTER
% ===================================================================
\pagestyle{fancy}
\fancyhf{}
\fancyhead[L]{NightPRO - Progetto Ingegneria del Software}
\fancyhead[R]{Anno Accademico 2025/2026}
\fancyfoot[C]{\thepage}
\renewcommand{\headrulewidth}{0.4pt}
\renewcommand{\footrulewidth}{0pt}

% ===================================================================
%  DOCUMENTO
% ===================================================================
\begin{document}

% -------------------------------------------------------------------
%  TITOLO
% -------------------------------------------------------------------
\thispagestyle{empty}

\begin{titlepage}
\centering

\includegraphics[width=0.4\textwidth]{logo.png} 

\vfill

{\small UNIVERSITÀ DEGLI STUDI DI PADOVA \par}
{\small CORSO DI LAUREA IN INFORMATICA (L-31) \par}
{\large Corso di Ingegneria del Software \par}
{\small Anno Accademico 2025/2026 \par}

\vfill
{\Huge \bfseries Piano di Qualifica \par}
\vspace{1cm}

{\Large Redattori: Mihaela-Mariana Romascu, Davide Biasuzzi \par}

\vfill
{\Large \bfseries Gruppo: NightPRO \par}
{\large \href{mailto:swe.nightpro@gmail.com}{swe.nightpro@gmail.com} \par}

\vfill

{\large Data: 2025-12-18 \par}
{\Large Versione: 0.2 \par}

\end{titlepage}

% -------------------------------------------------------------------
% TABELLA VERSIONI
% -------------------------------------------------------------------
\newpage
\section*{Tabella delle Versioni}
\addcontentsline{toc}{section}{Tabella delle Versioni}

\begin{center}
\renewcommand{\arraystretch}{1.5}
\begin{tabularx}{\textwidth}{@{}p{0.1\textwidth}p{0.15\textwidth}p{0.25\textwidth}X>{\centering\arraybackslash}p{0.15\textwidth}@{}}
\toprule
\textbf{Versione} & \textbf{Data} & \textbf{Autore/i} & \textbf{Descrizione} & \textbf{Verificatore} \\
\midrule
0.1 & 2025-12-05 & Mihaela-Mariana Romascu & Stesura iniziale del documento, definizione metriche di processo e prodotto. & Giovanni Ponso\\
0.2 & 2025-12-18 & Davide Biasuzzi & Aggiunto paragrafo relativo alla qualità della documentazione & Samuele Perozzo\\
\bottomrule
\end{tabularx}
\end{center}

% -------------------------------------------------------------------
% INDICE
% -------------------------------------------------------------------
\newpage
\tableofcontents
\newpage

% -------------------------------------------------------------------
% INFORMAZIONI GENERALI
% -------------------------------------------------------------------
\section*{Informazioni Generali}
\addcontentsline{toc}{section}{Informazioni Generali}

\subsection*{Componenti del Gruppo}

\begin{table}[h!]
\centering
\renewcommand{\arraystretch}{1.2}
\begin{tabular}{@{}llc@{}}
\toprule
\textbf{Cognome} & \textbf{Nome} & \textbf{Matricola} \\
\midrule
Biasuzzi & Davide & 2111000 \\
Bilato & Leonardo & 2071084 \\
Zanella & Francesco & 2116442 \\
Romascu & Mihaela-Mariana & 2079726 \\
Ogniben & Michele & 2042325 \\
Perozzo & Samuele & 2110989 \\
Ponso & Giovanni & 2000558 \\
\bottomrule
\end{tabular}
\caption{Componenti del gruppo NightPRO.}
\end{table}

% -------------------------------------------------------------------
% INTRODUZIONE
% -------------------------------------------------------------------
\newpage
\section{Introduzione}

Il \textit{Piano di Qualifica} rappresenta il documento strategico per la gestione della qualità all'interno del progetto \textbf{SmartOrder}. Esso stabilisce gli standard, le metodologie di verifica e le metriche quantitative necessarie per valutare sia la qualità del processo di sviluppo che la qualità del prodotto software finale.

\subsection{Scopo del documento}
Questo documento ha lo scopo di definire:
\begin{itemize}
    \item Gli \textbf{obiettivi di qualità} che il gruppo NightPRO si impegna a raggiungere.
    \item Le \textbf{metriche} utilizzate per monitorare l'andamento del progetto (costi, tempi) e la qualità del codice.
    \item Le \textbf{procedure di verifica e validazione} da applicare ai documenti e al software.
    \item I criteri di accettazione per le fasi di avanzamento.
\end{itemize}

L'adozione di questo piano garantisce che il prodotto finale sia conforme ai requisiti espressi nel Capitolato C8 e rispetti gli standard accademici richiesti.

\subsection{Glossario}
Al fine di evitare ambiguità, i termini tecnici o specifici del dominio sono raccolti nel documento \textit{Glossario}.
Nel presente testo, tali termini sono contrassegnati con una "G" in pedice alla loro prima occorrenza (es. Agile{\scriptsize\raisebox{-0.5ex}{G}}).

% --- Riferimenti ---
\section{Riferimenti}

\subsection{Riferimenti normativi}
\begin{itemize}
    \item \href{https://github.com/swenightpro/Documentazione/blob/main/docs/Candidatura/Documentazione%20Interna}{Norme di Progetto V1.0}
    \item \href{https://www.math.unipd.it/~tullio/IS-1/2025/Progetto/C8.pdf}{Capitolato d’appalto C8 - Smart Order}
    \item \href{https://www.math.unipd.it/~tullio/IS-1/2025/Progetto/C8.pdf}{Capitolato d'appalto C8 - Smart Order}
\end{itemize}

\subsection{Riferimenti informativi}
\begin{itemize}
    \item \href{https://github.com/swenightpro/Documentazione/blob/main/docs/RTB/Documentazione%20Esterna}{Glossario}
    \item Analisi dei requisiti [da inserire link]
    \item \href{https://github.com/swenightpro/Documentazione/tree/main/docs/RTB/Documentazione%20Esterna/Verbali%20Esterni}{Verbali esterni}
    \item \href{https://github.com/swenightpro/Documentazione/tree/main/docs/RTB/Documentazione%20Interna/Verbali%20Interni}{Verbali interni}
    \item \href{https://github.com/swenightpro/Documentazione/tree/main/docs/RTB/Documentazione%20Esterna}{Piano di progetto}
\end{itemize}

% -------------------------------------------------------------------
% QUALITÀ DI PROCESSO
% -------------------------------------------------------------------
\newpage
\section{Qualità di Processo}

\subsection{Scopo e Obiettivi}
La qualità di processo riguarda la capacità del gruppo di gestire le attività pianificate rispettando vincoli di budget e tempistiche. Il gruppo NightPRO adotta il ciclo di Deming, noto come PDCA:

\begin{itemize}
    \item \textbf{Plan (Pianificare)}: stabilire gli obiettivi e i processi necessari per fornire risultati in accordo con i risultati attesi.
    \item \textbf{Do (Fare)}: attuare il piano, eseguire i processi, realizzare il prodotto.
    \item \textbf{Check (Verificare)}: monitorare e misurare i processi e il prodotto a fronte delle politiche, degli obiettivi e dei requisiti.
    \item \textbf{Act (Agire)}: adottare azioni per migliorare continuamente le prestazioni dei processi.
\end{itemize}

\subsection{Parametri del Progetto}
I seguenti valori costituiscono la baseline per il calcolo delle metriche:

\begin{table}[h!]
\centering
\renewcommand{\arraystretch}{1.3}
\begin{tabularx}{\textwidth}{lXc}
\toprule
\textbf{Parametro} & \textbf{Descrizione} & \textbf{Valore} \\
\midrule
\textbf{BAC} (Budget At Completion) & Costo totale preventivato del progetto (v0.4) & € 12.850,00 \\
\textbf{Ore Totali} & Monte ore complessivo disponibile & 630 ore \\
\textbf{Componenti} & Numero di membri del gruppo & 7 \\
\textbf{Ore medie per componente} & Media ore di lavoro pro capite ($630/7$) & 90 ore \\
\textbf{Scadenza Progetto} & Data prevista per la consegna finale (PB) & 21/03/2026 \\
\bottomrule
\end{tabularx}
\caption{Parametri fondamentali del progetto}
\end{table}


\subsection{Metriche di Gestione del Progetto}
Per il monitoraggio dell'avanzamento, il gruppo utilizza la metodologia Earned Value Management.

\subsubsection{Definizione Metriche}
\begin{align*}
\text{AC (Actual Cost)} &: \text{Costo realmente sostenuto fino alla data corrente.} \\[4pt]
\text{EV (Earned Value)} &: EV = BAC \cdot (\%\text{ completamento attività}) \\[4pt]
\text{PV (Planned Value)} &: PV = BAC \cdot (\%\text{ lavoro pianificato alla data corrente}) \\[4pt]
\text{CV (Cost Variance)} &: CV = EV - AC \quad (\text{Differenza di costo}) \\[4pt]
\text{SV (Schedule Variance)} &: SV = EV - PV \quad (\text{Differenza di tempi}) \\[4pt]
\text{CPI (Cost Performance Index)} &: CPI = \frac{EV}{AC} \quad (\text{Efficienza economica}) \\[4pt]
\text{SPI (Schedule Performance Index)} &: SPI = \frac{EV}{PV} \quad (\text{Efficienza temporale}) \\[4pt]
\text{EAC (Estimate At Completion)} &: EAC = AC + \frac{BAC - EV}{CPI} \quad (\text{Stima costo a finire})
\end{align*}

\subsubsection{Valori di Accettazione}
\begin{table}[h!]
\centering
\renewcommand{\arraystretch}{1.3}
\begin{tabularx}{\textwidth}{lXcc}
\toprule
\textbf{Metrica} & \textbf{Descrizione} & \textbf{Range Ottimale} & \textbf{Accettabile} \\
\midrule
\textbf{CV} & Cost Variance & $\ge 0$ & $\ge -5\%$ BAC \\
\textbf{SV} & Schedule Variance & $\ge 0$ & $\ge -5\%$ PV \\
\textbf{CPI} & Cost Performance Index & $\ge 1.00$ & $0.95 \le CPI \le 1.05$ \\
\textbf{SPI} & Schedule Performance Index & $\ge 1.00$ & $0.95 \le SPI \le 1.05$ \\
\bottomrule
\end{tabularx}
\caption{Soglie di accettazione per le metriche EVM}
\end{table}

\newpage


% -------------------------------------------------------------------
%  QUALITÀ DI PRODOTTO
% -------------------------------------------------------------------
\section{Qualità di Prodotto}

Il gruppo NightPRO fa riferimento allo standard \textbf{ISO/IEC 25010} per definire le caratteristiche di qualità del prodotto software (es. Manutenibilità, Usabilità, Efficienza, Sicurezza).

\subsection{Metriche di Qualità del Software (Statica)}
Queste metriche mirano a garantire la manutenibilità e la qualità del codice sorgente prodotto.

\begin{align*}
\text{CX (Cyclomatic Complexity)} &: CX = E - N + 2P \quad (\text{Complessità del flusso di controllo}) \\[4pt]
\text{DI (Depth of Inheritance)} &: \text{Profondità massima dell'albero di ereditarietà} \\[4pt]
\text{CD (Code Duplication)} &: \text{Percentuale di codice duplicato sul totale} \\[4pt]
\text{CCh (Code Churn)} &: \text{Frequenza e volume delle modifiche al codice nel tempo} \\[4pt]
\text{TD (Technical Debt)} &: \text{Tempo stimato per il refactoring del codice problematico}
\end{align*}

\begin{table}[H]
\centering
\renewcommand{\arraystretch}{1.2}
\begin{tabularx}{\textwidth}{lXcc}
\toprule
\textbf{Metrica} & \textbf{Descrizione} & \textbf{Ottimale} & \textbf{Accettabile} \\
\midrule
\textbf{CX} & Complessità Ciclomatica (per metodo) & $\le 10$ & $\le 15$ \\
\textbf{DI} & Profondità Ereditarietà & $\le 4$ & $\le 6$ \\
\textbf{CD} & Duplicazione Codice & $0\%$ & $\le 5\%$ \\
\textbf{CCh} & Code Churn (linee modificate/settimana) & Basso & Medio \\
\textbf{TD} & Debito Tecnico & $\le 2$ gg & $\le 5$ gg \\
\bottomrule
\end{tabularx}
\caption{Metriche di Qualità del Codice}
\end{table}


\subsection{Metriche di Qualità del Software (Dinamica)}
Per garantire la correttezza del software, vengono misurati i livelli di copertura dei test (\textit{Code Coverage}).

\begin{align*}
\text{CC (Code Coverage)} &: \frac{\text{Linee di codice eseguite dai test}}{\text{Linee totali di codice}} \cdot 100\% \\[6pt]
\text{BC (Branch Coverage)} &: \frac{\text{Rami decisionali eseguiti}}{\text{Rami totali}} \cdot 100\%
\end{align*}

\begin{table}[h!]
\centering
\renewcommand{\arraystretch}{1.2}
\begin{tabularx}{\textwidth}{lXc}
\toprule
\textbf{Indicatore} & \textbf{Descrizione} & \textbf{Target (PB)} \\
\midrule
CC & Copertura del codice (Linee) & $\geq 80\%$ \\
BC & Copertura dei rami (Branch) & $\geq 70\%$ \\
\bottomrule
\end{tabularx}
\caption{Metriche di Verifica dinamica}
\end{table}

% -------------------------------------------------------------------
%  QUALITÀ DELLA DOCUMENTAZIONE
% -------------------------------------------------------------------
\newpage
\section{Qualità della Documentazione}

La documentazione accompagna il prodotto software e deve garantire accessibilità e comprensione a tutti gli stakeholder. La qualità della documentazione è verificata attraverso un insieme coordinato di strumenti automatizzati che controllano la leggibilità linguistica, la correttezza grammaticale e la sintassi \LaTeX.

\subsection{Metriche di Leggibilità e Correttezza Linguistica}

Per valutare la qualità dei documenti redatti in italiano, il gruppo utilizza un insieme di indici automatizzati, ognuno specifico per un aspetto della qualità documentale.

\subsubsection{Indice Gulpease}

L'indice Gulpease è calibrato specificamente per la lingua italiana e valuta la leggibilità di un testo basandosi sulla lunghezza delle parole e delle frasi. Un indice elevato indica un testo più facilmente comprensibile.

\[
G = 89 + \frac{300 \cdot (\text{numero frasi}) - 10 \cdot (\text{numero lettere})}{\text{numero parole}}
\]

L'indice viene calcolato automaticamente su tutti i file \texttt{.tex} presenti nel repository, escludendo da tale calcolo i comandi \LaTeX{}, le formule matematiche e i commenti. La verifica è eseguita mediante lo script \texttt{check\_gulpease.py}, che genera un report CSV contenente il valore Gulpease per ogni documento.

\subsubsection{Controllo Grammaticale e Linguistico (LanguageTool)}

LanguageTool è uno strumento di verifica grammaticale che identifica errori ortografici, grammaticali e stilistici nei testi in italiano. La verifica automatica:

\begin{itemize}
    \item Rileva errori di ortografia, concordanza verbale e nominale, punteggiatura;
    \item Segnala problemi di stile e ridondanze lessicali;
    \item Esclude dal controllo i nomi propri del gruppo, i termini tecnici presenti nel glossario e gli identificativi tecnici (branch, commit, tool name, ecc.).
\end{itemize}

Lo script \texttt{check\_languagetool.py} esegue questa verifica e genera un report Markdown dettagliato segnalando ogni errore rilevante con il numero della regola violata, il tipo di errore, il messaggio esplicativo e i suggerimenti di correzione.

\subsubsection{Controllo Sintattico \LaTeX{} (ChkTeX)}

ChkTeX è un linter specifico per file \LaTeX{} che verifica la correttezza sintattica e stilistica del codice sorgente. Controlla:

\begin{itemize}
    \item Correttezza dei comandi \LaTeX;
    \item Bilanciamento delle parentesi e delle graffe;
    \item Uso corretto degli spazi prima e dopo i comandi;
    \item Altre convenzioni stilistiche di buona pratica \LaTeX.
\end{itemize}

Lo script \texttt{check\_chktex.py} esegue ChkTeX su tutti i file \texttt{.tex} e filtra alcuni warning innocui (ad esempio riferimenti a comandi personalizzati o numeri di versione non seguiti da unità). Il report viene salvato in formato JSON.

\subsection{Soglie di Accettazione e Target}

Le seguenti tabelle definiscono i target di qualità e le soglie di accettazione per le metriche documentali:

\begin{table}[h!]
\centering
\renewcommand{\arraystretch}{1.3}
\begin{tabularx}{\textwidth}{lXcc}
\toprule
\textbf{Metrica} & \textbf{Descrizione} & \textbf{Ottimale} & \textbf{Accettabile} \\
\midrule
\textbf{Gulpease} & Indice di leggibilità per testi in italiano & $\ge 60$ & $\ge 45$ \\
\bottomrule
\end{tabularx}
\caption{Soglie per l'indice Gulpease}
\end{table}

\begin{table}[h!]
\centering
\renewcommand{\arraystretch}{1.3}
\begin{tabularx}{\textwidth}{lXc}
\toprule
\textbf{Indicatore} & \textbf{Descrizione} & \textbf{Target (PB)} \\
\midrule
Errori LanguageTool & Numero massimo di segnalazioni rilevanti per file & $\le 10$ \\
Errori ChkTeX & Numero massimo di warning significativi per file & $\le 5$ \\
\bottomrule
\end{tabularx}
\caption{Soglie per gli errori linguistici e \LaTeX}
\end{table}

\subsection{Grafici}

\subsubsection{Indice Gulpease}
\begin{figure}[H]
    \centering
    \includegraphics[width=0.8\linewidth]{assets/gulpease.png}
    \caption{Andamento indice Gulpease}
    \label{fig:gulpease}
\end{figure}

\subsubsection{LanguageTool}
\begin{figure}[H]
    \centering
    \includegraphics[width=0.8\linewidth]{assets/lt.png}
    \caption{Andamento errori LanguageTool}
    \label{fig:languagetool}
\end{figure}

\subsubsection{ChkTeX}
\begin{figure}[H]
    \centering
    \includegraphics[width=0.8\linewidth]{assets/chktex.png}
    \caption{Andamento errori ChkTeX}
    \label{fig:chktex}
\end{figure}


\subsection{Automazione e Verifica Continua}

Le verifiche di qualità della documentazione sono automatizzate e integrate nel processo di CI/CD tramite il workflow GitHub \texttt{quality\_checks.yml}. Questo workflow:

\begin{enumerate}
    \item Si attiva automaticamente dopo il completamento della compilazione \LaTeX{} (workflow \textit{Build LaTeX documents});
    \item Può essere eseguito manualmente in qualsiasi momento tramite \textit{workflow dispatch};
    \item Esegue in sequenza i tre script di verifica (Gulpease, LanguageTool, ChkTeX);
    \item Genera un report Markdown consolidato accessibile come artifact;
    \item Commenta automaticamente i pull request con i risultati della verifica.
\end{enumerate}

La verifica è non-bloccante per i build, ovvero gli errori rilevati non impediscono il progresso della pipeline, tuttavia vengono tracciati e riportati nel report per consentire al gruppo di monitorare gli standard di qualità e di correggere sistematicamente i problemi identificati.

\end{document}