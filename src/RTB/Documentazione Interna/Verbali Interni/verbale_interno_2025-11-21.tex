\documentclass[a4paper, 11pt, oneside]{scrartcl} % Classe KOMA-Script

% --- Pacchetti Fondamentali ---
\usepackage[utf8]{inputenc}      % Codifica UTF-8
\usepackage[T1]{fontenc}         % Font encoding moderno
\usepackage[italian]{babel}      % Lingua italiana 
\usepackage{lmodern}             % Font "Latin Modern"

% --- Grafica e Layout ---
\usepackage{graphicx}            % Per includere immagini
\usepackage{currfile}
\graphicspath{{../../../assets/}}

\usepackage[a4paper, top=2.5cm, bottom=3cm, left=2.5cm, right=2.5cm]{geometry} % Margini
\usepackage{fancyhdr}            % Per header e footer personalizzati
\usepackage{microtype}           % Migliora la tipografia
\usepackage[svgnames]{xcolor}    % Colori

% --- Utility ---
\usepackage{booktabs}            % Tabelle più professionali
\usepackage{enumitem}            % Per personalizzare liste
\usepackage{hyperref}            % Rende i link cliccabili
\hypersetup{
    colorlinks=true,
    linkcolor=DarkBlue,
    filecolor=DarkBlue,     
    urlcolor=DarkBlue,
    citecolor=DarkBlue,
    pdftitle={Verbale Interno 2025-11-21 - NightPRO},
    pdfauthor={Gruppo NightPRO},
}

% ===================================================================
%  IMPOSTAZIONE HEADER E FOOTER
% ===================================================================
\pagestyle{fancy}
\fancyhf{} % Pulisce tutti i campi
\fancyhead[L]{NightPRO - Progetto Ingegneria del Software}
\fancyhead[R]{Anno Accademico 2025/2026}
\fancyfoot[C]{\thepage} % Numero di pagina al centro in basso
\renewcommand{\headrulewidth}{0.4pt} % Linea sottile sotto l'header
\renewcommand{\footrulewidth}{0pt}

% ===================================================================
%  INIZIO DEL DOCUMENTO
% ===================================================================
\begin{document}

% -------------------------------------------------------------------
%  SEZIONE: intestazione_titolo.tex
% -------------------------------------------------------------------
\thispagestyle{empty}
\begin{titlepage}
    \centering
    
\begin{figure}
    \centering
    \includegraphics[width=0.4\textwidth]{logo.png}
\end{figure}

    \vfill
    
    {\small UNIVERSITÀ DEGLI STUDI DI PADOVA \par}
    {\small CORSO DI LAUREA IN INFORMATICA (L-31) \par}
    \vspace{0.5cm}
    {\large Corso di Ingegneria del Software \par}
    {\small Anno Accademico 2025/2026 \par}
    
    \vfill
    
    {\Huge \bfseries Verbale di Riunione \par}
    
    \vspace{1cm}
    
    {\Large \itshape Verbale Interno del 21 Novembre 2025 \par} 
    
    \vfill
    
    {\Large \bfseries Gruppo: NightPRO \par}
    \vspace{0.5cm}
    {\large \href{mailto:swe.nightpro@gmail.com}{swe.nightpro@gmail.com} \par}
    
    \vfill
 
    {\large Data: 2025-11-21 \par}

\end{titlepage}

% -------------------------------------------------------------------
%  SEZIONE: indice.tex
% -------------------------------------------------------------------
\newpage
\tableofcontents % Genera l'indice
\pagestyle{fancy} % Riattiva lo stile di pagina da qui in poi

% -------------------------------------------------------------------
%  SEZIONE: informazioni.tex
% -------------------------------------------------------------------
\newpage
\section{Informazioni Generali}

\subsection{Componenti del Gruppo}
Elenco dei membri del gruppo di lavoro NightPRO.
\begin{table}[h!]
\centering
\begin{tabular}{@{}llc@{}}
\toprule
\textbf{Cognome} & \textbf{Nome} & \textbf{Matricola} \\
\midrule
Biasuzzi & Davide & 2111000 \\
Bilato & Leonardo & 2071084 \\
Zanella & Francesco & 2116442 \\
Romascu & Mihaela-Mariana & 2079726 \\
Ogniben & Michele & 2042325 \\
Perozzo & Samuele & 2110989 \\
Ponso & Giovanni & 2000558 \\
\bottomrule
\end{tabular}
\caption{Componenti del Gruppo NightPRO.}
\end{table}

\subsection{Dettagli Riunione}
\begin{itemize}
    \item \textbf{Data:} 2025-11-21
    \item \textbf{Ora:} 09:00 - 11:10
    \item \textbf{Luogo:} Google Meet
    \item \textbf{Partecipanti:} Tutti i membri del gruppo
    \item \textbf{Redatto da:} Davide Biasuzzi
    \item \textbf{Verificato da:} ...
    \item \textbf{Versione:} 1.0
\end{itemize}


% -------------------------------------------------------------------
%  SEZIONE: odg.tex (Ordine del Giorno)
% -------------------------------------------------------------------
\newpage
\section{Ordine del Giorno (Agenda)}
\begin{itemize}
    \item[1.] Gestione della Documentazione (Pianificazione, Preventivo, Consuntivo)
    \item[2.] Suddivisione del lavoro e definizione dei periodi
    \item[3.] Dubbi sul versionamento per RTB
    \item[4.] Definizione Ruoli
\end{itemize}

% -------------------------------------------------------------------
%  SEZIONE: diario.tex (Diario della riunione)
% -------------------------------------------------------------------
\newpage
\section{Diario della Riunione}

\subsection{Gestione della Documentazione}
La discussione si è concentrata sulla stesura e organizzazione della documentazione relativa alla pianificazione economica e temporale.

\subsubsection*{Pianificazione, Preventivo e Consuntivo}
Il gruppo ha definito le linee guida per l'inserimento di questi elementi nei documenti:
\begin{itemize}
    \item \textbf{Pianificazione:} Serve a indicare gli obiettivi e le attività stabilite all'inizio di ogni periodo. Deve essere redatta a inizio periodo.
    \item \textbf{Preventivo:} Rappresenta la stima delle ore e dei costi previsti per ciascun ruolo prima di iniziare il lavoro. Anch'esso va redatto a inizio periodo.
    \item \textbf{Consuntivo:} È il resoconto finale di quanto effettivamente svolto, includendo le ore reali impiegate, i costi effettivi ed eventuali rischi o problemi emersi.
\end{itemize}
È stato deciso di raggruppare tutto ciò che riguarda uno specifico periodo all'interno della stessa sezione del documento, per mantenere coerenza e leggibilità. Inoltre, si ritiene utile giustificare testualmente eventuali scostamenti (in eccesso o difetto) tra le ore preventivate e quelle effettive.

\subsection{Suddivisione del Lavoro e Periodi}
Sono stati definiti i periodi di lavoro e assegnati i relativi task di stesura:
\begin{itemize}
    \item \textbf{Studio preliminare capitolati e introduzione pianificazione:} Assegnato a Giovanni Ponso.
    \item \textbf{Periodo 1 (03/11 - 09/11):} Assegnato a Mihaela-Mariana Romascu.
    \item \textbf{Periodo 2 (10/11 - 16/11):} Assegnato a Michele Ogniben.
    \item \textbf{Periodo 3 (17/11 - 23/11):} Assegnato a Samuele Perozzo.
    \item \textbf{Periodo 4 (24/11 - 30/11) - Pianificazione e Preventivo:} Assegnato a Francesco Zanella.
\end{itemize}

\subsection{Dubbi sul Versionamento RTB}
È emersa una problematica relativa al versionamento dei documenti "Norme di Progetto" e "Glossario" in vista della Revisione dei Requisiti (RTB). Poiché la versione 1.0 è già stata presentata in fase di Candidatura, il gruppo si interroga su come gestire la numerazione per la consegna RTB.
\textbf{Decisione:} Sarà necessario chiedere chiarimenti al docente in merito.

\subsection{Definizione Ruoli}
Sono stati stabiliti i ruoli da assumere a partire dalla prossima settimana:
\begin{itemize}
    \item \textbf{Amministratore di Sistema:} Michele Ogniben
    \item \textbf{Programmatore:} N/A (Nessuno assegnato per questa fase)
    \item \textbf{Progettista:} Francesco Zanella, Mihaela-Mariana Romascu
    \item \textbf{Analista:} Davide Biasuzzi, Samuele Perozzo
    \item \textbf{Verificatore:} Leonardo Bilato
    \item \textbf{Responsabile:} Giovanni Ponso
\end{itemize}

\subsection{Varie ed Eventuali}
\begin{itemize}
    \item Il gruppo procederà dalla prossima settimana con la continuazione dell'Analisi dei Requisiti.
    \item La stesura del "Diario di bordo" è affidata a Giovanni Ponso.
\end{itemize}

% -------------------------------------------------------------------
%  SEZIONE: decisioni.tex (Decisioni prese)
% -------------------------------------------------------------------
\newpage
\section{Decisioni Prese}

\begin{enumerate}
    \item Si giustificheranno testualmente gli scostamenti orari nel consuntivo.
    \item Assegnata la stesura dei vari periodi ai membri del gruppo (vedi sezione To-Do).
    \item Definiti i ruoli a rotazione per la prossima settimana.
    \item Si procederà a chiedere al docente chiarimenti sul versionamento dei documenti già presentati in candidatura.
\end{enumerate}

% -------------------------------------------------------------------
%  SEZIONE: todo.tex (Attività da svolgere)
% -------------------------------------------------------------------
\newpage
\section{Attività da Svolgere (To-Do)}

\begin{table}[h!]
\centering
\begin{tabular}{@{}lll@{}}
\toprule
\textbf{Attività} & \textbf{Assegnatario/i} & \textbf{Scadenza} \\
\midrule
Stesura Verbale Riunione & Davide Biasuzzi & 2025-11-21 \\
Studio prelim. capitolati e intro pianificazione & Giovanni Ponso & 2025-11-23 \\
Stesura Periodo 1 (03/11 - 09/11) & Mihaela-Mariana Romascu & 2025-11-23 \\
Stesura Periodo 2 (10/11 - 16/11) & Michele Ogniben & 2025-11-23 \\
Stesura Periodo 3 (17/11 - 23/11) & Samuele Perozzo & 2025-11-23 \\
Diario di bordo & Giovanni Ponso & 2025-11-24 \\
Stesura Periodo 4 (Pianificazione/Preventivo) & Francesco Zanella & 2025-11-25 \\
Continuazione Analisi dei Requisiti & Biasuzzi, Perozzo & 2025-11-28 \\
Chiedere al docente gestione versionamento RTB & Tutto il gruppo & Appena possibile \\
\bottomrule
\end{tabular}
\caption{Riepilogo task assegnati.}
\end{table}
\end{document}